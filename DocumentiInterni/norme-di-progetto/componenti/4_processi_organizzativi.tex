\documentclass[../norme_di_progetto.tex]{subfiles}

\begin{document}

\subsection{Processi di coordinamento}

    \subsubsection{Scopo}
    In questo paragrafo vengono raccolte le modalità di coordinamento adottate dal gruppo per le comunicazioni e gli incontri  con i vari soggetti coinvolti durante tutta la vita del prodotto. Si svolge questa attività per avere dei canali d'interazione comuni e non escludere nessuno o rischiare di perdere informazioni importanti. Rispetto agli anni passati, questo processo assume un ruolo di assoluta importanza, date le circostanze di questo periodo nel quale è impossibile incontrasi di persona.

    \subsubsection{Comunicazione}
    Durante lo svolgimento di questo progetto verranno usati più strumenti per la comunicazione a seconda del soggetto interessato. I soggetti divisi per ruolo sono:
    \begin{itemize}
        \item Proponente: \emph{Gregorio Piccoli} rappresentante dell'azienda \emph{Zucchetti S.p.A.};
        \item Committente: prof. \emph{Vardanega Tullio} e prof. \emph{Cardin Riccardo};
        \item \emph{SpaghettiCode}: tutti i membri del gruppo;
        \item Altri gruppi: attualmente i gruppi candidati allo stesso capitolato sono Gruppo 5,10 e 15.
    \end{itemize}
    Tutte le comunicazioni saranno svolte per via telematica, data l'impossibilità di incontrarsi di persona.
        \paragraph{Interna}
        I membri del gruppo si sono accordati per usare Discord come canale principale per le comunicazioni e Telegram come canale secondario. Su Discord sono stati creati dei canali appositi divisi a seconda delle tematiche che vengono trattate (es. generale, note-risorse, incontri, gestione-gruppo), inoltre ogni ruolo ha a disposizione un canale apposito nel quale scrivere tutto ciò che riguarda i loro compiti. Il gruppo di Telegram viene usato per comunicazioni molto informali e più istantanee.
        \paragraph{Esterna}
        Le comunicazioni con gli altri gruppi dello stesso capitolato avverranno tramite un gruppo Telegram.
        Le comunicazioni con i committenti avverranno tramite video-chiamate con Zoom e messaggi di posta elettronica.
        Le comunicazioni con il proponente avverranno tramite video-chiamate su Zoom, Skype e messaggi di posta elettronica.
        \paragraph{Riunioni}
        Tutte le riunioni saranno svolte tramite i canali di comunicazione scelti per il soggetto interessato. 
        \paragraph{Strumenti}
        \begin{itemize}
            \item Discord: applicazione VoIP e di messaggistica istantanea, facile da usare e versatile;
            \item Telegram: servizio di messaggistica istantanea;
            \item Zoom: servizio di teleconferenze;
            \item Skype: software di messaggi istantanea e VoIP.
        \end{itemize}

\subsection{Processi di pianificazione}
    \subsubsection{Scopo}
    Nei prossimi paragrafi si descriverà come il gruppo \emph{SpaghettiCode} intende lavorare. Si tratteranno i ruoli e la loro divisione tra i membri, i compiti di ciascun membro e l'assegnazione di essi. Prendendo come esempio da seguire lo standard ISO/IEC 12207, il processo di pianificazione sarà strutturato nel seguente modo:
    \begin{itemize}
        \item Ruoli con relativi compiti;
        \item Assegnazione ruoli;
        \item Divisione del lavoro.
    \end{itemize}
    \subsubsection{Ruoli di progetto}
        \paragraph{Responsabile}
        Il \textbf{responsabile} rappresenta il progetto ed è il punto di riferimento per le comunicazioni con il committente. Per poter pianificare ed anticipare l'evoluzione del progetto deve possedere capacità tecniche e delle competenze pregresse, deve essere in grado di gestire le risorse e tracciare i progressi. Ha la responsabilità di scelta e approvazione su gran parte del progetto e partecipa per tutta la durata di esso.\\
        In particolare, ha il compito di:
        \begin{itemize}
            \item Redigere l'organigramma;
            \item Redigere il Piano di Progetto;
            \item Coordinare i membri del gruppo, le attività e le risorse a disposizione;
            \item Gestire le criticità;
            \item Approvare i vari documenti;
            \item Approvare l'offerta del committente.
        \end{itemize}

        \paragraph{Amministratore}
        L'\textbf{\glossario{amministratore}} è responsabile dell'efficienza e dell'operatività dell'ambiente di lavoro, deve assicurarsi che le risorse siano sempre presenti e operanti.\\
        In particolare, ha il compito di:
        \begin{itemize}
            \item Gestire il controllo della configurazione del prodotto;
            \item Gestire il versionamento;
            \item Gestire la documentazione del progetto;
            \item Redigere le Norme di Progetto;
            \item Collaborare alla redazione del Piano di Progetto;
            \item Redazione e attuazione di piani e procedure di Gestione della Qualità;
            \item Risolvere i problemi legati alla gestione dei processi.
        \end{itemize}
        
        \paragraph{Analista}
        L'\textbf{\glossario{analista}} ha una notevole esperienza professionale e vasta conoscenza del dominio del problema, si occupa di esporre il problema in maniera chiara con un linguaggio simile a quello usato dal proponente. Ci possono essere più \textbf{analisti} contemporaneamente, il loro lavoro ha un grande impatto sulla riuscita del progetto, ma sono generalmente pochi e non seguono il progetto fino alla sua fine.\\
        In particolare, ha il compito di:
        \begin{itemize} 
            \item Redigere lo Studio di Fattibilità;
            \item Redigere l'Analisi dei Requisiti.
        \end{itemize}

        \paragraph{Progettista}
        Il \textbf{\glossario{progettista}} è una persona con competenze tecniche e tecnologiche avanzate e con un'ampia esperienza professionale. Si occupa dello sviluppo della soluzione al problema presentato tramite le attività di progettazione, spesso assumendo anche responsabilità di scelta e gestione. Possono essercene di più contemporaneamente, ma sono comunque pochi e possono seguire il progetto fino alla manutenzione.\\
        In particolare, ha il compito di:
        \begin{itemize}
            \item Redigere la Specifica Tecnica;
            \item Redigere la Definizione di Prodotto;
            \item Redigere la parte programmatica del Piano di Qualifica.
        \end{itemize}

        \paragraph{Programmatore}
        Il \textbf{\glossario{programmatore}} ha competenze tecniche specifiche, ma responsabilità limitate, che si occupa di implementare la soluzione trovata dal \textbf{progettista} tramite attività di codifica del prodotto e dei test di ausilio alla verifica. Rimane a lungo all'interno del progetto, partecipando anche alla manutenzione.\\
        In particolare, ha il compito di:
        \begin{itemize}
            \item Implementare la Specifica Tecnica tramite codifica;
            \item Implementare i test d'ausilio necessari per l'esecuzione delle prove di verifica e validazione.
        \end{itemize}

        \paragraph{Verificatore}
        Il \textbf{\glossario{verificatore}} ha competenze tecniche, esperienze di progetto e conoscenza delle norme, oltre che a capacità di giudizio e relazione. Si occupa di attività di verifica e validazione, partecipa all'intero ciclo di vita assicurandosi che quanto fatto sia conforme alle attese. Illustra nel Piano di Qualifica l'esito e la completezza delle verifiche e delle prove effettuate.\\
        In particolare, ha il compito di:
        \begin{itemize}
            \item Esaminare i prodotti in fase di revisione tramite le tecniche e gli strumenti descritti nelle Norme di Progetto;
            \item Segnalare eventuali errori o modifiche necessari ai diretti interessati, in modo che possano correggerli.
        \end{itemize}

    \subsubsection{Assegnazione dei ruoli}
    I vari ruoli verranno assegnati ai membri del gruppo a rotazione ad ogni raggiungimento di una scadenza terminale. Ogni membro dovrà ricoprire più ruoli durante l'intero ciclo di sviluppo, per un periodo significativo, abbastanza lungo da non interrompere la continuità delle attività in corso.
    Inizialmente i ruoli sono stati assegnati casualmente, perché nessuno dei membri del gruppo aveva conoscenze pregresse. Si prevede di fare scelte più mirate alla prossima rotazione, tenendo in considerazione le conoscenze acquisite nell'ultimo periodo e gli interessi sviluppati da parte dei membri verso le tecnologie usate.
    \subsubsection{Assegnazione dei compiti}
    Ogni membro dovrà svolgere i suoi compiti in base al ruolo assegnatogli. Per tenere traccia di cosa è stato fatto e di cosa bisogna fare si è deciso di usare il sistema di Issue Tracking offerto da GitHub. Ogni membro potrà creare delle Issue con il lavoro da svolgere, dovranno essere piccoli task che potranno essere svolti anche da una sola persona. Se il lavoro da svolgere non è strettamente legato ad un ruolo preciso sarà prenotabile da qualsiasi membro.
    \subsubsection{Metriche}
    Durante i meeting tra i membri del gruppo ci si dovrà accordare con delle scadenze e verranno fissate delle \glossario{Milestone} su GitHub. Tramite le Milestone si potrà conoscere l'andamento dei lavori e la percentuale di completamento, infatti, ogni Issue creata dovrà essere assegnata ad una Milestone. 
    \subsubsection{Gestione dei rischi}
    Per gestire i rischi, il gruppo si è accordato nell'usare un sistema di tag tramite Label delle Issue. Alla creazione di una nuova Issue sarà possibile assegnare un livello di priorità, che andrà da minore, normale, importante e critica a seconda dell'urgenza con la quale deve essere svolto un compito. Così ogni membro del gruppo potrà vedere se ci sono attività che necessitano più attenzione rispetto ad altre e potrà intervenire, assegnandosi la Issue oppure sollecitando i diretti interessati.

\subsection{Formazione}
    \subsubsection{Scopo}
    Lo scopo della formazione è quello di uniformare le capacità tecniche e le conoscenze tra i vari membri del gruppo in modo da poter lavorare e comunicare in sintonia.
    \subsubsection{Descrizione}
    Per ogni membro di \emph{SpaghettiCode} è prevista la formazione tramite studio autonomo delle varie tecnologie che vengono adoperate o che sono state richieste da Zucchetti S.p.A. durante la presentazione del capitolato e durante gli incontri successivi. In caso di difficoltà il gruppo è disponibile a fare formazione tramite incontri su Discord.
    \subsubsection{Aspettative}
    Ci si aspetta che tutti i membri del gruppo acquisiscano familiarità con le seguenti tecnologie:
    \begin{itemize}
        \item \LaTeX;
        \item Git e GitHub;
        \item JavaScript;
        \item Libreria D3.js.
    \end{itemize}

\end{document}
\documentclass[../norme-di-progetto.tex]{subfiles}

\begin{document}

\subsection{Processi di coordinamento}

    \subsubsection{Scopo}
    Il \emph{responsabile di progetto} deve raccogliere nel documento \emph{Piano di Progetto} le seguenti attività:
    \begin{itemize}
        \item identificare i possibili rischi;
        \item creare un piano per gestire i rischi individuati;
        \item definire un modello di sviluppo;
        \item definire un piano di lavoro che rispetti le scadenze temporali;
        \item calcolare un preventivo ore/persona per i singoli ruoli;
        \item calcolare un preventivo a finire delle risorse necessare.
    \end{itemize}

    \subsubsection{Descrizione}
    TODO

    \subsubsection{Aspettative}
    TODO

    \subsubsection{Comunicazione}
    Durante lo svolgimento di questo progetto verranno usati più strumenti per la comunicazione a seconda del soggetto interessato. I soggetti divisi per ruolo sono:
    \begin{itemize}
        \item Proponente: Gregorio Piccoli rappresentante dell'azienda Zucchetti SPA;
        \item Commitente: prof. Vardanega Tullio e prof. Cardin Riccardo;
        \item SpaghettiCode: tutti i membri del gruppo;
        \item Altri gruppi: attualmente i gruppi candidati allo stesso capitolato sono Gruppo 5,10 e 15.
    \end{itemize}
    Tutte le comunicazioni saranno svolte per via telematica, data l'impossiblità di incontrarsi di persona.
        \paragraph{Interna}
        I membri del gruppo si sono accordati per usare Discord come canale principale per le comunicazioni e Telegram come canale secondario. Su Discord sono stati creati dei canali appositi divisi a seconda delle tematiche che vengono trattate (es. generale, note-risorse, incontri, gestione-gruppo), inoltre ogni ruolo ha a disposizione un canale apposito nel quale scrivere tutto ciò che riguarda i loro compiti. Il gruppo di Telegram viene usato per comunicazioni molto informali e più istantanee.
        \paragraph{Esterna}
        Le comunicazioni con gli altri gruppi dello stesso capitolato avverranno tramite un gruppo Telegram.
        Le comunicazioni con i committenti avverranno tramite video-chiamate con Zoom e messaggi di posta elettronica.
        Le comunicazioni con il proponente avverranno tramite video-chiamate su Zoom, Skype e messaggi di posta elettronica.
        \paragraph{Riunioni}
        Tutte le riunioni saranno svolte tramite i canali di comunicazione scelti per il soggetto interessato. 
        \paragraph{Strumenti}
        \begin{itemize}
            \item Discord: applicazione VoIP e di messaggistica istantanea, facile da usare e versatile;
            \item Telegram: servizio di messaggistica istantanea;
            \item Zoom: servizio di teleconferenze;
            \item Skype: software di messaggi istantanea e VoIP.
        \end{itemize}

\subsection{Processi di pianificazione}
    \subsubsection{Scopo}
    Nei prossimi paragrafi si descriverà come il gruppo SpaghettiCode intende lavorare. Si tratteranno i ruoli e la loro divisione tra i membri, i compiti di ciascun membro e l'assegnazione di essi. Prendendo come esempio da seguire lo standard ISO/IEC 12207, il processo di pianificazione sarà strutturato nel seguente modo:
    \begin{itemize}
        \item Ruoli;
        \item Assegnazione ruoli;
        \item Divisione compiti.
    \end{itemize}
    \subsubsection{Ruoli di progetto}
        \paragraph{Responsabile}
        Il responsabile rappresenta il progetto ed è il punto di riferimento per le comunicazioni con il committente. Per poter pianificare ed anticipare l'evoluzione del progetto deve possedere capacità tecniche e delle competenze pregresse, deve essere in grado di gestire le risorse e tracciare i progressi. Ha la responsabilità di scelta e approvazione su gran parte del progetto e partecipa per tutta la durata di esso. Redige l'organigramma e il Piano di Progetto e si occupa di approvare i  vari documenti. Approva anche l'offerta e i relativi allegati.
        \paragraph{Amministratore}
        L'amministratore è responsabile dell'efficienza e dell'operatività dell'ambiente di lavoro, deve assicurarsi che le risorse siano sempre presenti e operanti. Ha il compito di gestione del controllo della configurazione del prodotto, del versionamento e della documentazione di progetto. Redige le Norme di Progetto, collabora alla redazione del Piano di Progetto e si occupa della redazione e attuazione di piani e procedure di Gestione per la Qualità. Inoltre risolvere i problemi legati alla gestione dei processi.
        \paragraph{Analista}
        L'analista ha una notevole esperienza professionale e vasta conoscienza del dominio del problema, si occupa di esporre il problema in maniera chiara con un linguaggio simile a quello usato dal proponente. Ci possono essere più analisti contemporaneamente, il loro lavoro ha un grande impatto sulla riuscita del progetto, ma sono generalmente pochi e non seguono il progetto fino alla sua fine. Redige lo Studio di Fattibilità e l'Analisi dei Requisiti.
        \paragraph{Progettista}
        Il progettista è una persona con competenze tecniche e tecnologiche avvanzate e con un'ampia esperienza professionale. Si occupa dello sviluppo della soluzione al problema presentato tramite le attività di progettazione, spesso assumendo anche responsabilità di scelta e gestione. Possono essercene di più contemporaneamente, ma sono comunque pochi e possono seguire il progetto fino alla manutenzione. Redige la Specifica Tecnica, la Definizione di Prodotto e la parte programmatica del Piano di Qualifica.
        \paragraph{Programmatore}
        Il programmatore ha competenze tecniche specifiche, ma responsabilità limitata, che si occupa di implementare la soluzione trovata dal progettista tramite attività di codifica del prodotto e dei test di ausilio alla verifica. Rimane a lungo all'interno del progetto, partecipando anche alla manutenzione.
        \paragraph{Verificatore}
        Il verificatore ha competenze tecniche, esperienze di progetto e conoscenza delle norme, oltre che a capacità di giudizio e relazione. Si occupa di attività di verifica e validazione, partecipa all'intero ciclo di vita assicurandosi che quanto fatto sia conforme alle attese. Illustra nel Piano di Qualifica l'esito e la completezza delle verifiche e delle prove effettuate.
    \subsubsection{Assegnazione dei ruoli}
    TODO
    \subsubsection{Ciclo di vita del ticket}
    TODO
    \subsubsection{Metriche}
    TODO
    \subsubsection{Strumenti}
    TODO
    \subsubsection{Gestione dei rischi}
    TODO

\subsection{Formazione}
    \subsubsection{Scopo}
    Lo scopo della formazione è quello di uniformare le capacità tecniche e le conoscenze tra i vari membri del gruppo in modo da poter lavorare e comunicare in sintonia.
    \subsubsection{Descrizione}
    Per ogni membro di SpaghettiCode è prevista la formazione tramite studio autonomo delle varie tecnologie che vengono adoperate o che sono state richieste da Zucchetti SPA durante la presentazione del capitolato e durante gli incontri successivi. In caso di difficoltà il gruppo è disponibile a fare formazione tramite incontri su Discord.
    \subsubsection{Aspettative}
    Il piano di formazione prevede:
    \begin{itemize}
        \item LATEX;
        \item Git e Github;
        \item JavaScript;
        \item Libreria D3.js.
    \end{itemize}

\end{document}
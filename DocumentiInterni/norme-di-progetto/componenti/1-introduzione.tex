\documentclass[../norme-di-progetto.tex]{subfiles}

\begin{document}

\subsection{Scopo del documento}
Questo documento ha lo scopo di fissare il \glossario{way of working} che tutti i membri del gurppo \emph{SpaghettiCode} dovranno seguire per tutto lo svolgimento del \glossario{progetto}.
Le varie \glossario{attività} presenti all'interno del documento andranno a formare \glossario{processi} conformi con lo standard \emph{ISO/IEC 12207:1995}.
Il documento segue uno sviluppo di tipo incrementale, ciò significa che sarà sottoposto continuamente a modifiche, rimozioni o aggiunte; al completamento di una qualsiasi di queste azioni, tutti i membri del gruppo dovranno essere notificati.

\subsection{Scopo del prodotto}
Il \glossario{capitolato} C4 - "HD Viz" ha come obbiettivo la realizzazione di una \glossario{web application} di visualizzazione di dati con molte dimensioni in grafici che andrannoo a supportare l'utente nella fase \glossario{Exploratory Data Analysis} (EDA).
I dati che saranno visualizzati dovranno essere caricati nel'applicazione web tramite \glossario{file CSV} o prelevati da \glossario{database} interni o esterni.

\subsection{Glossario}
Al fine di eliminare qualsiasi equivocità nei termini presenti all'interno del documento e quindi dell'insorgere di incompresioni, viene fornito il \textsc{Glossario v0.0.1}, documento nel quale vengono definiti i termini che presentano un "G" a pedice.

\subsection{Riferimenti}
\subsubsection{Riferimenti normativi}
\begin{itemize}
    \item \textbf{Capitolato d'appalto C4 - "HD Viz"}: \\ \url{https://www.math.unipd.it/~tullio/IS-1/2020/Progetto/C4.pdf}.
\end{itemize}

\subsubsection{Riferimenti informativi}
\begin{itemize}
    \item \textbf{Standard ISO/IEC 12207:1995}: \\ \url{https://www.math.unipd.it/~tullio/IS-1/2009/Approfondimenti/ISO_12207-1995.pdf};
\end{itemize}


\end{document}
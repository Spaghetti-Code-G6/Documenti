\documentclass[../norme_di_progetto.tex]{subfiles}

\begin{document}

\subsection{Documentazione}

\subsubsection{Scopo}
Lo scopo della documentazione è quello di scrivere in uno o più documenti tutto ciò che è necessario sapere per utilizzare correttamente il prodotto software prodotto alla fine del progetto.

\subsubsection{Descrizione}
La produzione della documentazione è un processo che si svolge durante tutto il corso temporale del progetto. Consiste nello scrivere dei documenti di varia natura (riportati nella sezione "Documenti prodotti"), ognuno dei quali ha una specifica funzione. In questa sezione del documento sono riportate le norme per una buona scrittura della documentazione.

\subsubsection{Aspettative}
Questo processo ha un'aspettativa principale,  ovvero la stesura di documenti, ognuno funzionale al suo scopo, scritti in maniera chiara e che rispetti le regole di seguito definite.

\subsubsection{Ciclo di vita di un documento}
Il ciclo di vita deve avere il seguente ciclo di vita:
\begin{itemize}
    \item \textbf{Creazione del documento}: Il documento viene creato a partire da un template, viene già fornito del registro delle modifiche e degli elenchi di figure e tabelle;
    \item \textbf{Stesura}: Il documento viene scritto da più membri del gruppo, riportando i vari incrementi nel registro delle modifiche;
    \item \textbf{Revisione}: Dopo la stesura il documento viene revisionato da una o più persone che valutano se il documento rispetta o meno le norme decise. È obbligatorio che un redattore non verifichi il suo stesso documento, nel caso in cui il documento sia stato scritto da più persone un redattore può essere anche verificatore a patto che non verifichi una sezione da lui scritta;
    \item \textbf{Approvazione}: Dopo la revisione il \emph{responsabile} da l'approvazione al documento. Da quel momento il documento è considerato completo.
\end{itemize}

\subsubsection{Template}
Il gruppo ha deciso di utilizzare, per il progetto, il linguaggio \LaTeX per la stesura dei documenti. Allo scopo di uniformare i documenti e velocizzare il processo di creazione e strutturazione del documento stesso è stato definito un template su cui ogni documento si basa.

\subsubsection{Documenti prodotti}
I documenti prodotti sono diversi e si dividono in due categorie:
\begin{itemize}
    \item \textbf{Informali}: In questa categoria rientrano i documenti non soggetti a versionamento. In particolare fanno parte di questa categoria i verbali. Questi si dividono in:
    \begin{itemize}
        \item \textbf{Interni}: A questa categoria appartengono i verbali degli incontri interni del gruppo. Riportano in breve dei resoconti degli incontri;
        \item \textbf{Esterni}: Questa, al contrario della precedente, è la categoria a cui appartengono i verbali degli incontri con membri esterni al gruppo, quali potrebbero essere i committenti e/o i proponenti.
    \end{itemize}
    
    \item \textbf{Formali}: In questa categoria rientrano tutti i documenti soggetti a versionamento. I documenti facenti parte di questa categoria regolano l'intera attività del gruppo. Un documento di questa categoria, oltre ad essere soggetto a versionamento, deve essere approvato dal responsabile di progetto. In alcuni casi è possibile che un documento abbia più versioni, in questo caso il responsabile deve dare l'approvazione ad ognuna e si considera quella corrente la più recente tra le versioni approvate.\\
    Come i documenti informali, anche questi si dividono in due categorie:
    \begin{itemize}
        \item \textbf{Interni}: I documenti appartenenti a questa categoria sono destinati ad uso interno del gruppo e non sono quindi di interesse primario per i committenti e il proponente;
        \item \textbf{Esterni}:I documenti appartenenti a questa categoria sono, al contrario, destinati ad essere consegnati al proponente e ai committenti. Verrà consegnata l'ultima versione approvata.
    \end{itemize}
    Di seguito viene riportato un elenco dei documenti formali prodotti:
    \begin{itemize}
        \item \textbf{Norme di progetto}: Documento che racchiude tutte le norme che regolamentano il progetto nella sua interezza. Questo documento rientra nella categoria dei documenti interni;
        \item \textbf{Glossario}: Documento che elenca tutti i termini che, secondo il gruppo, necessitano di una spiegazione esplicita. Questo documento rientra nella categoria dei documenti esterni;
        \item \textbf{Studio di fattibilità}: Documento che raccoglie i singoli studi di fattibilità sui capitolati secondo lo schema descritto in questo documento al paragrafo "Struttura dei documenti". Questo documento rientra nella categoria dei documenti interni;
        \item \textbf{Piano di progetto}: Documento che espone la pianificazione delle attività di progetto previste dal gruppo. In allegato a questo documento, che ricade nella categoria di quelli esterni, viene presentato un preventivo sia degli impegni orari che dei costi;
        \item \textbf{Piano di qualifica}: Documento che espone i criteri di valutazione della qualità del progetto. Questo documento rientra nella categoria dei documenti esterni;
        \item \textbf{Analisi dei requisiti}: Documento che raccoglie tutti i requisiti del progetto e che il prodotto software deve avere. Questo documento rientra nella categoria dei documenti esterni.
  \end{itemize}
  
\end{itemize}

\subsubsection{Directory dei documenti}
Ogni directory contenente un documento prende il nome dal documento stesso e utlizza la convenzione snake\_case. All'interno della directory si trova un file .tex principale con lo stesso nome del documento, anch'esso scritto con la convenzione snake\_case. In caso il documento sia strutturato attraverso la creazione di più sottofile, tutti i sottofile si devono trovare all'interno di una cartella detta componenti, la quale conterrà tutti i sottofile in formato .tex, nominati con una numerazione progressiva seguita dal nome della sezione che quel file rappresenta. Anche qui la convenzione usata è lo snake\_case.

\subsubsection{Struttura dei documenti}
Tutti i documenti prodotti si rifanno ad un template predefinito scritto in formato .tex che ne definisce la struttura e le parti comuni. Ogni documento è caratterizzato, nella prima pagina, che forma il frontespizio, da una serie di elementi. Dall'alto verso il basso ci sono:
\begin{itemize}
    \item Il logo a colori del gruppo;
    \item Il nome del gruppo;
    \item Il contatto e-mail del gruppo;
    \item Il nome del documento;
    \item Una tabella riportante la versione, l'approvatore, i redattori, i verificatori, l'uso e i destinatari;
    \item Una breve descrizione del contenuto del documento.
\end{itemize}
A seguire c'è una pagina che riporta il registro delle modifiche, in cui vengono specificati:
\begin{itemize}
    \item Versione del documento;
    \item Nome di chi ha apportato la modifica;
    \item Ruolo all'interno del gruppo;
    \item Data della modifica;
    \item Descrizione della modifica.
\end{itemize}
Di seguito si trova una sezione in cui c'è l'indice del documento ed eventualmente, se necessarie, una lista di tabelle e una lista di figure.\\
A seguire c'è il corpo del documento, una sezione molto variabile a seconda del documento considerato. Nonostante questa variabilità la struttura della singola pagina rimane fissa ed è così strutturata:
\begin{itemize}
    \item In alto a sinistra ci sono il nome del gruppo e il nome del documento
    \item In alto a destra c'è il logo del gruppo nella versione colorata di nero
    \item Al di sotto di questi due elementi c'è una linea nera che li separa dal contenuto della pagina
    \item Il contenuto della pagina 
    \item Nel piè di pagina si trova, a destra, il numero della pagina corrente e il numero di pagine totali
\end{itemize}
Ai documenti informali, ovvero ai verbali, non viene applicato nessun versionamento, questo fa si che il registro delle modifiche abbia sempre 3 righe: una corrispondente alla stesura, una corrispondente alla verifica e una corrispondente all'approvazione. I verbali hanno la seguente struttura:
\begin{itemize}
    \item Frontespizio;
    \item Registro delle modifiche;
    \item Indice;
    \item Informazioni generali. Le informazioni generali sono formate da: luogo dell'incontro, data dell'incontro, orario dell'incontro, partecipanti;
    \item Ordine del giorno
    \item Resoconto
    \item Conclusione
\end{itemize}

\subsubsection{Norme tipografiche}

\paragraph{Attribuzione del nome}
Per l'attribuzione del nome si segue la convenzione detta snake\_case. La convenzione scelta prevede la scrittura di tutti i nomi con stampatello minuscolo e in caso di multiple parole di separarle con l'underscore.

\paragraph{Stili di testo}
(?)

\paragraph{Elenchi puntati}
Per gli elenchi puntati la convenzione utilizzata è il punto. Nel caso in cui l'elenco sia annidato si usa il trattino alto (-).

\paragraph{Formati di dato}
Le date vengono indicate con il formato:
\begin{center}
    \textbf{[YYYY]-[MM]-[DD]}
\end{center}
Secondo lo standard ISO 8601. In questa notazione YYYY indica l'anno, MM indica il mese e DD indica il giorno.

\paragraph{Sigle}
Le sigle utilizzate sono riportate di seguito.
Per quanto riguarda i documenti, le sigle utilizzate sono:
\begin{itemize}
    \item Analisi dei requisiti: AdR;
    \item Piano di progetto: PdP;
    \item Piano di qualifica: PdQ;
    \item Studio di fattibilità: SdF;
    \item Norme di progetto: NdP;
    \item Verbali interni: VI;
    \item Verbali esterni: VE;
\end{itemize}

Per quanto riguarda le revisioni di progetto le sigle utilizzate sono:
\begin{itemize}
    \item Revisione dei requisiti: RR
    \item Revisione di progettazione: RP
    \item Revisione di qualifica: RQ
    \item Revisione di accettazione: RA
\end{itemize}

\subsubsection{Elementi grafici}

\paragraph{Immagini}
Le immagini devono essere centrate nella pagina e devono avere un'opportuna didascalia.

\paragraph{Grafici UML}
I grafici UML sono inseriti come immagini.

\paragraph{Tabelle}
Ogni tabella, ad eccezione del registro delle modifiche, è accompagnata da una didascalia e deve essere centrata nella pagina.

\subsubsection{Strumenti}

\paragraph{\LaTeX}
Questo strumento è stato scelto dal gruppo per far si che ci fosse uniformità nella stesura dei documenti e al contempo per avere uno strumento che permettesse grande flessibilità nella stesura stessa.

\paragraph{Draw.io}


\subsubsection{Metriche}

\paragraph{Indice gulpease}
Quest'indice riporta il grado di leggibilità di un documento. Viene adottato, per calcolarlo, il sito \url{https://farfalla-project.org/readability_static/} che permette di calcolarlo semplicemente incollando il testo nell'apposita casella dedicata.

\paragraph{Correzione errori ortografici}
Per la correzione di errori ortografici viene utilizzato lo strumento di correzione automatica dell'editor usato per la stesura del documento.




\end{document}
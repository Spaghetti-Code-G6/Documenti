\documentclass[../norme-di-progetto.tex]{subfiles}

\begin{document}
    
\subsection{Fornitura}
\subsubsection{Scopo}
Il processo di fornitura consiste nelle seguenti attività:
\begin{itemize}
    \item Analisi di strumenti e competenze fondamentali e individuazione di rischi e criticità per il completamento del progetto. Questa analisi viene redatta nel documento \textsc{Studio di Fattibilità};
    \item Stabilire ed esporre l'organizzazione del lavoro che il gruppo seguirà per la realizzazione del prodotto; queste possono essere trovate nel \textsc{Piano di Progetto}. Il piano di progetto sarà, come questo documento, ancora non del tutto completo in quanto non è facile pianificare a lungo termine, perciò quella che viene presentata al momento è una versione che verrà poi aggiornata successivamente in base ad eventuali anticipi e/o ritardi sulle scadenze presenti al momento;
    \item Verificare la qualità del materiale elaborato, sia per quanto riguarda i documenti, sia per quanto riguarderà, più avanti nel tempo, le attività di progettazione e verifica. Le linee guida per la gestione della qualità sono descritte nel documento \textsc{Piano Di Qualifica}.
\end{itemize}

\setlength{\parindent}{0pt}Questo processo è composto dalle seguenti fasi:
\begin{itemize}
    \item Avvio;
    \item Preparazione di risposte alle richieste;
    \item Contrattazione;
    \item Pianificazione;
    \item Esecuzione e controllo;
    \item Revisione e valutazione;
    \item Consegna e completamento.
\end{itemize}

\subsubsection{Descrizione}
Vengono qui descritte e trattare tutte norme a cui il gruppo \emph{SpaghettiCode} deve attenersi, con lo scopo di diventare i \glossario{fornitori} del prodotto \emph{HD Viz} del proponente \emph{Zucchetti S.p.A.} e dei committenti \emph{prof. Tullio Vardanega} e \emph{prof. Riccardo Cardin}.

\subsubsection{Aspettative}
Questo processo si pone gli obbiettivi di mantenere un confronto costante con il proponente \emph{Zucchetti S.p.A.} e nello specifico con il referente \emph{dott. Gregorio Piccoli} al fine di stimare le tempistiche di lavoro, verificare in modo continuo quanto prodotto dal gruppo, determinare i requisiti del prodotto e infine chiarire eventuali dubbi.
Successivamente all'avvenuta consegna, il gruppo \emph{SpaghettiCode}, non seguirà la fase di manutenzione del prodotto, salvo eventuali accordi.


\subsubsection{Attività}
\paragraph{Studio di Fattibilità}
Nel documento \textsc{Studio di Fattibilità}, redatto dagli \emph{analisti} dopo aver deciso la prima scelta del gruppo tra i vari progetti disponibili, viene fornita un'analisi generale di tutti i capitolati proposti, e le motivazioni che hanno spinto il gruppo \emph{SpaghettiCode} a proporsi o meno come fornitore di uno specifico capitolato.
Tale documento è strutturato nel seguente modo:
\begin{itemize}
    \item \textbf{Informazioni generali}: informazioni basilari riguardanti il capitolato che comprendono nome, proponente e committente;
    \item \textbf{Descrizione}: breve sintesi delle caratteristiche del prodotto da sviluppare;
    \item \textbf{Finalità del progetto}: descrizione degli obbiettivi del capitolato d'appalto;
    \item \textbf{Tecnologie interessate}: tecnologie richieste dal progetto, comprendenti linguaggi di programmazione e strumenti specifici che il gruppo dovrà utilizzare;
    \item \textbf{Aspetti positivi}: aspetti presi in considerazione dal gruppo a favore di una eventuale scelta del capitolato;
    \item \textbf{Rischi}: aspetti negativi da affrontare individuate dal gruppo in caso di scelta del capitolato;  
    \item \textbf{Conclusioni}: motivazioni che hanno spinto il gruppo a scegliere o meno il capitolato.
\end{itemize}

\paragraph{Piano di Progetto}
Gli \emph{\glossario{amministratori}}, sotto la supervisione del \glossario{responsabile di progetto}, redigono il \textsc{Piano di Progetto}; questo documento andrà redatto e aggiornato per tutta lal durata del progetto ed è strutturato nel seguente modo:
\begin{itemize}
    \item \textbf{Analisi dei rischi}: sezione in cui vengono analizzati i rischi che possono presentarsi nel corso del progetto. Vengono fornite anche le modalità con cui vengono risolti o ridimensionati questi rischi. Un'analisi più esaustiva si trovaerà nel documento \glossario{Analisi dei requisiti};
    \item \glossario{\textbf{Modello di sviluppo}}: sezione in cui viene descritto il \glossario{modello di sviluppo} scelto dal gruppo e le motivazioni che hanno portato a scegliere quel determinato modello;
    \item \textbf{Pianificazione}: vengono descritte e pianificate le attività da eseguire nelle vari fasi del progetto, stabilendo i termini temporali (\glossario{deadlines}) per il loro completamento. Queste deadlines non sono però rigide perchè, come accennato prima, non è facile pianificare sul lungo termine e proprio per questo motivo il piano di progetto sarà sempre soggetto a modifiche e aggiornamenti;
    \item \textbf{Preventivo e consuntivo}: viene stimata la quantità di lavoro necessaria per ogni processo del ciclo di vita del progetto. Viene quindi esposto un preventivo e un successivo consuntivo, entrambi relativi ad un dato periodo.
\end{itemize}

\paragraph{Piano di Qualifica}
Il \textsc{Piano di Qualifica}, redatto dai \glossario{\emph{verificatori}}, contiene le strategie, i processi e le linee guida da adottare per garantire la qualità nei materiali prodotti dal gruppo.
Il documento è strutturato nel seguente modo:
\begin{itemize}
    \item \textbf{Qualità di processo}: sono individuati i processi dagli \glossario{standard di processo}, definiti degli obbiettivi, strategie per attuarli e metriche per controllarli e misurarli;
    \item \textbf{Qualità di prodotto}: vengono individuate le caratteristiche più importati del prodotto, gli obbiettivi necessari per raggiungerle e le metriche per misurarle;
    \item \textbf{Specifiche dei test}: vengono definiti dei test attraverso cui il prodotto deve passare per garantire il soddisfacimento dei requisiti;
    \item \textbf{Standard di qualità}: descritti gli \glossario{standard di qualità} selezionati;
    \item \textbf{Resoconto delle attività di verifica}: vengono esposti i risultati dei test eseguiti durante il periodo di revisione; le metriche usate per l'ottenimento di questi risultati sono redatte nel documento;
    \item \textbf{Valutazioni per il miglioramento}: vengono elencati i problemi riscontrati durante lo sviluppo del progetto e vengono proposte delle soluzioni che potrebbero portare alla risoluzione o alla mitigazione dei problemi individuati.
\end{itemize} 

\subsubsection{Strumenti}
Saranno riportati di seguito gli strumenti che il gruppo ha deciso di utilizzare durante il progetto.

\paragraph{Strumenti usati}


\subsection{Sviluppo}

\subsubsection{Scopo}
Aderendo a quanto specificato nello standard \emph{ISO/IEC  12207:1995}, lo sviluppo consiste nella descrizione le varie attività di analisi, progettazione, codifica, integrazione, test, installazione e accettazione.

\subsubsection{Descrizione}

Questo processo è formato da tre attività principali:
\begin{itemize}
    \item \textbf{Analisi dei requisiti}
    \item \textbf{Progettazione architettura}
    \item \textbf{Codifica del software}
\end{itemize}

Ognuna di queste verrà descritta meglio nella sezione dedicata.

\subsubsection{Aspettative}
Le aspettative dello sviluppo sono principalmente le seguenti:
\begin{itemize}

\item Stabilire gli obbiettivi del prodotto;
\item Stabilire i requisiti tecnologici;
\item Stabilire i vincoli di design;
\item Realizzare un prodotto che soddisfi le richieste del proponente.

\end{itemize}

\subsubsection{Analisi dei requisiti}

\paragraph{Scopo}
Lo scopo dell'analisi dei requisiti è di redigere un documento che raccolga tutti requisiti che il proponente richiede per il progetto.\\
I requisiti hanno le seguenti finalità:
\begin{itemize}
    \item Descrivere lo scopo del lavoro;
    \item Fornire le indicazioni necessarie ai progettisti;
    \item Fissare le funzionalità concordate con il cliente;
    \item Fornire una base per un miglioramento continuo;
    \item Dare ai verificatori un modo per misurare le attività di controllo;
    \item Dare dei riferimenti per poter fare una stima del lavoro necessario.
\end{itemize}
Tale documento viene redatto dagli analisti.

\paragraph{Descrizione}
I requisiti, parte fondamentale di questo documento, si possono ricavare da varie fonti:
\begin{itemize}
    \item \textbf{Capitolato d'appalto}: la prima descrizione del prodotto messa a disposizione dal proponente. Da qui è possibile estrarre alcuni dei requisiti del progetto;
    \item \textbf{Incontri interni}: una possibilità è che un requisito emerga durante uno degli incontri interni, quindi quelli in cui partecipano i soli membri del gruppo, durante una discussione;
    \item \textbf{Incontri esterni}: un'ulteriore possibilità è che un requisito emerga in un incontro esterno, quindi in un incontro tra il gruppo e il referente dell'azienda proponente;
    \item \textbf{Casi d'uso}: infine è possibile che il requisito emerga durante la stesura dei casi d'uso perchè si presenta una necessità riguardante quello specifico caso d'uso.
\end{itemize}

\paragraph{Aspettative}

Lo scopo dell'analisi dei requisiti è scrivere un documento che raccolga tutti i requisiti individuati dagli analisti.

\paragraph{Struttura del documento}

\paragraph{Classificazione dei requisiti}

\paragraph{Classificazione dei casi d'uso}

\paragraph{Tracciamento dei requisiti e dei casi d'uso}

\paragraph{Metriche}




\end{document}
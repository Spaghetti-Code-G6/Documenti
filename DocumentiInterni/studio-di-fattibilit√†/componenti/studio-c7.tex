\documentclass[../studio-di-fattibilita.tex]{subfiles}

\begin{document}

\subsection{Informazioni generali}%
\label{sub:c7_informazioni_generale}
\begin{description}
  \item \textbf{Nome}: SSD: soluzioni di sincronizzazione desktop.
  \item \textbf{Proponente}: Zextras.
  \item \textbf{Committente}: prof. Tullio Vardanega e prof. Riccardo Cardin.
\end{description}

\subsection{Descrizione}%
\label{sub:c7_descrizione}
L’utente professionista si aspetta di avere accesso dinamico a tutti i contenuti, creati o modificati con il loro dispositivo principale, con tutti i propri dispositivi (es. mobile o web).
Viene richiesto di sviluppare un \glossario{algoritmo} di sincronizzazione desktop, il quale si appoggerà alla piattaforma Zextras Drive, e un’interfaccia multipiattaforma (Linux, MacOs, Windows).
Il prodotto finale dovrà aderire al \glossario{design pattern} MVC.

\subsection{Finalità del progetto}%
\label{sub:c7_finalita_del_progetto}
Il progetto consiste nello sviluppare un’applicazione di sincronizzazione Desktop strutturata nel seguente modo:
\begin{itemize}
  \item \textbf{Algoritmo di sincronizzazione}: un algoritmo solido ed efficiente in grado di garantire il salvataggio in cloud del lavoro e contemporaneamente la sincronizzazione dei cambiamenti presenti in cloud;
  \item \textbf{Interfaccia multipiattaforma}: interfaccia pensata per l’uso dell’algoritmo. Deve essere sviluppata per i tre sistemi operativi più popolari (Linux, MacOs, Windows);
  \item \textbf{Integrazione con Zextras Drive}: entrambi i punti sviluppati sopra, dovranno essere integrate con Zextras Drive e quindi poter fruire dei contenuti di questo prodotto.
\end{itemize}

L’algoritmo e l’interfaccia sviluppati dovranno poter essere utilizzabili senza richiedere all’utente l’installazione manuale di ulteriori prodotti.\\ \\

Le seguenti funzionalità base dovrebbero essere integrate:
\begin{itemize}
  \item Configurazione e autenticazione dell’utente;
  \item Sincronizzazione costante dei contenuti (locali o remoti);
  \item Gestione dei file da sincronizzare e da ignorare nelle cartelle locali e in quelle cloud, con la possibilità di modificare in qualsiasi momento le preferenze di sincronizzazione;
  \item Sistema di notifica l’utente dei cambiamenti relativi ai file; 
  \item Altre funzionalità avanzate presente nei principali competitor:
  \begin{itemize}
    \item Gestione delle condivisioni;
    \item Integrazione con il protocollo MAPI;
    \item Integrazione con il prodotto web.
  \end{itemize}
\end{itemize}

\subsection{Tecnologie interessate}%
\label{sub:c7_tecnologie_interessate}
Per questo capitolato viene consigliato di utilizzare le seguenti tecnologie:
\begin{itemize}
  \item \textbf{Qt Framework}: basato su C++, è consigliato per lo sviluppo dell’interfaccia grafica e del controller d’architettura;
  \item \textbf{Python}: linguaggio di programmazione, consigliato per lo sviluppo della Business Logic (algoritmo di sincronizzazione). Include le chiamate \glossario{API} verso i sistemi di Zextras Drive, facilitandone l’integrazione.
\end{itemize}

Tecnologie necessaria alla realizzazione del progetto:
\begin{itemize}
  \item \textbf{Zextras Drive}: sistema di gestione file in cloud;
  \item \textbf{Zimbra}: software applicativo di gruppo per la gestione della posta elettronica.
\end{itemize}

\subsection{Aspetti positivi}%
\label{sub:c7_aspetti_positivi}
Il capitolato risulta interessante per la scelta delle tecnologie di sviluppo consigliate, python infatti non è presente nel nostro percorso di studi e potrebbe essere una buona occasione per il suo apprendimento.

\subsection{Rischi}%
\label{sub:c7_rischi}
Il gruppo ha poca familiarità con lo sviluppo delle applicazioni desktop. Inoltre, la documentazione delle API Zextras Drive sembra poco accurata.
\subsection{Conclusioni}%
\label{sub:c7_conclusioni}
In seguito alla valutazione da parte del gruppo, è stato deciso di non prendere in considerazione il capitolato come prima scelta.

\end{document}

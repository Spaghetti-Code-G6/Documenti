\documentclass[../studio-di-fattibilita.tex]{subfiles}

\begin{document}

\subsection{Informazioni generali}%
\label{sub:c4_informazioni_generale}
\begin{description}
  \item \textbf{Nome}: HD Viz: visualizzazione di dati multidimensionali.
  \item \textbf{Proponente}: Zucchetti S.p.A..
  \item \textbf{Committente}: prof. Tullio Vardanega e prof. Riccardo Cardin.
\end{description}

\subsection{Descrizione}%
\label{sub:c4_descrizione}
Il capitolato richiede lo sviluppo di una \glossario{web application} che abbia come scopo la 
traduzione di dati con molte dimensioni in grafici che aiutino l’utente a trarre delle interpretazioni e conclusioni. Questi dati dovranno essere inseriti tramite file CSV oppure ottenuti tramite query da un database.

\subsection{Finalità del progetto}%
\label{sub:c4_finalita_del_progetto}
L’azienda richiede che l’applicazione presenti almeno le seguenti visualizzazioni:
\begin{itemize}
  \item \textbf{Scatter plot matrix}: è una delle visualizzazioni facilmente ottenibili con D3.js, ed è la presentazione a riquadri disposti a matrice di tutte le combinazioni di scatter plot;
  \item \textbf{Force field}: è un grafico che interpreta le distanze nello spazio a molte dimensioni in forze di attrazione e repulsione tra i punti proiettati nello spazio bidimensionale o tridimensionale;
  \item \textbf{Heat map}: trasforma la distanza tra i punti i colori più o meno intensi, facendo così capire quali oggetti sono vicini tra loro e quali sono distanti;
  \item \textbf{Proiezione lineare multi asse}: si occupa di posizionare i punti dello spazio multidimensionale in un piano cartesiano, rappresentando a 2 dimensioni anche dati con molte più dimensioni. 
\end{itemize}


\subsection{Tecnologie interessate}%
\label{sub:c4_tecnologie_interessate}
Il capitolato prevede quindi lo sviluppo di:
\begin{itemize}
  \item Una parte di HTML e CSS che dia una struttura e una presentazione all’applicazione;
  \item Un database che abbia la possibilità di memorizzare i dati in ingresso;
  \item Una parte di Java con server Tomcat o in Javascript con server Node.js per comunicare col server;
  \item Una parte di JavaScript che visualizzi i dati multidimensionali con l'utilizzo di D3.js, una libreria JavaScript;
  \item Una parte di JavaScript che si occupi di caricare i dati da file CSV;
\end{itemize}

\subsection{Aspetti positivi}%
\label{sub:c4_aspetti_positivi}
Gli aspetti positivi che si sono evidenziati sono la possibilità di approfondire la conoscenza di Javascript tramite l’utilizzo di D3.js; il gruppo ritiene molto utile poter studiare e applicare direttamente uno dei linguaggi più richiesti ed utilizzati nel mondo del lavoro, inoltre sono richiesti alcuni linguaggi che verranno affrontati con Tecnologie Web.


\subsection{Rischi}%
\label{sub:c4_rischi}
Alcuni aspetti negativi invece sono lo studio di algoritmi matematici complessi lontani dalla nostra preparazione accademica, quali ad esempio quelli che trattano le distanze tra i dati. Inoltre è richiesto molto intuito e capacità interpretative per poter capire quali siano i grafici più adeguati o esplicativi per determinati tipi di dato.

\subsection{Conclusioni}%
\label{sub:c4_conclusioni}
Il gruppo ha accolto con entusiasmo il progetto proposto dalla Zucchetti, soprattutto a seguito del seminario esplicativo che ha stimolato curiosità ed interesse verso l’argomento. C’è stato inoltre molto fervore verso le molteplici possibilità esplorative che offre il progetto nella sua applicazione d’uso.

\end{document}
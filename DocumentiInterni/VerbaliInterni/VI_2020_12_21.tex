\documentclass{article}

\input{../../risorse/config}

\appendToGraphicspath{../../risorse/img/}

\setTitle{Verbale Interno}

\setVersione{v0.0.3}

\setResponsabile{
	XX
}

\setRedattori{
	Contro Daniel Eduardo
}

\setVerificatori{
	XX
}

\setUso{Interno}

\setDestinatari{
	prof. Vardanega Tullio \\ &
	prof. Cardin Riccardo \\ &
	SpaghettiCode
}

\setDescrizione{Riassunto dell'incontro realizzato dal gruppo Spaghetti Code tenutosi il 21 Dicembre 2020 in forma di meeting online.}

\setModifiche{
	v0.0.3 & Daniel Eduardo Contro & Analista & 2020-12-23 & Redatte \hyperref[sec:info_generali]{\S 1}, \hyperref[sub:sub:organizzazione_gruppo]{\S 3.1}, \hyperref[sub:convenzioni]{\S 3.4} e \hyperref[sec:conclusione]{\S 4} e riformattata \hyperref[sub:registro_modifiche]{\S 3.3} \\
	v0.0.2 & Daniel Eduardo Contro & Analista & 2020-12-22 & Redatte \hyperref[sub:git_latex]{\S 3.2} e \hyperref[sub:registro_modifiche]{\S 3.3} \\
	v0.0.1 & Daniel Eduardo Contro & Analista & 2020-12-21 & Creazione del documento e della suddivisione
}

\disabilitaElencoFigure
\disabilitaElencoTabelle

\begin{document}
	
\pagenumbering{gobble}

\newif\iffirstpage
\firstpagetrue

\backgroundsetup{
  scale=1,
  opacity=0.2,
  angle=0,
  placement=top,
  contents={%
    \iffirstpage
      \includegraphics[width=\paperwidth]{datascience-og_colori.png}%
      \global\firstpagefalse
    \fi
  }%
  }

\begin{titlepage}% per non stampare il numero della pagina
  
  


  \raggedright % allinea a destra la pagina
  %\rule{1pt}{\textheight}% linea verticale
  \hspace{0.05\textwidth}% spazio tra linea e testo
  % lasciare questa riga per il corretto funziomento di \parbox
  \parbox[b]{0.75\textwidth}{% paragrafo che tiene il testo a destra della riga cambiando la larghezza il testo si muove a destra o a sinistra
  {\hspace{0.07\textwidth}\includegraphics[width=3.5cm,height=3.5cm]{logo-nero.png}}\\[3\baselineskip] % logo
  {\Huge\bfseries SpaghettiCode}\\ [\baselineskip] %titolo
  {\texttt{spaghetti.code.g6@gmail.com}}\\[\baselineskip]\\[4\baselineskip] % 
  {\Large\textsc{\placeholderTitle{}}}\\[4\baselineskip] % nome del documento
  {\begin{tabular}{r|l}
    \hline \\
    % testo in grassetto
    \textbf{Versione}     & \versione{}               \\
    \rule{0pt}{3ex}%  EXTRA vertical height 
    \textbf{Approvazione} & \responsabile{}           \\
    \rule{0pt}{3ex}%  EXTRA vertical height 
    \textbf{Redazione}    & \redattori{}              \\
    \rule{0pt}{3ex}%  EXTRA vertical height 
    \textbf{Verifica}     & \verificatori{}           \\
    \rule{0pt}{3ex}%  EXTRA vertical height 
    \textbf{Uso}          & \uso{}                    \\
    \rule{0pt}{3ex}%  EXTRA vertical height 
    \textbf{Destinato a}  & \destinatari{}            \\
    \rule{0pt}{3ex}%  EXTRA vertical height 
    \ifthenelse{\equal{\uso}{Esterno}}{
                          & Zucchetti S.p.A.       \\
    }{}
  \end{tabular}}\\[4\baselineskip]

  {\bfseries Descrizione}\\
  {\descrizione{}}\\[1\baselineskip]
  }

\end{titlepage}

\newgeometry{textheight=660pt, lmargin=2cm, tmargin=2cm, rmargin=2cm}

% setup di header e footer nelle pagine senza numero
\fancypagestyle{nopage}{%
  \fancyhf{}%
  \fancyhead[R]{\includegraphics[width=1.3cm]{logo-nero.png}}%
  \fancyhead[L]{\emph{SpaghettiCode}\\\placeholderTitle{}}%
}
% setup di header e footer nelle pagine col numero
\fancypagestyle{usual}{%
  \fancyhf{}%
  \fancyhead[R]{\includegraphics[width=1.3cm]{logo-nero.png}}%
  \fancyhead[L]{\emph{SpaghettiCode}\\\placeholderTitle{}}%
  \fancyfoot[R]{\thepage\ di~\pageref{LastPage}}%
}
\setlength{\headheight}{1.8cm}

\newpage
\pagestyle{nopage}

\setcounter{table}{-1}


%REGISTRO DELLE MODIFICHE

\section*{Registro delle modifiche}%
\label{sec:registro_delle_modifiche}

\rowcolors{2}{white!80!lightgray!90}{white}
\renewcommand{\arraystretch}{2} % allarga le righe con dello spazio sotto e sopra
\begin{longtable}[H]{>{\centering\bfseries}m{2cm} >{\centering}m{3.5cm} >{\centering}m{2.5cm} >{\centering}m{3cm} >{\centering\arraybackslash}m{5cm}}
  \rowcolor{lightgray}
  {\textbf{Versione}} & {\textbf{Nominativo}} & {\textbf{Ruolo}} & {\textbf{Data}} & {\textbf{Descrizione}}  \\
  \endfirsthead%
  \rowcolor{lightgray}
  {\textbf{Versione}} & {\textbf{Nominativo}}  & {\textbf{Ruolo}} & {\textbf{Data}} & {\textbf{Descrizione}}  \\
  \endhead%
  \modifiche{}%
\end{longtable}
% section registro_delle_modifiche (end)

\newpage
\thispagestyle{nopage}
\pagenumbering{roman}
\tableofcontents

\elencoFigure{}%

\elencoTabelle{}%

\newpage

\pagestyle{usual}
\pagenumbering{arabic}
	

\section{Informazioni generali}
\label{sec:info_generali}

\subsection{Informazioni incontro}
\label{sub:info_incontro}

\begin{itemize}
	\item \textbf{Luogo}: Applicazione desktop \glossario{Discord};
	\item \textbf{Data}: 2020-12-21;
	\item \textbf{Ora}: 17:30-19:30
	\item \textbf{Partecipanti}:
	\begin{itemize}
		\item Contro Daniel Eduardo
		\item Fichera Jacopo
		\item Kostadinov Samuel
		\item Masevski Martin
		\item Pagotto Manuel
		\item Paparazzo Giorgia
		\item Rizzo Stefano
	\end{itemize}
\end{itemize}
	

\section{Ordine del giorno}
\label{sec:ordine_del_giorno}
	Vengono riportati gli elementi che sono stati discussi nel corso del meeting:
	\begin{itemize}
		\item \nameref{sub:organizzazione_gruppo};
		\item \nameref{sub:registro_modifiche};
		\item \nameref{sub:git_latex};
		\item \nameref{sub:convenzioni}.
	\end{itemize}

\section{Resoconto}
\label{sec:resoconto}

	\subsection{Organizzazione del gruppo}
	\label{sub:organizzazione_gruppo}
	L'incontro si é aperto con una discussione riguardante la pianificazione dei ruoli nell'arco di tutta la durata del progetto, i cui risultati sono stati riportati nel 
	\textit{Piano di Progetto}. Inoltre si é discusso di alcune possibili problematiche identificate da alcuni membri del gruppo durante la stesura del documento 
	\textit{Studio di Fattibilità}.
	
	\subsection{Versionamento dei file su \glossario{Github} e utilizzo di \glossario{\LaTeX}}
	\label{sub:git_latex}
	A seguito della necessità di possedere una storia sequenziale e precisa dei documenti, emersa durante la stesura dello studio di fattibilità, si é svolta una discussione in merito 
	all'utilizzo della repository su \glossario{Github} e dell'utilizzo di \glossario{\LaTeX} per la redazione dei documenti. In un primo momento a questi strumenti era stato preferito 
	l'utilizzo di \glossario{Google Docs}, ma ora che il template per i documenti in \glossario{\LaTeX} é giunto ad una maturità ritenuta adeguata dal gruppo, é stato deciso di 
	proseguire con la redazione dei documenti utilizzando quest'ultimo, mantenendo la storia delle diverse versioni nel repository \textit{Documenti} dell'organizzazione 
	\glossario{Spaghetti Code} su \glossario{Github}.

	\subsection{Problematiche del registro delle modifiche}
	\label{sub:registro_modifiche}
	Il gruppo si é confrontato, in seguito alla stesura del documento interno \textit{Studio di Fattibilità} e all'incontro (del 17 Dicembre 2020) svoltosi con il prof. Tullio 
	Vardanega, sul come compilare il registro delle modifiche di modo che non risultino conflitti di ruolo. Si é quindi deciso di procedere effettuando una verifica del documento ogni 
	qualvolta venga fatta una modifica sostanziale, in modo da assicurare la conformità dello stesso; inoltre si é deciso di incrementare la versione \textit{minor} ad ogni verifica.

	\subsection{Convenzioni da adottare nei documenti}
	\label{sub:convenzioni}
	Si é discusso in merito alle convenzioni da adottare nella nomenclatura dei documenti del repository ufficiale, e al termine si é optato per l'utilizzo dello stile \glossario
	{snake\_case}. Si é deciso anche di evitare l'utilizzo di caratteri speciali e spazi per non incorrere in possibili problematiche di compatibilità tra sistemi operativi diversi.

\section{Conclusione dell'incontro}
\label{sec:conclusione}
L'incontro si é concluso con una pianificazione dettagliata e distribuzione dei task da eseguire entro il prossimo incontro.

\end{document}
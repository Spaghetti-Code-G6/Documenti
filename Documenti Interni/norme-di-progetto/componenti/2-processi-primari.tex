\documentclass[../norme-di-progetto.tex]{subfiles}

\begin{document}
    
\subsection{Fornitura}
\subsubsection{Scopo}
Il processo di fornitura consiste nelle seguenti attività:
\begin{itemize}
    \item Analisi di strumenti e competenze fondamentali e individuazione di rischi e criticità per il completamento del progetto. Questa analisi viene redatta nel documento \textsc{Studio di Fattibilità};
    \item Verificare la qualità del materiale eaborato. Le linee guida per la gestione della qualità sono descritte nel documento \textsc{Piano Di Qulifica};
    \item Stabilire ed esporre l'organizzazione del lavoro che il gruppo seguirà per la realizzazione del prodotto; queste possono essere trovate nel \textsc{Piano di Progetto}.
\end{itemize}

\setlength{\parindent}{0pt}Tale processo è composto dalle seguenti fasi:
\begin{itemize}
    \item Avvio;
    \item Approntamento di risposte alle richieste;
    \item Contrattazione;
    \item Pianificazione;
    \item Esecuzione e controllo;
    \item Revisione e valutazione;
    \item consegna e completamento.
\end{itemize}

\subsubsection{Descrizione}
Vengono qui descritte e trattare tutte norme a cui il gruppo \emph{SpaghettiCode} deve attenersi, con lo scopo di diventare i \glossario{fornitori} del prodotto \emph{HD Viz} del proponente \emph{Zucchetti S.p.A.} e dei committenti \emph{prof. Tullio Vardanega} e \emph{prof. Riccardo Cardin}.

\subsubsection{Aspettative}
Questo processo si pone gli obbiettivi di mantenere un confronto costante con il proponente \emph{Zucchetti S.p.A.} al fine di stimare le tempistiche di lavoro, verificare in modo continuo quanto prodotto dal gruppo, determinare i requisiti del prodotto e infine chiarire eventuali dubbi.
Successivamente all'evventua consegna, il gruppo \emph{SpaghettiCode}, non seguirà la fase di manutenzione del prodotto, salvo eventuali accordi.


\subsubsection{Attività}
\paragraph{Studio di Fattibilità}
Nel documento \textsc{Studio di Fattibilità}, redatto dagli \emph{analisti}, viene fornita un'analisi generale di tutti i capitolati proposti, e le motivazioni che hanno spinto il gruppo \emph{SpaghettiCode} a proporsi o meno come fornitore di uno specifico capitolato. 
Tale documento è strutturato nel seguente modo:
\begin{itemize}
    \item \textbf{Informazioni generali}: informazioni base del capitolato che comprende nome, proponente e committente;
    \item \textbf{Descrizione del capitolato}: breve sintesi delle caratteristiche del prodotto da sviluppare;
    \item \textbf{Finalità del progetto}: descrizione degli obbietivi richiesti dal capitolato d'appalto;
    \item \textbf{Tecnologie interessate}: tecnologie richieste dal progetto, comprendenti linguaggi di progammazioni e strumenti specifici che il gruppo dovrà utilizzare;
    \item \textbf{Aspetti positivi}: aspetti presi in considerazione dal gruppo a favore di una eventuale scelta del capitolato;
    \item \textbf{Rischi}: aspetti negativi da affrontare individuate dal gruppo in caso di scelta del capitolato;  
    \item \textbf{Conclusioni}: motivivazioni che hanno spinto il gruppo a scegliere o meno il capitolato.
\end{itemize}

\paragraph{Piano di Progetto}
Gli \emph{\glossario{amministratori}}, sotto la supervisione del \glossario{responsabile di progetto}, regigono il \textsc{Piano di Progetto}; questo documento andrà seguito per tutto lo svolgimento del progetto ed è strutturato nel seguente modo:
\begin{itemize}
    \item \textbf{Analisi dei rischi}: sezione in cui vengono analizzati i rischi che possono presentarsi nel corso del progetto. Venogono fornite anche le modalità con cui vengono risolti o ridimensionati questi rischi;
    \item \glossario{\textbf{Modello di sviluppo}}: sezione in cui viene descritto il \glossario{modello di sviluppo} scelto dal gruppo;
    \item \textbf{Pianificazione}: vengono descritte e pianificate le attività da eseguire nelle vari fasi del progetto, stabilendo i termini temporali (\glossario{deadlines}) per il loro completamento;
    \item \textbf{Preventivo e consuntivo}: viene stimata la quantità di lavoro neccessaria per ogni fase del progetto. Viene quindi esposto un preventivo e un successivo consuntivo relativi ad un dato periodo.
\end{itemize}

\paragraph{Piano di Qualifica}
Il \textsc{Piano di Qualifica}, redatto dai \emph{verificatori}, contiene le strategie da adottare per garantire la qualità nei materiali prodotti dal gruppo.
Il documento è strutturato nel seguente modo:
\begin{itemize}
    \item \textbf{Qualità di processo}:
    \item \textbf{Qualità di prodotto}:
    \item \textbf{Specifiche dei test}:
    \item \textbf{Standard di qualità}:
    \item \textbf{Valutazioni per il miglioramento}:
    \item \textbf{Resoconto delle attività di verifica}:
\end{itemize} 

\subsubsection{Strumenti}

\end{document}
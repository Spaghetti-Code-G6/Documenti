\section{Processi primari}
\label{sec:processi_primari}

\subsection{Fornitura}

\subsubsection{Scopo}

% TODO: ampliare intro
Il primo processo ad essere istanziato è quello di fornitura. Questo processo sarà attivo per tutta la durata del 
progetto perchè segue un modello di sviluppo incrementale, perciò subirà continue modifiche e migliorie.
Il processo di fornitura consiste nelle seguenti attività:
\begin{itemize}
    \item Analisi di strumenti, di competenze fondamentali e individuazione di rischi e \glossario{criticità} per il 
    completamento del progetto. I risultati di questa analisi vengono redatti nel documento 
    \textsc{Studio di Fattibilità};
    \item Stabilimento dell'organizzazione del lavoro che il gruppo seguirà durante lo svolgimento del progetto. 
    I relativi risultati sono riportati nel documento \textsc{Piano di Progetto}. 
    Il \textsc{Piano di Progetto} sarà, come del resto questo documento, in continua evoluzione, in quanto la 
    pianificazione a lungo termine potrebbe subire modifiche a consuntivo di periodo causate da ritardi o anticipi 
    rispetto a quanto preventivato;
    \item Stabilimento delle metriche di verifica della qualità del materiale prodotto durante lo svolgimento delle 
    diverse attività. Le linee guida per la gestione della qualità sono descritte nel documento 
    \textsc{Piano di Qualifica}.
    % , sia per quanto riguarda i documenti, sia per quanto 
    % riguarderà , più avanti nel tempo, le attività di progettazione e verifica. 
\end{itemize}

Questo processo è composto dalle seguenti parti:
\begin{itemize}
    \item Avvio;
    \item Preparazione di risposte alle richieste;
    \item Contrattazione;
    \item Pianificazione;
    \item Esecuzione e controllo;
    \item Revisione e valutazione;
    \item Consegna e completamento.
\end{itemize}

\subsubsection{Descrizione}

In questa sezione vengono descritte e trattate tutte le norme a cui il gruppo \emph{SpaghettiCode} intende attenersi 
durante il processo di fornitura, con lo scopo di diventare i fornitori del prodotto \emph{HD Viz} del 
\glossario{proponente} \emph{Zucchetti S.p.A.} e dei committenti \emph{prof. Tullio Vardanega} e 
\emph{prof. Riccardo Cardin}.

\subsubsection{Aspettative}

Nello svolgersi di tale processo, il gruppo di lavoro si pone gli obiettivi di mantenere un confronto costante con il 
proponente \emph{Zucchetti S.p.A.}, nello specifico con il referente \emph{dott. Gregorio Piccoli}, al fine di stimare 
le tempistiche di lavoro, verificare in modo continuo quanto prodotto dal gruppo, determinare i requisiti del prodotto 
e infine chiarire eventuali dubbi.
All'avvenuta consegna, il gruppo \emph{SpaghettiCode} non seguirà il processo di manutenzione del prodotto salvo 
eventuali accordi.

\subsubsection{Attività}

\paragraph{Studio di Fattibilità}
\label{par:studio_fattibilita}

In seguito alla presentazione ufficiale dei diversi capitolati d'appalto, tenutasi in data 2020-11-05, i vari componenti 
del gruppo esprimono i propri interessi e dubbi in merito ai diversi capitolati.
Successivamente, dopo aver deciso la prima scelta del gruppo tra i vari capitolati disponibili, gli \emph{analisti} 
raccolgono ed analizzano meglio i punti di forza e le criticità emersi nei diversi capitolati, e riportano quanto 
riscontrato nel documento \textsc{Studio Di Fattibilità}, fornendo così le motivazioni che hanno spinto il gruppo 
\emph{SpaghettiCode} a proporsi o meno come \glossario{fornitore} di uno specifico capitolato. Questo documento è il 
prodotto dell'omonima attività, la quale è strutturata come segue: 
\begin{itemize}
    \item \textbf{Raccolta di informazioni generali}: si raccolgono tutte le informazioni basilari riguardanti il 
    capitolato che comprendono nome, proponente e committente;
    \item \textbf{Comprensione delle caratteristiche}: si studiano e comprendono le caratteristiche del prodotto che 
    deve essere sviluppato;
    \item \textbf{Comprensione dello scopo del progetto}: si studiano e comprendono i possibili fini del progetto;
    \item \textbf{Comprensione delle tecnologie interessate}: si studia e si determina quali siano le tecnologie 
    interessate dal capitolato, se imposte o suggerite dal proponente;
    \item \textbf{Valutazione degli aspetti positivi}: vengono individuati gli aspetti positivi di ogni capitolato;
    \item \textbf{Valutazione dei rischi}: vengono individuati tutti i possibili rischi del capitolato proposto;  
    \item \textbf{Conclusioni}: vengono ponderati gli aspetti positivi e gli aspetti negativi del capitolato, traendo 
    quindi le conclusioni in merito alla fattibilità
\end{itemize}
% Questo si riflette poi in un documento con la seguente struttura:
% \begin{itemize}
%     \item \textbf{Informazioni generali}
%     \item \textbf{Descrizione}
%     \item \textbf{Finalità del progetto}
%     \item \textbf{Tecnologie interessate}
%     \item \textbf{Aspetti positivi}
%     \item \textbf{Rischi}
%     \item \textbf{Conclusioni}
% \end{itemize}

\paragraph{Piano di Progetto}
\label{par:pdp}

Il gruppo deve pianificare come si svolgerà il progetto e considerare le tempistiche con cui ogni altra attività verrà 
eseguita. Quest'attività si formalizza nell'omonimo documento, redatto dagli \emph{amministratori}, sotto la 
supervisione del \emph{\glossario{responsabile di progetto}}. Vista la continua evoluzione del piano di progetto anche 
il relativo documento deve essere redatto e aggiornato per tutta la durata del progetto. La sua struttura è la seguente:
\begin{itemize}
    % TODO: what?
    \item \textbf{Analisi dei rischi}: sezione in cui vengono analizzati i rischi che possono presentarsi nel corso del 
    progetto; vengono inoltre fornite le modalità con cui vengono risolti o ridimensionati. Un'analisi più esaustiva 
    si troverà nel documento \glossario{Analisi dei requisiti};
    \item \textbf{\glossario{Modello di sviluppo}}: sezione in cui viene descritto il \glossario{modello di sviluppo} 
    scelto dal gruppo e le motivazioni che hanno portato a scegliere quel determinato modello;
    \item \textbf{Pianificazione}: vengono descritte e pianificate le attività da eseguire nelle vari fasi del progetto, 
    stabilendo i termini temporali (\glossario{deadline}) per il loro completamento. Queste deadline non sono però 
    rigide perché, come accennato prima, la pianificazione sarà sempre soggetta a modifiche e/o aggiornamenti;
    \item \textbf{Preventivo e consuntivo}: viene stimata la quantità di lavoro necessaria per ogni processo del 
    \glossario{ciclo di vita} del progetto. Viene quindi esposto un \glossario{preventivo} e un successivo 
    \glossario{consuntivo}, entrambi relativi ad un dato periodo.
\end{itemize}

\paragraph{Piano di Qualifica}

La presente attività raccoglie tutte le regole e le linee guida per garantire che i materiali prodotti rispettino 
determinati standard di qualità. Questo insieme di regole deve essere, come per il piano di progetto, formalizzato 
nell'omonimo documento redatto dai \emph{verificatori}.
Questo documento è strutturato nel seguente modo:
\begin{itemize}
    \item \textbf{Qualità di processo}: sono individuati i processi dagli \glossario{standard di processo}, definiti 
    degli obiettivi, strategie per attuarli e \glossario{metriche} per controllarli e misurarli;
    \item \textbf{Qualità di prodotto}: vengono individuate le caratteristiche più importanti del prodotto, gli 
    obiettivi necessari per raggiungerle e le metriche per misurarle;
    \item \textbf{Specifiche dei test}: vengono definiti dei \glossario{test} che il prodotto deve superare per 
    garantire il soddisfacimento dei requisiti;
    \item \textbf{Standard di qualità}: descritti gli \glossario{standard di qualità} selezionati;
    \item \textbf{Resoconto delle attività di verifica}: vengono esposti i risultati dei test eseguiti durante il 
    periodo di revisione; le metriche usate per l'ottenimento di questi risultati sono redatte nel documento;
    \item \textbf{Lista di controllo}:  lista che contiene gli errori riscontrati (sezione in continua fase di 
    aggiornamento per l'intera durata del progetto);
    \item \textbf{Valutazioni per il miglioramento}: vengono elencati i problemi riscontrati durante lo sviluppo del 
    progetto e vengono proposte delle soluzioni che potrebbero portare alla risoluzione o alla mitigazione degli stessi.
\end{itemize} 
In questo documento si fa uso di un codice identificativo dei rischi. Il codice avrà questa forma:
\begin{center}
	\textbf{[Tipologia][Codice]}
\end{center}
Dove:
\begin{itemize}

	\item \textbf{Tipologia}: indica la tipologia di rischio e può assumere tre valori:
	\begin{itemize}
		\item \textbf{RR}: indica un rischio legato ai requisiti;
		\item \textbf{RT}: indica un rischio legato alle tecnologie;
		\item \textbf{RO}: indica un rischio legato all'organizzazione.
	\end{itemize}
    \item \textbf{Codice}: Il codice è un numero progressivo univoco all'interno della tipologia, che permette di 
    identificare univocamente il rischio.
\end{itemize}

Ogni rischio avrà, oltre al codice, due informazioni: 
\begin{itemize}
	\item \textbf{Occorrenza}: indica a frequenza con cui occorre il rischio;
	\item \textbf{Gravità}: indica la gravità del rischio.
\end{itemize}

\subsubsection{Strumenti}

Saranno riportati di seguito gli strumenti che il gruppo ha deciso di utilizzare durante il progetto.

\paragraph{GitHub}

Il gruppo ha deciso, di comune accordo, di affidarsi a \emph{GitHub} come strumento di condivisione dei file e 
\glossario{versionamento}, sia per quanto riguarda il codice, sia per quanto riguarda i documenti. La scelta è 
ricaduta su questo strumento per la sua integrazione con un sistema di \glossario{issue tracking}, per la possibilità 
di lavorare in più \glossario{branch} contemporaneamente e anche per la sua grande diffusione.

\paragraph{Google Docs/Google Drive}

Questo strumento viene utilizzato dal gruppo per la sua estrema semplicità, che permette una veloce creazione e stesura 
di bozze. Nelle cartelle di Google Drive si troveranno quindi dei fogli di documenti di Google, ovvero documenti 
informali generalmente bozze di documenti poi formalizzati o scalette dei documenti che devono essere scritti.

\paragraph{\LaTeX}

Il gruppo ha di comune accordo deciso, dopo aver valutato l'opzione, di usare lo strumento \LaTeX\ per la scrittura 
formale dei documenti, perciò ha predisposto su \emph{GitHub} un \glossario{repository} in cui mantenere versionati i 
file .tex in modo da avere uno storico dei documenti e gestire in maniera più semplice le modifiche.

\paragraph{Discord}

Questo strumento viene utilizzato dal gruppo come canale di comunicazione principale, per organizzare gli incontri, per 
condividere risorse tra i membri e per suddividere in vari canali, corrispondenti ai diversi ruoli, così che ci sia una 
maggiore organizzazione e divisione dei ruoli stessi.

\paragraph{Telegram}

Si è deciso di creare un gruppo \glossario{Telegram} per le comunicazioni di minore importanza e per avere un canale di 
comunicazione alternativo in quanto lo strumento è stato considerato di più veloce accesso rispetto a \emph{Discord} 
dai membri del gruppo.

\paragraph{Trello e GitHub Projects}

Il gruppo aveva inizialmente deciso di provvedere all'organizzazione tramite le board di \glossario{Trello}, tuttavia 
successivamente è stato preferito l'utilizzo dei \glossario{GitHub Projects}. Entrambe sono delle applicazioni web per 
la creazione di liste in stile Kanban.

\subsection{Sviluppo}

\subsubsection{Scopo}

Secondo quanto specificato nello standard \textsc{ISO/IEC  12207:1995}, lo sviluppo consiste nella descrizione dei 
diversi compiti ed attività di analisi, progettazione, codifica, integrazione, test, installazione e accettazione 
relative al prodotto in considerazione.

\subsubsection{Descrizione}

Questo processo è formato da tre attività principali:
\begin{itemize}
    \item \textbf{Analisi dei requisiti}
    \item \textbf{Progettazione architetturale}
    \item \textbf{Codifica del software}
\end{itemize}

Ognuna di queste verrà descritta meglio nella sezione dedicata.

\subsubsection{Aspettative}

Le aspettative dello sviluppo sono principalmente le seguenti:

\begin{itemize}
    \item Stabilire gli obiettivi del prodotto;
    \item Stabilire i requisiti tecnologici;
    \item Stabilire i vincoli di design;
    \item Realizzare un prodotto che soddisfi le richieste del proponente.
\end{itemize}

\subsubsection{Analisi dei requisiti}

\paragraph{Scopo}

Lo scopo dell'analisi dei requisiti è quello di individuare tutti i requisiti richiesti dal proponente o intrinseci nel 
prodotto stesso. L'attività viene svolta da parte degli \emph{analisti}, e il suo risultato viene in seguito riportato 
nel documento \glossario{\textsc{Analisi dei Requisiti}} redatto dagli \emph{analisti} stessi. \\
I requisiti hanno le seguenti finalità:
\begin{itemize}
    \item Descrivere lo scopo del lavoro;
    \item Fornire le indicazioni necessarie ai \emph{progettisti};
    \item Fissare le funzionalità concordate con il cliente;
    \item Fornire una base per un miglioramento continuo;
    \item Dare ai verificatori un modo per misurare le attività di controllo;
    \item Dare dei riferimenti per poter fare una stima del lavoro necessario.
\end{itemize}

\paragraph{Descrizione}

I requisiti, parte fondamentale di questo documento, possono essere ricavati da varie fonti:
\begin{itemize}
    \item \textbf{Capitolato d'appalto}: la prima descrizione del prodotto messa a disposizione dal proponente. Da qui 
    è possibile estrarre alcuni dei requisiti del progetto;
    \item \textbf{Incontri interni}: è possibile che emergano requisiti durante una discussione ad uno degli incontri interni;
    \item \textbf{Incontri esterni}: è possibile che emergano requisiti durante una discussione ad uno degli incontri esterni con il proponente;
    \item \textbf{Casi d'uso}: infine è possibile che il requisito emerga durante la stesura dei casi d'uso perché si 
    presenta una necessità specifica.
\end{itemize}

\paragraph{Aspettative}

Le aspettative dell'attività di analisi dei requisiti sono quelle di individuare e fissare i requisiti del prodotto e di 
produrre un documento che li raccolga.

\paragraph{Struttura del documento}

La struttura del documento è contraddistinta dalle seguenti parti:
\begin{itemize}
    \item \textbf{Introduzione}: contiene informazioni riguardanti lo scopo del documento, lo scopo del progetto, il 
    glossario, i riferimenti (normativi e informativi);
    \item \textbf{Descrizione generale}: contiene una descrizione del documento che riporta informazioni circa il 
    prodotto, i suoi obiettivi, le sue funzioni, i suoi utenti con le loro caratteristiche, le architetture da adottare, 
    le tecnologie consigliate ed eventuali vincoli;
    \item \textbf{Casi d'uso}: sezione in cui è possibile trovare la struttura e gli attori primari e secondari dei vari 
    casi d'uso, seguita da una lista dei casi d'uso individuati;
    \item \textbf{Requisiti}: in questa sezione del documento è possibile trovare una lista dei requisiti divisi tra 
    funzionali, di qualità, di vincolo e prestazionali, oltre che al tracciamento tra fonti e requisiti e viceversa.
\end{itemize}

\paragraph{Classificazione dei requisiti}
La rappresentazione dei requisiti avviene secondo un codice, non variabile, concordato internamente che si presenta nella seguente forma:

\begin{center}
    \textbf{R[Tipologia][Importanza][Codice]}
\end{center}

Questa notazione, spiegata nel dettaglio di seguito, permette di identificare in modo univoco ogni requisito del prodotto.\\
\begin{itemize}
    \item \textbf{Tipologia}: i requisiti possono avere diversi tipi, questa parte del codice identificativo permette di 
    identificare la tipologia del requisito. I possibili valori sono:
    \begin{itemize}
        \item \textbf{V}: indica che il requisito è di vincolo, ossia una limitazione imposta dal 
        proponente circa i servizi offerti dal prodotto software all'utilizzatore finale;
        \item \textbf{F}: indica che il requisito è di tipo funzionale, quindi viene usato per 
        rappresentare le funzioni del prodotto;
        \item \textbf{P}: indica che il requisito è di tipo prestazionale, quindi viene usato per dare 
        delle limitazioni sulle prestazioni del prodotto software;
        \item \textbf{Q}: indica che il requisito è di qualità, quindi un vincolo sulla qualità del prodotto.
    \end{itemize}
    \item \textbf{Importanza}: i requisiti hanno diversa importanza all'interno del progetto. Questa parte del codice 
    permette di individuare quale sia l'importanza di ogni requisito. I possibili valori sono:
    \begin{itemize}
        \item \textbf{O}: indica che il requisito è obbligatorio, quindi questo requisito dovrà essere necessariamente 
        soddisfatto;
        \item \textbf{D}: indica che il requisito è desiderabile, quindi eventualmente negoziabile con il proponente. 
        Il soddisfacimento del requisito verrebbe visto positivamente dal proponente, in quanto fornirebbe al prodotto 
        una maggiore completezza, eppure non ne viene vincolata l'implementazione;
        \item \textbf{F}: indica che il requisito è facoltativo, dunque anche se porta un valore aggiunto al prodotto, 
        molto probabilmente comporta una piccola miglioria a dispendio di molto tempo e lavoro.
    \end{itemize}
    
    \item \textbf{Codice}: rappresenta il codice identificativo vero e proprio e si trova nella forma:
    \begin{center}
        \textbf{[CodiceBase](.[CodiceSottocaso])*}
    \end{center}
    Il codice così formato permette di riferire uno specifico caso d'uso (la cui identificazione è spiegata nel paragrafo successivo).
\end{itemize}

Oltre al codice ogni requisito avrà associate una serie di informazioni, quali:
\begin{itemize}
    \item \textbf{Descrizione}: una breve descrizione del requisito;
    \item \textbf{Fonte}: la fonte da cui è stato estrapolato (capitolato d'appalto, caso d'uso, verbali interni o verbali esterni).
\end{itemize}

\paragraph{Classificazione dei casi d'uso}

Analogamente ai requisiti, anche i casi d'uso hanno un codice immutabile che li identifica univocamente all'interno del progetto. Tale codice ha la forma:
\begin{center}
    \textbf{UC[CodiceBase](.[CodiceSottoCaso])*}
\end{center}
Il codice base permette di identificare il caso d'uso generale, mentre il codice del sotto caso fa riferimento ad eventuali sottocasi del caso generale.\\
Ogni caso d'uso ha una struttura ben definita, riportata di seguito:
\begin{itemize}
    \item \textbf{Descrizione}: breve descrizione del caso d'uso;
    \item \textbf{Attore primario}: entità che interagisce direttamente con il prodotto;
    \item \textbf{Attori secondari}: entità che supportano l'attore primario nel portare a termine il caso d'uso in 
    esame, questo elemento non è necessariamente presente;
    \item \textbf{Precondizione}: condizione del sistema necessaria affinché possa compiersi il caso d'uso in esame;
    % \item \textbf{Input}: Ciò che l'attore porta all'interno del sistema. Non necessariamente presente, serve a specificare con precisione cosa l'attore deve introdurre nel sistema;
    \item \textbf{Postcondizione}: condizione in cui si trova il sistema immediatamente dopo il compimento del caso 
    d'uso in esame;
    % \item \textbf{Output}: Ciò che il prodotto darà come risultato alla fine del flusso di eventi dello use case in esame. Non è necessariamente presente;
    \item \textbf{Scenario principale}: rappresentazione del flusso degli eventi previsti dal caso d'uso;
    \item \textbf{Scenari alternativi}: rappresentazioni alternative del flusso degli eventi del caso d'uso, questo 
    elemento non è necessariamente presente;
    \item \textbf{Estensioni}: elemento che indica casi d'uso eseguiti condizionatamente che determinano l'interruzione 
    dell'esecuzione del caso d'uso in esame, questo elemento non è necessariamente presente;
    \item \textbf{Inclusioni}: Eelemento che indica casi d'uso eseguiti incondizionatamente successivamente a quello in 
    esame, questo elemento non è necessariamente presente;
    \item \textbf{Generalizzazioni}: rappresentano delle possibili specializzazioni del caso d'uso, questo elemento non è necessariamente presente.
\end{itemize}

% TODO: rivedere
\paragraph{Tracciamento dei progressi e delle risorse}
Per tenere traccia dell'andamento del lavoro e dei ruoli assegnati, il gruppo ha deciso di usare come strumento 
l'\glossario{issue tracking system} di \emph{GitHub}, strumento che tramite il \glossario{workflow} delle 
\glossario{issue} stesse, diviso tra "To do", "In progress" e "Done". In questo modo ogni membro del gruppo, in 
base allo stato della issue può conoscere lo stato di avanzamento del lavoro sia nel caso della specifica issue
che globale considerando l'insieme delle issue. Ogni membro del gruppo si impegna quindi ad aggiornare lo stato
di avanzamento della issue in base al lavoro svolto e all'effettivo stato della stessa.

\paragraph{Metriche}

Il gruppo ha deciso di tenere traccia del completamento del progetto in base ai requisiti implementati. A questo scopo è 
stato deciso di calcolare una percentuale di completamento del progetto come il rapporto tra il numero di requisiti 
implementati e il numero dei requisiti totali, moltiplicato per 100. Tale metrica permette così di individuare lo stato 
di avanzamento del progetto in maniera immediata. Il valore di questo indice varia tra 0 e 100, dove 100 indica che il 
progetto è completato, mentre 0 indica che deve ancora essere iniziato.
\begin{center}
    \textbf{Nota}: \emph{Il gruppo ha deciso di considerare, inizialmente, il numero totale di requisiti uguale al numero totale di requisiti obbligatori e si riserva la 
    possibilità di aumentare il numero totale di requisiti in seguito ad eventuali negoziazioni con il proponente.}
\end{center}
 
\subsubsection{Progettazione}
 
\paragraph{Scopo}

Lo scopo di quest'attività è quello di individuare le caratteristiche che il prodotto deve avere per soddisfare nel 
modo migliore possibile le richieste del proponente in risposta ai requisiti individuati dall'analisi dei requisiti. \\
In quest'attività bisogna rispettare i seguenti vincoli:
\begin{itemize}
    \item Garantire la qualità del prodotto seguendo un principio di correttezza costruttivo;
    \item Organizzare e suddividere i compiti in modo da diminuire la complessità del problema, riducendolo via via in 
    sottoproblemi sempre più elementari fino ad arrivare ai singoli componenti;
    \item Ottimizzare l'uso di risorse.
\end{itemize}

\paragraph{Descrizione}
La progettazione è divisa in due parti fondamentali:
\begin{itemize}
    \item \glossario{Technology Baseline}: contiene le specifiche ad alto livello della progettazione del software, 
    i relativi diagrammi \glossario{UML} e dei test;
    \item \glossario{Product Baseline}: arricchisce di dettagli quanto specificato nella Technology baseline e 
    definisce i test necessari.
\end{itemize}

\paragraph{Aspettative}

La progettazione è un'attività svolta dai \emph{Progettisti}, volta a produrre l'architettura logica del prodotto. 
L'architettura deve essere formata da componenti chiari, riusabili e utilizzabili in modo che ci sia coesione tra 
le parti; inoltre è necessario rimanere entro i costi fissati.\\
L'architettura dovrà necessariamente rispettare i seguenti punti: 
\begin{itemize}
    \item Soddisfare i requisiti individuati dall'analisi dei requisiti;
    \item Adattarsi in caso i requisiti evolvano;
    \item Riuscire a gestire situazioni erronee;
    \item Risultare affidabile anche in situazioni sfavorevoli come temporanee mancanze;
    \item Garantire un certo livello di sicurezza rispetto ai malfunzionamenti;
    \item Presentare solo il minimo intervallo possibile di indisponibilità durante i periodi di manutenzione;
    \item Impiegare efficientemente le risorse;
    \item Garantire la riusabilità delle sue parti anche in altri applicativi;
    \item Presentare componenti semplici e con basso livello di accoppiamento.
\end{itemize}
 
\paragraph{Design pattern}

La scelta dei \glossario{design pattern} da utilizzare è lasciata ai \emph{progettisti}, i quali dovranno 
assicurarsi che le loro scelte portino a una soluzione che sia flessibile e lasci una certa libertà ai 
\emph{Programmatori}. Ogni design pattern utilizzato andrà spiegato e rappresentato in modo da poterne esporre 
significato e struttura.

\paragraph{Diagrammi UML}

Il gruppo ha scelto di utilizzare diagrammi UML allo scopo di rendere più chiare le scelte compiute in ambito di 
progettazione, dei diagrammi UML. Tra questi spiccano i diagrammi delle attività e quelli di sequenza. I primi vengono 
usati per descrivere il flusso di operazioni di un'attività, i secondi per illustrare sequenze di azioni.\\
Ci potranno essere, oltre ai due tipi già menzionati, diagrammi di altro tipo se i \emph{progettisti} lo riterranno 
utile.

\paragraph{Test}

Come specificato all'inizio, la definizione dei test è parte dell'attività di progettazione, quindi ogni 
\emph{progettista} dovrà affiancare alla progettazione del sistema la definizione dei relativi test. Le regole di 
nomenclatura da seguire sono le stesse valide per la nomenclatura dei metodi, descritte nella sezione 
\refSec{par:stile_codifica}.

\begin{center}
    \textbf{Nota}: \emph{Il gruppo si riserva di modificare e in particolar modo di ampliare, in caso fosse 
    necessario, la progettazione nelle fasi successive alla sua definizione.}
\end{center}

\subsubsection{Codifica}
 
\paragraph{Scopo}

Quest'attività, svolta dai \emph{programmatori}, ha lo scopo di scrivere del codice che traduca lo schema architetturale 
prodotto dai \emph{progettisti} in un prodotto eseguibile. Quest'attività è soggetta alle regole descritte nel paragrafo 
\refSec{par:stile_codifica}, per far sí che il codice sia più leggibile possibile.

\paragraph{Descrizione}

La scrittura del codice dovrà tradurre lo schema architetturale prodotto dai \emph{progettisti} mantenendo lo standard 
qualitativo richiesto e descritto nel \textsc{Piano di Qualifica}.

\paragraph{Aspettative}

L'obiettivo dell'attività di codifica è creare un prodotto software che permetta di soddisfare le richieste del 
proponente e che al contempo mantenga un certo livello di qualità, al fine di: garantire la leggibilità del codice; agevolare manutenzione, verifica e validazione.

\paragraph{Stile di codifica}
\label{par:stile_codifica}

Al fine di garantire uniformità nel codice prodotto, si è deciso di stabilire delle regole nella scrittura del codice:
\begin{itemize}
    \item Indentazioni: i blocchi di codice innestati, ad esclusione dei commenti, devono presentare 4 spazi di rientro 
    rispetto al livello precedente;
    \item Parentesi: inserire le parentesi sulla stessa riga del costrutto che le usa;
    \item Struttura dei metodi: la struttura dei metodi deve sempre avere alcune caratteristiche:
    \begin{itemize}
        \item I nomi dei metodi devono rispettare la convenzione snake\_case;
        \item I metodi devono essere il più brevi possibile e non possono essere contenute più istruzioni nella stessa 
        riga;
        \item Deve essere presente una spaziatura tra la parentesi tonda che contiene i parametri dei metodi e la 
        parentesi graffa di apertura di blocco.
    \end{itemize}
    \item Univocità dei nomi: le variabili e i metodi devono avere un nome univoco e rappresentativo che permetta di 
    identificarli univocamente e che eviti ambiguità;
    \item Costanti: i nomi delle costanti devono essere scritti in stampatello maiuscolo;
    \item Lingua: la lingua utilizzata per i nomi delle variabili e dei metodi deve essere l'inglese.
\end{itemize}
Il gruppo si riserva la possibilità di cambiare, prima di iniziare l'attività di codifica, le norme qui specificate in 
caso se ne sentisse la necessità.

\paragraph{Metriche}

Il gruppo ritiene prematuro stabilire una metrica definitiva per misurare la leggibilità del codice, perciò ha definito 
un indice indicativo che verrà integrato nei successivi stadi del progetto.\\
La formula momentaneamente adottata è:
\begin{center}
    $Leggibilit\grave{a} = \frac{numero\ di\ linee\ di\ commento}{numero\ di\ linee\ di\ codice}$
\end{center}

\paragraph{Strumenti}

Sono riportati di seguito gli strumenti utilizzati nel processo di sviluppo.

\subparagraph{HTML, CSS}

HTML e CSS sono due linguaggi usati nello sviluppo di pagine web, vista la natura di HD Viz l'utilizzo di queste 
tecnologie sarà necessario per il completamento del progetto.

\subparagraph{Database}

Il proponente vuole lasciare aperta la possibilità di caricare dati da un database senza tuttavia fornire dei vincoli 
a tal proposito, sarà perciò compito dei \emph{progettisti} scegliere il database più adatto al progetto.

\subparagraph{JavaScript}

Linguaggio di scripting per web application richiesto dal proponente che fornisce la possibilità di utilizzare la 
libreria D3.js.

\subparagraph{D3.js}

Libreria open source scritta in \emph{JavaScript} con lo scopo di facilitare la visualizzazione dei dati in grafici. 
Richiesta dal proponente, è lo strumento principale per la realizzazione del prodotto HD Viz.
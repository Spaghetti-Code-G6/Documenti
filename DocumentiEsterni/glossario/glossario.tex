\documentclass{article}

\input{../../risorse/config}
%aggiungere percorsi dai quali il documento prende le immagini
\appendToGraphicspath{../../risorse/img/}

\setTitle{Glossario}

\setVersione{v0.0.1}

\setResponsabile{Paparazzo Giorgia}

\setRedattori{
  Pagotto Manuel \\ &
	... \\
}

\setVerificatori{
  Pagotto Manuel \\ &
  ... \\
}

\setUso{Esterno}

\setDestinatari{
  prof. Vardanega Tullio \\ &
  prof. Cardin Riccardo \\ &
  SpaghettiCode
}

\setDescrizione{Documento con lo scopo di chiarire i possibili termini ambigui utilizzati
all’interno dei documenti scritti e redatti dal team.}

\setModifiche{
  v0.0.1 & Pagotto Manuel & Analista & 2021-01-05 & Creazione del documento e Aggiunte definizioni
}

\disabilitaElencoFigure{}
\disabilitaElencoTabelle{}

\begin{document}

\pagenumbering{gobble}

\newif\iffirstpage
\firstpagetrue

\backgroundsetup{
  scale=1,
  opacity=0.2,
  angle=0,
  placement=top,
  contents={%
    \iffirstpage
      \includegraphics[width=\paperwidth]{datascience-og_colori.png}%
      \global\firstpagefalse
    \fi
  }%
  }

\begin{titlepage}% per non stampare il numero della pagina
  
  


  \raggedright % allinea a destra la pagina
  %\rule{1pt}{\textheight}% linea verticale
  \hspace{0.05\textwidth}% spazio tra linea e testo
  % lasciare questa riga per il corretto funziomento di \parbox
  \parbox[b]{0.75\textwidth}{% paragrafo che tiene il testo a destra della riga cambiando la larghezza il testo si muove a destra o a sinistra
  {\hspace{0.07\textwidth}\includegraphics[width=3.5cm,height=3.5cm]{logo-nero.png}}\\[3\baselineskip] % logo
  {\Huge\bfseries SpaghettiCode}\\ [\baselineskip] %titolo
  {\texttt{spaghetti.code.g6@gmail.com}}\\[\baselineskip]\\[4\baselineskip] % 
  {\Large\textsc{\placeholderTitle{}}}\\[4\baselineskip] % nome del documento
  {\begin{tabular}{r|l}
    \hline \\
    % testo in grassetto
    \textbf{Versione}     & \versione{}               \\
    \rule{0pt}{3ex}%  EXTRA vertical height 
    \textbf{Approvazione} & \responsabile{}           \\
    \rule{0pt}{3ex}%  EXTRA vertical height 
    \textbf{Redazione}    & \redattori{}              \\
    \rule{0pt}{3ex}%  EXTRA vertical height 
    \textbf{Verifica}     & \verificatori{}           \\
    \rule{0pt}{3ex}%  EXTRA vertical height 
    \textbf{Uso}          & \uso{}                    \\
    \rule{0pt}{3ex}%  EXTRA vertical height 
    \textbf{Destinato a}  & \destinatari{}            \\
    \rule{0pt}{3ex}%  EXTRA vertical height 
    \ifthenelse{\equal{\uso}{Esterno}}{
                          & Zucchetti S.p.A.       \\
    }{}
  \end{tabular}}\\[4\baselineskip]

  {\bfseries Descrizione}\\
  {\descrizione{}}\\[1\baselineskip]
  }

\end{titlepage}

\newgeometry{textheight=660pt, lmargin=2cm, tmargin=2cm, rmargin=2cm}

% setup di header e footer nelle pagine senza numero
\fancypagestyle{nopage}{%
  \fancyhf{}%
  \fancyhead[R]{\includegraphics[width=1.3cm]{logo-nero.png}}%
  \fancyhead[L]{\emph{SpaghettiCode}\\\placeholderTitle{}}%
}
% setup di header e footer nelle pagine col numero
\fancypagestyle{usual}{%
  \fancyhf{}%
  \fancyhead[R]{\includegraphics[width=1.3cm]{logo-nero.png}}%
  \fancyhead[L]{\emph{SpaghettiCode}\\\placeholderTitle{}}%
  \fancyfoot[R]{\thepage\ di~\pageref{LastPage}}%
}
\setlength{\headheight}{1.8cm}

\newpage
\pagestyle{nopage}

\setcounter{table}{-1}


%REGISTRO DELLE MODIFICHE

\section*{Registro delle modifiche}%
\label{sec:registro_delle_modifiche}

\rowcolors{2}{white!80!lightgray!90}{white}
\renewcommand{\arraystretch}{2} % allarga le righe con dello spazio sotto e sopra
\begin{longtable}[H]{>{\centering\bfseries}m{2cm} >{\centering}m{3.5cm} >{\centering}m{2.5cm} >{\centering}m{3cm} >{\centering\arraybackslash}m{5cm}}
  \rowcolor{lightgray}
  {\textbf{Versione}} & {\textbf{Nominativo}} & {\textbf{Ruolo}} & {\textbf{Data}} & {\textbf{Descrizione}}  \\
  \endfirsthead%
  \rowcolor{lightgray}
  {\textbf{Versione}} & {\textbf{Nominativo}}  & {\textbf{Ruolo}} & {\textbf{Data}} & {\textbf{Descrizione}}  \\
  \endhead%
  \modifiche{}%
\end{longtable}
% section registro_delle_modifiche (end)

\newpage
\thispagestyle{nopage}
\pagenumbering{roman}
\tableofcontents

\elencoFigure{}%

\elencoTabelle{}%

\newpage

\pagestyle{usual}
\pagenumbering{arabic}


%TODO cambiare virgola in D3,js

\section{A}
\subfile{componenti/A.tex}
\newpage

%TODO creare i vari file *.tex in componenti e aggiungere definizioni 

\section{C}
\begin{description}
  \item[CSV] Comma-separated values (abbreviato in CSV) è un formato di file basato su file di testo utilizzato per l'importazione ed esportazione di una tabella di dati.
  \item[Capitolato] Il capitolato è un documento tecnico redatto dal cliente, generalmente allegato ad un contratto di appalto, in cui sono specificati i vincoli contrattuali per lo sviluppo di determinato prodotto software.
\end{description}
\newpage


\section{D}
\begin{description}
  \item[D3.js] D3.js è una libreria JavaScript per creare visualizzazioni dinamiche ed interattive partendo da dati organizzati, visibili attraverso un comune browser.
  \item[Database] Insieme di informazioni (o dati) strutturate in genere memorizzate elettronicamente in un sistema informatico.
  \item[Deadline] Termine temporale massimo non superabile.
  \item[Design pattern] Soluzione generale e riutilizzabile a un problema che si verifica comunemente in un determinato contesto.
  \item[Discord] Applicazione di messaggistica istantanea e VoIP.
\end{description}
\newpage
\section{E}
\begin{description}
  \item[Exploratory Data Analysis] Approccio all'analisi di dataset per riassumere le loro caratteristiche principali, spesso con metodi visivi.
\end{description}
\newpage
\section{F}
\begin{description}
  \item[Force Field] Grafo di rete in grado di esprimere mediante archi e forze di attrazione o repulsione tra i nodi le relazioni nei dati. 
  \item[Fornitori] {scrivere o ignorare questa definizione}
\end{description}
\newpage
\section{G}
\begin{description}
  \item[GitHub] GitHub è un servizio di hosting per progetti software. È una implementazione dello strumento di controllo versione distribuito Git.
  \item[Google Docs] Applicazione web gratuita di elaborazizone del testo.
\end{description}
\newpage
\section{H}
\begin{description}
  \item[Heat Map] Metodo di visualizzazione di dati a molte dimensioni che consiste in una tabella le cui celle sono colorate in maniera diversa in base al valore che contengono.
\end{description}
\newpage
\section{J}
\begin{description}
  \item[JavaScript] Linguaggio di programmazione orientato agli oggetti e agli eventi, comunemente utilizzato nella programmazione Web lato client (esteso poi anche al lato server) per la creazione, in siti web e applicazioni web, di effetti dinamici interattivi.
\end{description}
\newpage
\section{L}
\begin{description}
  \item[LaTeX] {scrivere o ignorare questa definizione}
\end{description}
\newpage
\section{M}
\begin{description}
  \item[Modello di sviluppo] {scrivere o ignorare questa definizione}
\end{description}
\newpage
\section{O}
\begin{description}
  \item[Open source] {scrivere o ignorare questa definizione}
\end{description}
\newpage
\section{P}
\begin{description}
  \item[Piano Di Qualifica] {scrivere o ignorare questa definizione}
  \item[Piano di Progetto] {scrivere o ignorare questa definizione}
  \item[Postcondizione] {scrivere o ignorare questa definizione}
  \item[Precondizione] {scrivere o ignorare questa definizione}
  \item[Processi] {scrivere o ignorare questa definizione}
  \item[Product baseline] {scrivere o ignorare questa definizione}
  \item[Progetto] Nasce da una richiesta (e/o pagamento) che diventa poi un impegno se accettato. Porta dei vincoli : di costo, di risorse, di tempo. Spesso insieme di attività da svolgere in modo collaborativo.
  \item[Proiezione Lineare Multi Asse] Metodo di visualizzazione di dati a molte dimensioni che consiste nel proiettare i punti sulle n dimensioni in un grafico scatterplot a n assi. 
  \item[Pull request] Meccanisco usato dagli sviluppatori per notificare i membri del gruppo che è stata completata una funzionalità software
\end{description}
\newpage
\section{Q}
\begin{description}
  \item[Query] Interrogazione di un database per estrarre o aggiornare i dati che soddisfano un certo criterio di ricerca.
\end{description}
\newpage
\section{R}
\begin{description}
  \item[Requisti] {scrivere o ignorare questa definizione}
  \item[Responsabile di progetto] {scrivere o ignorare questa definizione}
\end{description}
\newpage
\section{S}
\begin{description}
  \item[Scatter Plot Matrix] Metodo di visualizzazione di dati n-dimensionali che prevede la creazione di una matrice nxn di grafici scatterplot. All’n-esima riga e all’n-esima colonna corrisponde la n-esima dimensione del dato. In questo modo si ottengono tutte le possibili combinazioni di grafici scatterplot sulle n dimensioni.
  \item[Skype] {scrivere o ignorare questa definizione}
  \item[Spaghetti Code] {scrivere o ignorare questa definizione}
  \item[Standard di processo] Riferimento di base generico usato come stile comune per lo svolgimento delle funzioni aziendali, pensato per una collettività di casi afferenti ad un certo dominio applicativo.
  \item[Standard di qualità] Criteri da seguire nello sviluppo software.
\end{description}
\newpage
\section{T}
\begin{description}
  \item[Technology baseline] {scrivere o ignorare questa definizione}
  \item[Trello] {scrivere o ignorare questa definizione}
\end{description}
\newpage
\section{V}
\begin{description}
  \item[Verificatori] {scrivere o ignorare questa definizione}
\end{description}
\newpage
\section{W}
\begin{description}
  \item[Way of working] Insieme di processi di progetto istanziati per una particolare fase del ciclo di vita del progetto. Definisce per costruzione la conformità del prodotto agli obiettivi di qualità e funzionalità identificati.
  \item[Web application] Applicazione accessibile via web per mezzo di una architettura cliente-server, che offre determinati servizi all'utente.
\end{description}
\newpage

\end{document}
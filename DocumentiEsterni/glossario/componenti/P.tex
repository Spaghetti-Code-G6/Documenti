\documentclass[../glossario.tex]{subfiles}

\begin{document}

\subsection*{Piano di Progetto}
\addcontentsline{toc}{subsection}{Piano di Progetto}
Vedi sezione \S2.1.4.2 delle \textsc{Norme di Progetto}.

\subsection*{Piano Di Qualifica} 
Vedi sezione \S2.1.4.3 delle \textsc{Norme di Progetto}.

\subsection*{Preventivo}
\addcontentsline{toc}{subsection}{Preventivo}
Documento che riporta dettagliatamente, in cifre, le previsioni relative alla gestione di un'azienda oppure a singole operazioni.

\subsection*{Processi} 
\addcontentsline{toc}{subsection}{Processi}
Insieme di attività correlate tra loro, che vengono eseguite con una certa continuità per raggiungere un preciso obbiettivo, consumando risorse.

\subsection*{Product baseline} 
\addcontentsline{toc}{subsection}{Product baseline}
Documentazione tecnica che descrive la configurazione di una Continuous Integration durante la fasi di produzione e di rilascio del suo ciclo di vita.

\subsection*{Progettista} 
\addcontentsline{toc}{subsection}{Progettista}
Colui che sintetizza una soluzione a partire dalle specifiche di un problema già analizzato.

\subsection*{Progetto} 
\addcontentsline{toc}{subsection}{Progetto}
Nasce da una richiesta (e/o pagamento) che diventa poi un impegno se accettato. Porta dei vincoli : di costo, di risorse, di tempo. Spesso insieme di attività da svolgere in modo collaborativo.

\subsection*{Programmatore} 
\addcontentsline{toc}{subsection}{Programmatore}
Colui che implementa una parte della soluzione dei progettisti.

\subsection*{Proiezione Lineare Multi Asse}
\addcontentsline{toc}{subsection}{Proiezione Lineare Multi Asse} 
Metodo di visualizzazione di dati a molte dimensioni che consiste nel proiettare i punti sulle n dimensioni in un grafico scatterplot a n assi. 

\subsection*{Proof of Concept} 
\addcontentsline{toc}{subsection}{Proof of Concept}
Si tratta di una realizzazione incompleta o abbozzata di un progetto o metodo, allo scopo di provarne la fattibilità o dimostrare la fondatezza di alcuni principi o concetti costituenti.

\subsection*{Proponente} 
\addcontentsline{toc}{subsection}{Proponente}
Azienda o ente che propone il capitolato d'appalto per un progetto.

\subsection*{Pull request} 
\addcontentsline{toc}{subsection}{Pull request}
Meccanismo usato dagli sviluppatori per notificare i membri del gruppo che è stata completata una funzionalità software



    
\end{document}
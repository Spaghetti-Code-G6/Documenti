\documentclass[../glossario.tex]{subfiles}




\begin{document}

\subsection*{Machine learning} 
\addcontentsline{toc}{subsection}{Machine learning}
Branca dell'intelligenza artificiale che utilizza metodi statistici per migliorare la performance di un algoritmo nell'identificare pattern nei dati.


\subsection*{Metadati}
\addcontentsline{toc}{subsection}{Metadati}
PLACEHOLDER


\subsection*{Metriche} 
\addcontentsline{toc}{subsection}{Metriche}
Standard per la misura di alcune proprietà del software o delle sue specifiche.

\subsection*{Milestone} 
\addcontentsline{toc}{subsection}{Milestone}
Importante traguardo intermedio nello svolgimento di un progetto.

\subsection*{Modello} 
\addcontentsline{toc}{subsection}{Modello}
L'oggetto o il termine atto a fornire un conveniente schema di punti di riferimento ai fini della riproduzione o dell'imitazione, talvolta dell'emulazione.

\subsection*{Modello di sviluppo} 
\addcontentsline{toc}{subsection}{Modello di sviluppo}
Principio teorico che indica il metodo da seguire nel progettare e nello scrivere un programma.

    
\end{document}
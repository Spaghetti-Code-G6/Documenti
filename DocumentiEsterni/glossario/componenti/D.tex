\documentclass[../glossario.tex]{subfiles}

\begin{document}
    
\subsection*{D3.js}
\addcontentsline{toc}{subsection}{D3.js}
Libreria JavaScript per creare visualizzazioni dinamiche ed interattive partendo da dati organizzati, visibili attraverso un comune browser.


\subsection*{DataFrame}
\addcontentsline{toc}{subsection}{DataFrame}
Un data frame è una tabella o una struttura a matrice bidimensionale in cui ogni colonna contiene i valori di una variabile e ogni riga contiene un insieme di valori da ogni colonna.


\subsection*{Database}
\addcontentsline{toc}{subsection}{Database}
Insieme di informazioni (o dati) strutturate in genere memorizzate elettronicamente in un sistema informatico.

\subsection*{Deadline}
\addcontentsline{toc}{subsection}{Deadline}
Termine temporale massimo non superabile.

\subsection*{Dendrogramma}
\addcontentsline{toc}{subsection}{Dendrogramma}
Il dendrogramma è un albero (grafo) utilizzato per visualizzare la somiglianza nel processo di “raggruppamento”. Nelle tecniche di clustering, il dendrogramma viene utilizzato per fornire una rappresentazione grafica del processo di raggruppamento delle istanze.

\subsection*{Design pattern}
\addcontentsline{toc}{subsection}{Design pattern}
Soluzione generale e riutilizzabile a un problema che si verifica comunemente in un determinato contesto.

\subsection*{Discord}
\addcontentsline{toc}{subsection}{Discord}
Applicazione di messaggistica istantanea e VoIP progettata per permettere a più utenti di comunicare tra loro.

\subsection*{Distanza Euclidea}
\addcontentsline{toc}{subsection}{Distanza Euclidea}
E' una distanza tra due punti, in particolare è una misura della lunghezza del segmento avente per estremi i due punti. Usando questa distanza, lo spazio euclideo diventa uno spazio metrico

\subsection*{Docker}
\addcontentsline{toc}{subsection}{Docker}
Progetto open-source che automatizza il deployment di applicazioni all'interno di contenitori software, fornendo un'astrazione aggiuntiva grazie alla virtualizzazione a livello di sistema operativo di Linux.


\end{document}
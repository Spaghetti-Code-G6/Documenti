\documentclass[../glossario.tex]{subfiles}


\begin{document}


\subsection*{Filosofia incrementale}
\addcontentsline{toc}{subsection}{Filosofia incrementale}
PLACEHOLDER

\subsection*{Force Field}
\addcontentsline{toc}{subsection}{Force Field}
Grafo di rete in grado di esprimere mediante archi e forze di attrazione o repulsione tra i nodi le relazioni nei dati.

\subsection*{Fornitore}
\addcontentsline{toc}{subsection}{Fornitore}
Soggetto (individuo, team o azienda) incaricato di realizzare il prodotto richiesto dal proponente.


\subsection*{Filosofia Incrementale}
\addcontentsline{toc}{subsection}{Filosofia Incrementale}
Si intende un modello di sviluppo basato sulla successione di pianificazione, analisi dei requisiti, progettazione, implementazione, test, valutazione del progetto software.


\subsection*{Framework}
\addcontentsline{toc}{subsection}{Framework}
Architettura logica di supporto sulla quale un software può essere progettato e realizzato, spesso facilitandone lo sviluppo da parte del programmatore.

\subsection*{Front-end}
\addcontentsline{toc}{subsection}{Front-end}
Parte visibile all'utente di un programma e con cui egli può interagire. Tipicamente il front-end corrisponde all'interfaccia utente.

    
\end{document}
\documentclass[../glossario.tex]{subfiles}


\begin{document}

\subsection*{API}
\addcontentsline{toc}{subsection}{API}
Acronimo di Application Programming Interface. Insieme di definizioni e protocolli con i quali vengono realizzati e integrati software applicativi.

\subsection*{API REST}
\addcontentsline{toc}{subsection}{API REST}
Interfaccia di programmazione delle applicazioni conforme ai vincoli dell'architettura REST. REST è l'acronimo di REpresentational State Transfer.

\subsection*{Algoritmo}
\addcontentsline{toc}{subsection}{Algoritmo}
Sequenza finita di operazioni che consente di risolvere un dato problema.

\subsection*{Amazon Web Services}
\addcontentsline{toc}{subsection}{Amazon Web Services}
Azienda statunitense di proprietà del gruppo Amazon, che fornisce servizi di cloud computing su un'omonima piattaforma on demand.

\subsection*{Amministratore}
\addcontentsline{toc}{subsection}{Amministratore}
Persona che controlla ad ogni istante della vita del progetto che le risorse (umane, materiali, economiche e strutturali) siano presenti e operanti; inoltre, gestisce la documentazione, controlla il versionamento e la configurazione.

\subsection*{Analisi dei requisiti}
\addcontentsline{toc}{subsection}{Analisi dei requisiti}
Definire cosa è necessario fare per portare a termine un progetto.

\subsection*{Analista}
\addcontentsline{toc}{subsection}{Analista}
Persona che si occupa di tutte le attività di analisi.

\subsection*{Architettura}
\addcontentsline{toc}{subsection}{Architettura}
Insieme di criteri di progetto sui quali è stato sviluppato il prodotto.

\subsection*{Attività}
\addcontentsline{toc}{subsection}{Attività}
Parte di un processo che dev'essere compiuta entro un determinato periodo di tempo.



\end{document}
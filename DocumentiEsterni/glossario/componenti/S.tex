\documentclass[../glossario.tex]{subfiles}


\begin{document}

\subsection*{Scatter Plot Matrix}
\addcontentsline{toc}{subsection}{Scatter Plot Matrix} 
Metodo di visualizzazione di dati n-dimensionali che prevede la creazione di una matrice NxN di grafici scatter plot. All’n-esima riga e all’n-esima colonna corrisponde la n-esima dimensione del dato. In questo modo si ottengono tutte le possibili combinazioni di grafici scatterplot sulle n dimensioni.

\subsection*{Scatter Plot} 
\addcontentsline{toc}{subsection}{Scatter Plot}
È un tipo di grafico in cui due variabili di un set di dati sono riportate su uno spazio cartesiano. I dati sono visualizzati tramite una collezione di punti ciascuno con una posizione sull'asse orizzontale determinato da una variabile e sull'asse verticale determinato dall'altra.


\subsection*{Serverless} 
\addcontentsline{toc}{subsection}{Serverless}
Con il termine serverless si intende un network la cui gestione non viene incentrata su dei server, come spesso accade, ma viene dislocata fra i vari utenti che utilizzano il network stesso, quindi il lavoro necessario di gestione del network viene eseguito dagli stessi utilizzatori.

\subsection*{Skype} 
\addcontentsline{toc}{subsection}{Skype}
Software proprietario freeware di messaggistica istantanea e VoIP.

\subsection*{Standard di processo} 
\addcontentsline{toc}{subsection}{Standard di processo}
Riferimento di base generico usato come stile comune per lo svolgimento delle funzioni aziendali, pensato per una collettività di casi afferenti ad un certo dominio applicativo.

\subsection*{Standard di qualità} 
\addcontentsline{toc}{subsection}{Standard di qualità}
Criteri da seguire nello sviluppo software.


\subsection*{Suite} 
\addcontentsline{toc}{subsection}{Suite}
Pacchetto di programmi complementari, in grado di interagire e di scambiarsi reciprocamente i dati.

    
\end{document}
\documentclass[../glossario.tex]{subfiles}



\begin{document}

\subsection*{CSS}
\addcontentsline{toc}{subsection}{CSS}
PLACEHOLDER

\subsection*{CSV}
\addcontentsline{toc}{subsection}{CSV}
Comma-separated values (abbreviato in CSV) è un formato di file basato su file di testo utilizzato per l'importazione ed esportazione di una tabella di dati.

\subsection*{Capitolato}
\addcontentsline{toc}{subsection}{Capitolato}
Documento tecnico redatto dal cliente, generalmente allegato ad un contratto di appalto, in cui sono specificati i vincoli contrattuali per lo sviluppo di determinato prodotto software.


\subsection*{Ciclo di Deming}
\addcontentsline{toc}{subsection}{Ciclo di Deming}
PLACEHOLDER


\subsection*{Ciclo di vita}
\addcontentsline{toc}{subsection}{Ciclo di vita}
Modo in cui una metodologia di sviluppo scompone l'attività di realizzazione di prodotti software in sotto attività fra loro coordinate, il cui risultato finale è la realizzazione del prodotto stesso e tutta la documentazione ad esso associata: fasi tipiche includono lo studio o analisi, la progettazione, la realizzazione, il collaudo, la messa a punto, l'installazione, la manutenzione e l'estensione, il tutto ad opera di uno o più sviluppatori software.

\subsection*{Cloud}
\addcontentsline{toc}{subsection}{Cloud}
Modello di conservazione dati su computer in rete dove i dati stessi sono memorizzati su molteplici server virtuali generalmente ospitati presso strutture di terze parti o su server dedicati. I dati sono accessibili ovunque e in qualsiasi momento utilizzando una connessione ad Internet.

\subsection*{Consuntivo}
\addcontentsline{toc}{subsection}{Consuntivo}
Rendiconto dei risultati di un dato periodo di attività di un ente o di un'impresa.

\subsection*{Container}
\addcontentsline{toc}{subsection}{Container}
Nei sistemi di virtualizzazione che operano a livello di sistema operativo, un container è un’istanza isolata nello spazio utente.

\subsection*{Criticità}
\addcontentsline{toc}{subsection}{Criticità}
Distanza troppo breve tra attività dipendenti.

\end{document}
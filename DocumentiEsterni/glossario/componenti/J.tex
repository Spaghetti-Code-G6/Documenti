\documentclass[../glossario.tex]{subfiles}

\begin{document}

\subsection*{JavaScript}
\addcontentsline{toc}{subsection}{JavaScript}
Linguaggio di programmazione orientato agli oggetti e agli eventi, comunemente utilizzato nella programmazione Web lato client (esteso poi anche al lato server) per la creazione, in siti web e applicazioni web, di effetti dinamici interattivi.

\subsection*{Just in time}
\addcontentsline{toc}{subsection}{Just in time}
È un modo di eseguire codice informatico che implica la compilazione durante l'esecuzione di un programma - in fase di esecuzione - piuttosto che prima dell'esecuzione. Negli ambiti al di fuori del codice, significa eseguire azioni o processi quando necessario.
    
\end{document}
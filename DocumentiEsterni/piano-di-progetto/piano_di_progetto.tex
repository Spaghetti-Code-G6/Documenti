\documentclass{article}

\input{../../risorse/config}
%aggiungere percorsi dai quali il documento prende le immagini
\appendToGraphicspath{../../risorse/img/}

\setTitle{Piano di Progetto}

\setVersione{v1.0.0}

\setResponsabile{
  Masevski Martin
}

\setRedattori{
  Paparazzo Giorgia
}

\setVerificatori{
  Pagotto Manuel \\ &
  Kostadinov Samuel
}

\setUso{Esterno}

\setDestinatari{
  prof. Vardanega Tullio \\ &
  prof. Cardin Riccardo 
}

\setDescrizione{Questo documento ha lo scopo di descrivere la pianificazione del gruppo \emph{SpaghettiCode} nello sviluppo del progetto \emph{HD Viz}}


%\disabilitaElencoFigure{}
%\disabilitaElencoTabelle{}

\setModifiche{
  v1.0.0 & Masevski Martin & Responsabile & 2021-01-07 & Approvazione documento \\
  v0.2.0 & Kostadinov Samuel & Verificatore & 2021-01-07 & Verifica \S\ref{sec:introduzione}, \S\ref{sec:rischi}, \S\ref{sec:sviluppo} \\
  v0.1.0 & Pagotto Manuel & Verificatore & 2021-01-06 & Verifica \S\ref{sec:pianif}, \S\ref{sec:prev}, \S\ref{sec:cons}, \S\ref{sec:org} \\
  v0.0.6 & Paparazzo Giorgia & Responsabile & 2021-01-05 & Modifica \S\ref{sec:pianif}, \S\ref{sec:org} \\
  v0.0.5 & Paparazzo Giorgia  & Responsabile & 2021-01-04 & Modifica \S\ref{sec:rischi}, \S\ref{sec:prev}, \S\ref{sec:cons} \\ 
  v0.0.4 & Paparazzo Giorgia & Responsabile & 2020-12-27 & Modifica  \S\ref{sec:org}, \S\ref{sec:cons}, \S\ref{sec:prev} \\
  v0.0.3 & Paparazzo Giorgia & Responsabile & 2020-12-21 & Inizio \S\ref{sec:prev}, proseguimento \S\ref{sec:pianif}  \\
  v0.0.2 & Paparazzo Giorgia & Responsabile & 2020-12-18 & Stesura \S\ref{sec:introduzione}, \S\ref{sec:rischi} ,\S\ref{sec:sviluppo}, \S\ref{sec:pianif}  \\
  v0.0.1 & Paparazzo Giorgia & Responsabile & 2020-12-17 & Inizio stesura \S\ref{sec:org}
}

\begin{document}

\pagenumbering{gobble}

\newif\iffirstpage
\firstpagetrue

\backgroundsetup{
  scale=1,
  opacity=0.2,
  angle=0,
  placement=top,
  contents={%
    \iffirstpage
      \includegraphics[width=\paperwidth]{datascience-og_colori.png}%
      \global\firstpagefalse
    \fi
  }%
  }

\begin{titlepage}% per non stampare il numero della pagina
  
  


  \raggedright % allinea a destra la pagina
  %\rule{1pt}{\textheight}% linea verticale
  \hspace{0.05\textwidth}% spazio tra linea e testo
  % lasciare questa riga per il corretto funziomento di \parbox
  \parbox[b]{0.75\textwidth}{% paragrafo che tiene il testo a destra della riga cambiando la larghezza il testo si muove a destra o a sinistra
  {\hspace{0.07\textwidth}\includegraphics[width=3.5cm,height=3.5cm]{logo-nero.png}}\\[3\baselineskip] % logo
  {\Huge\bfseries SpaghettiCode}\\ [\baselineskip] %titolo
  {\texttt{spaghetti.code.g6@gmail.com}}\\[\baselineskip]\\[4\baselineskip] % 
  {\Large\textsc{\placeholderTitle{}}}\\[4\baselineskip] % nome del documento
  {\begin{tabular}{r|l}
    \hline \\
    % testo in grassetto
    \textbf{Versione}     & \versione{}               \\
    \rule{0pt}{3ex}%  EXTRA vertical height 
    \textbf{Approvazione} & \responsabile{}           \\
    \rule{0pt}{3ex}%  EXTRA vertical height 
    \textbf{Redazione}    & \redattori{}              \\
    \rule{0pt}{3ex}%  EXTRA vertical height 
    \textbf{Verifica}     & \verificatori{}           \\
    \rule{0pt}{3ex}%  EXTRA vertical height 
    \textbf{Uso}          & \uso{}                    \\
    \rule{0pt}{3ex}%  EXTRA vertical height 
    \textbf{Destinato a}  & \destinatari{}            \\
    \rule{0pt}{3ex}%  EXTRA vertical height 
    \ifthenelse{\equal{\uso}{Esterno}}{
                          & Zucchetti S.p.A.       \\
    }{}
  \end{tabular}}\\[4\baselineskip]

  {\bfseries Descrizione}\\
  {\descrizione{}}\\[1\baselineskip]
  }

\end{titlepage}

\newgeometry{textheight=660pt, lmargin=2cm, tmargin=2cm, rmargin=2cm}

% setup di header e footer nelle pagine senza numero
\fancypagestyle{nopage}{%
  \fancyhf{}%
  \fancyhead[R]{\includegraphics[width=1.3cm]{logo-nero.png}}%
  \fancyhead[L]{\emph{SpaghettiCode}\\\placeholderTitle{}}%
}
% setup di header e footer nelle pagine col numero
\fancypagestyle{usual}{%
  \fancyhf{}%
  \fancyhead[R]{\includegraphics[width=1.3cm]{logo-nero.png}}%
  \fancyhead[L]{\emph{SpaghettiCode}\\\placeholderTitle{}}%
  \fancyfoot[R]{\thepage\ di~\pageref{LastPage}}%
}
\setlength{\headheight}{1.8cm}

\newpage
\pagestyle{nopage}

\setcounter{table}{-1}


%REGISTRO DELLE MODIFICHE

\section*{Registro delle modifiche}%
\label{sec:registro_delle_modifiche}

\rowcolors{2}{white!80!lightgray!90}{white}
\renewcommand{\arraystretch}{2} % allarga le righe con dello spazio sotto e sopra
\begin{longtable}[H]{>{\centering\bfseries}m{2cm} >{\centering}m{3.5cm} >{\centering}m{2.5cm} >{\centering}m{3cm} >{\centering\arraybackslash}m{5cm}}
  \rowcolor{lightgray}
  {\textbf{Versione}} & {\textbf{Nominativo}} & {\textbf{Ruolo}} & {\textbf{Data}} & {\textbf{Descrizione}}  \\
  \endfirsthead%
  \rowcolor{lightgray}
  {\textbf{Versione}} & {\textbf{Nominativo}}  & {\textbf{Ruolo}} & {\textbf{Data}} & {\textbf{Descrizione}}  \\
  \endhead%
  \modifiche{}%
\end{longtable}
% section registro_delle_modifiche (end)

\newpage
\thispagestyle{nopage}
\pagenumbering{roman}
\tableofcontents

\elencoFigure{}%

\elencoTabelle{}%

\newpage

\pagestyle{usual}
\pagenumbering{arabic}


\section{Introduzione}%
\label{sec:introduzione}

\subsection{Scopo del documento}%
\label{sub:scopo_del_documento}
Lo scopo del documento è fornire una descrizione sulle modalità con cui il gruppo \emph{SpaghettiCode} realizzerà il progetto \emph{HD Viz}. Nello specifico verranno trattati i seguenti punti:
\begin{itemize}
  \item Modello di sviluppo adottato
  \item Pianificazione delle attività
  \item Stima dei costi e dei ruoli
  \item Organigramma 
\end{itemize}

\subsection{Scopo del prodotto}%
\label{sub:scopo_prodotto}
Il capitolato C4 - \emph{HD Viz} nasce dalla necessità di trasformare grosse moli di dati multidimensionali in grafici che diano la possibilità di interpretare le informazioni o apprenderne di nuove. Il gruppo \emph{SpaghettiCode} si offre quindi di sviluppare la web application commissionata dall’azienda \emph{Zucchetti S.p.A.} seguendo le tecnologie richieste dal proponente. 


\subsection{Glossario}%
\label{sub:glossario}
Per aiutare il lettore nella comprensione di tale documento verrà allegato un \emph{Glossario}. Ogni parola contenuta in esso verrà qui indicata con una G a pedice.


\subsection{Riferimenti}%
\label{sub:riferimenti}
\begin{itemize}
  \item Regolamento organigramma e specifica tecnico-economica: \url{https://www.math.unipd.it/~tullio/IS-1/2020/Progetto/RO.html};
  \item Capitolato d’appalto C4: \url{https://www.math.unipd.it/~tullio/IS-1/2020/Progetto/C4.pdf};
  \item Slide "Gestione di progetto": \url{https://www.math.unipd.it/~tullio/IS-1/2020/Dispense/L06.pdf};
\end{itemize}

\subsection{Scadenze}%
\label{sub:scad}

\begin{itemize}
  \item Revisione dei requisiti: 18 gennaio 2021
  \item Revisione di Progettazione: 8 marzo 2021
  \item Revisione di Qualifica: 9 aprile 2021
  \item Revisione di Accettazione: 10 maggio 2021

\end{itemize}

\newpage

\section{Analisi dei rischi}
\label{sec:rischi}
\subfile{componenti/2-analisi_rischi.tex}

\newpage

\section{Modello di sviluppo}
\label{sec:sviluppo}
\subfile{componenti/3-modello_di_sviluppo.tex}

\newpage

\section{Pianificazione}
\label{sec:pianif}
\subfile{componenti/4-pianificazione.tex}

\newpage
\section{Preventivo}
\label{sec:prev}
\subfile{componenti/5-preventivo.tex}

\newpage

\section{Consuntivo}
\label{sec:cons}
\subfile{componenti/6-consuntivo.tex}

\newpage

\section{Organigramma}
\label{sec:org}
\subfile{componenti/7-organigramma.tex}

\end{document}

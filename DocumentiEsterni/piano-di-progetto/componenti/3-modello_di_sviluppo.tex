\documentclass[../piano_di_progetto.tex]{subfiles}

\begin{document}

\subsection{Metodo Incrementale}
\label{sub:incr}
Per garantire la qualità del prodotto e uno sviluppo corretto sul lungo periodo il gruppo ha scelto l’approccio incrementale, ovvero tramite rilasci che mirino ad integrare nel sistema ogni nuova funzionalità. Grazie ad esso i cicli di incremento quali pianificazione, analisi dei requisiti, progettazione, implementazione, test e valutazione verranno ripetuti per ogni funzionalità finché il prodotto non raggiungerà i requisiti richiesti. I requisiti fondamentali sono quelli che verranno affrontati per primi, in seguito verranno sviluppati quelli opzionali.
I vantaggi del modello incrementale sono:
\begin{itemize}
    \item Priorità nello sviluppo delle funzionalità primarie;
    \item Ogni incremento viene sottoposto all’immediato controllo qualità sia dal gruppo, sia dal proponente;
    \item Gli errori di un singolo incremento sono più facili da individuare e correggere di conseguenza si ha minor rischio di fallimento.
\end{itemize}

\end{document}
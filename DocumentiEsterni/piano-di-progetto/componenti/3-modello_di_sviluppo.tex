\documentclass[../piano_di_progetto.tex]{subfiles}

\begin{document}

\subsection{Metodo Incrementale}
\label{sub:incr}
Per garantire la qualità del prodotto e uno sviluppo corretto sul lungo periodo il gruppo ha scelto l’ \glossario{approccio incrementale}, ovvero l'impiego di \glossario{rilasci} che mirino ad integrare nel sistema ogni nuova funzionalità. Grazie ad esso i cicli di incremento quali pianificazione, analisi dei requisiti, progettazione, implementazione, test e valutazione verranno ripetuti per ogni funzionalità e fase finchè il prodotto non raggiungerà i requisiti richiesti. I requisiti fondamentali sono quelli che verranno affrontati per primi, in seguito verranno sviluppati quelli opzionali.
I vantaggi del modello incrementale sono:
\begin{itemize}
\item priorità nello sviluppo delle funzionalità primarie
\item ogni incremento viene sottoposto all’immediato controllo qualità sia dal gruppo, sia dal proponente
\item gli errori di un singolo incremento sono più facili da individuare e correggere
\item di conseguenza si ha minor rischio di fallimento
\end{itemize}

\end{document}
\documentclass[../piano_di_progetto.tex]{subfiles}

\begin{document}

La pianificazione si è basata sulle scadenze descritte in \S\ref{sub:scad}; il gruppo ha scelto di suddividere il periodo che va dalla formazione alla Revisione di Accettazione nelle seguenti fasi:
\begin{itemize}
\item Analisi dei requisiti;
\item Progettazione architetturale;
\item Progettazione di dettaglio e codifica;
\item Validazione e collaudo
\end{itemize}
Ad ognuna di queste attività verranno destinate delle risorse di seguito descritte.


\subsection{Analisi}%
\label{sub:analisi}
Questa attività si svolge nel periodo che va dalla formazione del gruppo alla consegna della \glossario{Revisione dei Requisiti}. In questo periodo il gruppo inizia con la visione dei \glossario{capitolati} proposti, e per ognuno di essi traccia un prototipo di \glossario{studio di fattibilità} dove vengono evidenziati gli aspetti positivi e negativi di ciascuno. Nel frattempo vengono stabilite delle \glossario{norme di progetto}, utili a fissare degli standard di utilizzo ai quali il gruppo si deve attenere. Successivamente alla scelta del capitolato vengono tracciati i requisiti minimi richiesti dal proponente, inoltre il gruppo incontra l’azienda tramite meeting online allo scopo di risolvere eventuali dubbi.\\

\begin{figure}[H]
\centering
\includegraphics[width=12cm]{img/fase_analisi}
\caption{ \glossario{Diagramma di Gantt} della fase di analisi dei requisiti}
\end{figure}

\subsection{Consolidamento dei requisiti}%
\label{sub:cons_req}
La fase di consolidamento dei requisiti inizia subito dopo la consegna della Revisione dei Requisiti, e prevede l’approfondimento delle tecnologie richieste attraverso lo studio autonomo, con conseguente aggiornamento degli stessi, se necessario. Sarà inoltre previsto qualche contatto con la Zucchetti S.p.A. per fugare eventuali dubbi. Per concludere verrà preparata la presentazione. 
\begin{figure}[H]
\centering
\includegraphics[width=12cm]{img/fase_consolid}
\caption{ \glossario{Diagramma di Gantt} della fase di analisi di consolidamento dei requisiti}
\end{figure}

\subsection{Progettazione architetturale}%
\label{sub:prog_arc}
La fase di progettazione architetturale inizia subito dopo la presentazione della Revisione dei Requisiti.\\
In questa fase verranno aggiornati i documenti redatti, verranno aggiornati i requisiti in base ai contatti avuti col proponente, verrà fatto uno studio più approfondito delle tecnologie coinvolte nel progetto, al fine di realizzare un \glossario{Proof of Concept} che farà da base per la fase successiva. Verrà infine scritta la lettera di presentazione al fine di candidarsi alla revisione di progettazione.

\begin{figure}[H]
\centering
\includegraphics[width=12cm]{img/fase_prog_archit}
\caption{Diagramma di Gantt della fase di progettazione architetturale}
\end{figure}


\subsection{Progettazione di dettaglio e codifica}%
\label{sub:prog_dett}
Questa fase prevede la stesura del codice sulla base dei dettagli precedentemente tracciati; vi saranno aggiustamenti della pianificazione e aggiornamenti dei documenti. Verrà inoltre redatto il \glossario{manuale d’utente}. I contatti con il proponente si manterranno stretti al fine di poter effettuare i primi rilasci e assicurarsi di essere in linea con le richieste.\\
Per concludere verrà redatta la presentazione da portare in sede di revisione, e verrà redatta la lettera di presentazione per candidarsi alla revisione di qualifica. 

\begin{figure}[H]
\centering
\includegraphics[width=12cm]{img/fase_dett_cod}
\caption{Diagramma di Gantt della fase di dettaglio e codifica}
\end{figure}


\subsection{Validazione e Collaudo}%
\label{sub:valid_coll}
Questa fase terminerà con la Revisione di Accettazione; in questo periodo verranno aggiornati i documenti e verrà verificato il software creato. Saranno impiegati principalmente verificatori e programmatori al fine di effettuare un controllo serrato che software che verrà rilasciato. Sarà completato il manuale d'utente e sarà redatta la presentazione e la lettera finale. 

\begin{figure}[H]
\centering
\includegraphics[width=12cm]{img/fase_valid_collaudo}
\caption{Diagramma di Gantt della fase di validazione e collaudo}
\end{figure}

\end{document}
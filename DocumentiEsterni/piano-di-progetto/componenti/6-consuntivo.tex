\documentclass[../piano_di_progetto.tex]{subfiles}

\begin{document}
In questa sezione verranno illustrati i prospetti orari e relativi costi per le varie fasi di lavoro. Il bilancio si suddivide in:
\begin{itemize}
\item positivo: se sono state necessarie meno ore di quelle preventivate;
\item paritario: se sono state svolte effettivamente le ore preventivate;
\item negativo: se sono state necessarie più ore di quelle preventivate;
\end{itemize}

\subsection{ Periodo di Analisi}%
\label{sub:cons_analisi}
Le ore di lavoro svolte in queste fase sono destinate alla scelta del capitolato e allo studio autonomo, quindi alcune delle ore svolte da ogni componente non verranno rendicontate.

\begin{center}
	\begin{longtable}{|l|c|c|c|c|c|c|c|}
		\hline
		\rowcolor{lightgray}
		\textbf{Ruolo} & \textbf{Ore} & \textbf{Costo in €}\\
		\hline
		Responsabile & 17 & 510,00 \\
		\hline
		Amministratore & 20 & 400,00 \\
		\hline
		Analista & 65 & 1.625,00 \\
		\hline
		Progettista & 0 & 0 \\
		\hline
		Programmatore & 0 & 0 \\
		\hline
		Verificatore & 87 & 1.305,00 \\
		\hline
		\textbf{Totale preventivo} & 189 & 3.840,00 \\
		\hline
		\textbf{Totale consuntivo} & 189 & 3.840,00 \\
		\hline
		\textbf{Differenza} & 0 & 0\\
		\hline	
		\caption{Consuntivo nella fase di analisi}
	\end{longtable}
\end{center}

\subsection{ Conclusioni}%
\label{sub:cons_fine}
Come discusso precedentemente questa fase ha visto un maggior impiego di figure quali amministratori, analisti e verificatori.\\
In questa fase non risulta necessasrio apportare modifica nè sul preventivo orario nè su quello economico. 

\end{document}
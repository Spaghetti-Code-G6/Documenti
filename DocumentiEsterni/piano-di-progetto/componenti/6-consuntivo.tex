\documentclass[../piano_di_progetto.tex]{subfiles}

\begin{document}
In questa sezione verranno illustrati i prospetti orari e relativi costi per le varie fasi di lavoro. Il bilancio si suddivide in:
\begin{itemize}
\item \textbf{Positivo}: se sono state necessarie meno ore di quelle preventivate;
\item \textbf{Paritario}: se sono state svolte effettivamente le ore preventivate;
\item \textbf{Negativo}: se sono state necessarie più ore di quelle preventivate;
\end{itemize}

\subsection{ Periodo di Analisi}%
\label{sub:cons_analisi}
Le ore di lavoro svolte in queste fase sono destinate alla scelta del capitolato e allo studio autonomo, quindi queste ore svolte non verranno rendicontate.\\
Poichè l'analisi dei requisiti è un documento molto importante, ha richiesto l'investimento di più ore da parte degli analisti, che di conseguenza hanno impiegato meno ore nella verifica. 
Tutto il resto del gruppo ha controbilanciato queste esigenze investendo più ore nella verifica.

\begin{table}[!ht]
	\centering
	\begin{tabular}{|l|c|c|c|c|c|c|c|}
	\hline
	\rowcolor{lightgray}
	\textbf{Nome} & \textbf{Re} & \textbf{Am} & \textbf{An} & \textbf{Pg}  & \textbf{Pr}   & \textbf{Ve} & \textbf{Totale}\\
	\hline
		Contro Daniel Eduardo & 0 & 0 & 13(+3) & 0 & 0 & 14(-3) & 27 \\
	\hline
		Fichera Jacopo & 0 & 0 & 13(+3) & 0 & 0 & 14(-3) & 27 \\
	\hline
		Kostadinov Samuel & 0 & 10(-1) & 5(-2) & 0 & 0 & 12(+3) & 27 \\			
	\hline
		Masevski Martin & 4(-1) & 10(-1) & 5(-2) & 0 & 0 & 8(+4) & 27 \\
	\hline
		Pagotto Manuel & 0 & 0 & 13(-2) & 0 & 0 & 14(+2) & 27 \\			
	\hline
		Paparazzo Giorgia & 13(-1) & 0 & 4(-1) & 0 & 0 & 10(+2) & 27 \\
	\hline
		Rizzo Stefano & 0 & 0 & 12(+3) & 0 & 0 & 15(-3) & 27 \\
	\hline	
	\end{tabular}
	\caption{Tabella contenente la suddivisione delle ore nella fase di Analisi}
\end{table}

\begin{center}
	\begin{longtable}{|l|c|c|c|c|c|c|c|}
		\hline
		\rowcolor{lightgray}
		\textbf{Ruolo} & \textbf{Ore} & \textbf{Costo in €}\\
		\hline
		Responsabile & 17(-2) & 510,00(-60 €) \\
		\hline
		Amministratore & 20(-2) & 400,00(-40 €) \\
		\hline
		Analista & 65(+3) & 1.625,00(+75 €) \\
		\hline
		Progettista & 0 & 0 \\
		\hline
		Programmatore & 0 & 0 \\
		\hline
		Verificatore & 87(+1) & 1.305,00(+15 €) \\
		\hline
		\textbf{Totale preventivo} & \textbf{189} & \textbf{3.840,00 €} \\
		\hline
		\textbf{Totale consuntivo} & \textbf{189} & \textbf{3.830,00 €} \\
		\hline
		\textbf{Differenza} & \textbf{0} & \textbf{-10 €}\\
		\hline
		\rowcolor{white}
		\caption{Consuntivo nella fase di analisi}
	\end{longtable}
\end{center}

\subsection{ Conclusioni}%
\label{sub:cons_fine}
Come discusso precedentemente questa fase ha visto un maggior impiego di figure quali amministratori, analisti e verificatori.\\
In questa fase non risulta necessario apportare modifiche al preventivo orario poichè il gruppo è stato in grado di controbilanciarsi nelle ore svolte. Il preventivo economico invece ha subito una leggera modifica che ha avuto come conseguenza un minore costo rispetto a quello preventivato. \\
Poichè questa fase non è rendicontata, non risulta necessario apportare modifiche al budget finale. 


\end{document}
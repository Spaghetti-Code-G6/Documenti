\documentclass[../piano_di_progetto.tex]{subfiles}

\begin{document}

L’ \glossario{analisi dei rischi} viene effettuata allo scopo di ragionare e quindi prevenire alcuni rischi che si possono presentare nella realizzazione di un progetto così grande. Infatti è facile incorrere in problematiche derivanti dalle tempistiche prefissate, oppure dalle tecnologie nuove.\\
I principali rischi sono stati suddivisi nelle seguenti categorie:
\begin{itemize}
	\item Rischi dei requisiti - Codice \textbf{RR}
	\item Rischi tecnologici  - Codice \textbf{RT}
	\item Rischi Organizzativi - Codice \textbf{RO}
\end{itemize}
Nei seguenti paragrafi questi rischi verranno classificati secondo le seguenti metriche:
\begin{itemize}
	\item \textbf{Probabilità di occorrenza}: alta frequenza, media frequenza, bassa frequenza
	\item \textbf{Pericolosità del rischio}: alta, medio, bassa
\end{itemize}

\subsection{Rischi dei requisiti}%
\label{sub:rischi_req}

\begin{center}
	\begin{longtable}{|c|p{4.5cm}|c|c|p{4.5cm}|}
		\hline
		\rowcolor{lightgray}
		{\textbf{Codice}} & {\textbf{Descrizione}} & {\textbf{Occorrenza}} & {\textbf{Gravità}} & {\textbf{Mitigazione}}                                                     \\

		\hline
		RR01              &
		Sebbene il proponente abbia descritto dettagliatamente l’obiettivo del progetto e le relative richieste, l’inesperienza del gruppo potrebbe comportare una errata analisi dei requisiti
		                  &
		Media
		                  &
		Alta
		                  &
		Il gruppo si impegna a comunicare con il proponente per poter chiarire eventuali dubbi e accertarsi che l’andamento progettuale sia positivo e rispetti le richieste \\

		RR02              &
		Potrebbe verificarsi una modifica dei requisiti minimi o opzionali in corso d’opera
		                  &
		Bassa
		                  &
		Alta
		                  &
		Grazie ad una regolare comunicazione con il proponente il gruppo sarà in grado di rettificare il lavoro svolto nelle tempistiche opportune                           \\
		\hline
		\rowcolor{white}
		\caption{Tabella contenente l'elenco dei rischi dei requisiti}
	\end{longtable}

\end{center}

\newpage

\subsection{Rischi tecnologici}%
\label{sub:rischi_tec}

\begin{center}
	\begin{longtable}{|c|p{4.5cm}|c|c|p{4.5cm}|}
		\hline
		\rowcolor{lightgray}
		{\textbf{Codice}} & {\textbf{Descrizione}} & {\textbf{Occorrenza}} & {\textbf{Gravità}} & {\textbf{Mitigazione}}                                                                                                                                                                                                                \\

		\hline
		RT01              &
		Non tutti i componenti hanno conoscenze degli ambienti di sviluppo e linguaggi richiesti dal proponente
		                  &
		Media
		                  &
		Alta
		                  &
		Sarà compito di ciascun componente gestire il proprio tempo da dedicare allo studio autonomo affinché tutti possano raggiungere un livello paritario di preparazione                                                                                                                                                            \\
		RT02              &
		Tutti i componenti hanno dei personal computer con cui lavorare al progetto; potrebbe succedere che qualcuno abbia un guasto hardware o software che ne comporti la perdita di progressi
		                  &
		Bassa
		                  &
		Bassa
		                  &
		Ciascuno si impegna a lavorare su file online grazie all’ampio supporto software che l’Università offre. Nel caso qualcuno sia impossibilitato a proseguire nel proprio lavoro, costui comunicherà tempestivamente il problema al gruppo ed il responsabile ridistribuirà il lavoro da concludere al primo compagno disponibile \\
		\hline
		\rowcolor{white}
		\caption{Tabella contenente l'elenco dei rischi tecnologici}
	\end{longtable}
\end{center}

\subsection{Rischi Organizzativi}%
\label{sub:rischi_org}

\begin{center}
	\begin{longtable}{|c|p{4.5cm}|c|c|p{4.5cm}|}
		\hline
		\rowcolor{lightgray}
		{\textbf{Codice}} & {\textbf{Descrizione}} & {\textbf{Occorrenza}} & {\textbf{Gravità}} & {\textbf{Mitigazione}}\\
		\hline

		\endfirsthead
	
		\hline
		\rowcolor{lightgray}
		{\textbf{Codice}} & {\textbf{Descrizione}} & {\textbf{Occorrenza}} & {\textbf{Gravità}} & {\textbf{Mitigazione}}\\
		\hline
		\endhead
		
		\hline
		\multicolumn{5}{|c|}{\emph{Continua alla pagina successiva...}}\\
		\hline
		\endfoot

		\endlastfoot

		RO01              &
		Ogni componente ha degli impegni personali, questo comporta che non sempre saranno tutti disponibili ad incontrarsi ad una determinata ora
		                  &
		Media
		                  &
		Medio-alta
		                  &
		Il gruppo si accorda con anticipo per eventuali incontri; qualora dovessero mancare uno o due componenti il responsabile si occuperà di aggiornare gli assenti circa le riunioni svolte                                                         \\
		RO02              &
		Dal momento che ognuno ha degli impegni personali potrebbe non essere possibile completare i singoli compiti assegnati entro le tempistiche richieste
		                  &
		Medio-bassa
		                  &
		Alta
		                  &
		Ciascun componente si impegna a svolgere al meglio delle proprie possibilità i singoli compiti assegnati; qualora per incombenze non sia possibile concluderli entro le tempistiche richieste, il singolo si impegna a comunicarlo con anticipo \\
		RO03              &
		Nessun componente del gruppo ha mai lavorato in team così ampi, quindi potrebbe rivelarsi non facile mettere d’accordo 7 soggetti differenti
		                  &
		Media
		                  &
		Media
		                  &
		Il gruppo si impegna a prendere le decisioni rispettando il pensiero di chiunque, e mettendo ogni membro sullo stesso piano. Qualora dovessero sorgere incertezze il gruppo sceglie di adottare la votazione come soluzione dei conflitti.      \\
		\hline
		\rowcolor{white}
		\caption{Tabella contenente l'elenco dei rischi organizzativi}
	\end{longtable}

\end{center}

\end{document}
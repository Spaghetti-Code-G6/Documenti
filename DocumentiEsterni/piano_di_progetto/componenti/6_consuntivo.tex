\documentclass[../piano_di_progetto.tex]{subfiles}

\begin{document}
In questa sezione verranno illustrati i prospetti orari e relativi costi per le varie fasi di lavoro. Il bilancio si suddivide in:
\begin{itemize}
\item \textbf{Positivo}: se sono state necessarie meno ore di quelle preventivate;
\item \textbf{Paritario}: se sono state svolte effettivamente le ore preventivate;
\item \textbf{Negativo}: se sono state necessarie più ore di quelle preventivate;
\end{itemize}

\subsection{ Periodo di Analisi}%
\label{sub:cons_analisi}
Le ore di lavoro svolte in questa fase sono destinate alla scelta del capitolato e allo studio autonomo, quindi queste ore svolte non verranno rendicontate.\\
Poiché l'analisi dei requisiti è un documento molto importante, ha richiesto l'investimento di molte più ore da parte degli analisti, che di conseguenza hanno destinato meno ore nella verifica. 
Tutto il resto del gruppo ha controbilanciato queste esigenze investendo più ore nella verifica.

\begin{table}[!ht]
	\centering
	\begin{tabular}{|l|c|c|c|c|c|c|c|}
	\hline
	\rowcolor{lightgray}
	\textbf{Nome} & \textbf{Re} & \textbf{Am} & \textbf{An} & \textbf{Pg}  & \textbf{Pr}   & \textbf{Ve} & \textbf{Totale}\\
	\hline
		Contro Daniel Eduardo & 0 & 0 & 16(+3) & 0 & 0 & 15 (+1) & 31 \\
	\hline
		Fichera Jacopo &        0 & 0 & 16(+3) & 0 & 0 & 15 (+1) & 31 \\
	\hline
		Kostadinov Samuel &     0 & 10 & 5     & 0 & 0 & 12 & 27 \\			
	\hline
		Masevski Martin &       4 & 10 & 5     & 0 & 0 & 8 & 27 \\
	\hline
		Pagotto Manuel &        0 & 0 & 15(+2) & 0 & 0 & 16 (+2) & 31 \\			
	\hline
		Paparazzo Giorgia &     13 & 0 & 4     & 0 & 0 & 10 & 27 \\
	\hline
		Rizzo Stefano &         0 & 0 & 16(+4) & 0 & 0 & 15 & 31 \\
	\hline	
	\end{tabular}
	\caption{Tabella contenente la suddivisione delle ore nella fase di Analisi}
\end{table}

\begin{center}
	\begin{longtable}{|l|c|c|}
		\hline
		\rowcolor{lightgray}
		\textbf{Ruolo} & \textbf{Ore} & \textbf{Costo in €}\\
			
		\endhead
		
		\hline
		\multicolumn{3}{|c|}{\emph{Continua alla pagina successiva...}}\\
		\hline
		\endfoot

		\endlastfoot
		Responsabile &   17      & 510,00 \\
		\hline
		Amministratore & 20      & 400,00 \\
		\hline
		Analista &       77(+12) & 1.625,00(+300,00 €) \\
		\hline
		Progettista &    0       & 0 \\
		\hline
		Programmatore &  0       & 0 \\
		\hline
		Verificatore &   91(+4)      & 1.365,00 (+60,00€) \\
		\hline
		\textbf{Totale preventivo} & \textbf{189} & \textbf{3.840,00 €} \\
		\hline
		\textbf{Totale consuntivo} & \textbf{205	} & \textbf{4.200,00 €} \\
		\hline
		\textbf{Differenza} & \textbf{0} & \textbf{+360 €}\\
		\hline
		\rowcolor{white}
		\caption{Consuntivo nella fase di analisi}
	\end{longtable}
\end{center}

\subsection{ Conclusioni}%
\label{sub:cons_fine}
Come discusso precedentemente questa fase ha visto un maggior impiego di figure quali analisti e verificatori.\\
A fronte di una settimana di ritardo nella consegna, si è vista quindi la necessità di aumentare le ore di lavoro di alcuni dei componenti del gruppo e di conseguenza anche il bilancio economico ha visto un aumento di costi. Nonostante l'aumento del consuntivo di fase sfori il limite del 5\% stabilito nel Piano di Qualifica, questo non è un problema in termini di consuntivo rendicontato in quanto questa prima parte del lavoro non viene rendicontata. Infatti, proprio perchè questa fase non è rendicontata, non risulta necessario apportare modifiche al budget finale rendicontato.


\end{document}
\documentclass[../piano_di_progetto.tex]{subfiles}

\begin{document}
In questa sezione verranno illustrati i prospetti orari e relativi costi per le varie fasi di lavoro. Il bilancio si suddivide in:
\begin{itemize}
\item \textbf{Positivo}: se sono state necessarie meno ore di quelle preventivate;
\item \textbf{Paritario}: se sono state svolte effettivamente le ore preventivate;
\item \textbf{Negativo}: se sono state necessarie più ore di quelle preventivate;
\end{itemize}

\subsection{ Periodo di Analisi}%
\label{sub:cons_analisi}
Le ore di lavoro svolte in questa fase sono destinate alla scelta del capitolato e allo studio autonomo, quindi queste ore svolte non verranno rendicontate.\\
Poiché l'analisi dei requisiti è un documento molto importante, ha richiesto l'investimento di molte più ore da parte degli analisti.

\begin{table}[!ht]
	\centering
	\begin{tabular}{|l|c|c|c|c|c|c|c|}
	\hline
	\rowcolor{lightgray}
	\textbf{Nome} & \textbf{Re} & \textbf{Am} & \textbf{An} & \textbf{Pg}  & \textbf{Pr}   & \textbf{Ve} & \textbf{Totale}\\
	\hline
	Contro Daniel Eduardo & 0 & 0 & 16(+3) & 0 & 0 & 15 (+1) & 31(+4) \\
	Fichera Jacopo & 0 & 0 & 16(+3) & 0 & 0 & 15 (+1) & 31(+4) \\
	Kostadinov Samuel & 0 & 10 & 5 & 0 & 0 & 12 & 27 \\			
	Masevski Martin & 4 & 10 & 5 & 0 & 0 & 8 & 27 \\
	Pagotto Manuel & 0 & 0 & 15(+2) & 0 & 0 & 16 (+2) & 31(+4) \\			
	Paparazzo Giorgia & 13 & 0 & 4 & 0 & 0 & 10 & 27 \\
	Rizzo Stefano & 0 & 0 & 16(+4) & 0 & 0 & 15 & 31(+4) \\
	\hline
	\end{tabular}
	\caption{Tabella contenente la suddivisione delle ore nella fase di Analisi}
\end{table}

\begin{center}
	\begin{longtable}{|l|c|c|}
		\hline
		\rowcolor{lightgray}
		\textbf{Ruolo} & \textbf{Ore} & \textbf{Costo in €}\\
		\hline
		\endhead
		
		\hline
		\multicolumn{3}{|c|}{\emph{Continua alla pagina successiva...}}\\
		\hline
		\endfoot

		\endlastfoot
		Responsabile & 17 & 510,00 \\
		Amministratore & 20 & 400,00 \\
		Analista & 77(+12) & 1.625,00(+300,00 €) \\
		Progettista &    0       & 0 \\
		Programmatore &  0       & 0 \\
		Verificatore &   91(+4)      & 1.365,00 (+60,00€) \\
		\hline
		\textbf{Totale preventivo} & \textbf{189} & \textbf{3.840,00 €} \\
		\hline
		\textbf{Totale consuntivo} & \textbf{205	} & \textbf{4.200,00 €} \\
		\hline
		\textbf{Differenza} & \textbf{+16} & \textbf{+360 €}\\
		\hline
		\rowcolor{white}
		\caption{Consuntivo nella fase di analisi}
	\end{longtable}
\end{center}

\subsection{ Conclusioni}%
\label{sub:cons_fine}
A fronte di una settimana di ritardo nella consegna, si è vista la necessità di aumentare le ore di lavoro di alcuni dei componenti del gruppo, di conseguenza anche il bilancio economico ha visto un aumento di costi. Sebbene il surpls sfori il limite del 5\% stabilito nel \textsc{Piano di Qualifica}, questo costo non inteccherà il budget totale rendicontato perchè questa fase non è rendicontata.


\end{document}
\documentclass{article}

\input{../../risorse/config}
%aggiungere percorsi dai quali il documento prende le immagini
\appendToGraphicspath{../../risorse/img/}

\setTitle{Piano di Qualifica}

\setVersione{v0.0.1}

\setResponsabile{
  Giorgia Paparazzo
}

\setRedattori{
  Giorgia Paparazzo
}

\setVerificatori{
  XX
}

\setUso{Esterno}

\setDestinatari{
  prof. Vardanega Tullio \\ &
  prof. Cardin Riccardo \\ &
  Zucchetti S.p.A.
}

\setDescrizione{Questo documento ha lo scopo di descrivere le strategie di verifica e validazione del gruppo SpaghettiCode nello sviluppo del progetto HD Viz}

%\disabilitaElencoFigure{}
%\disabilitaElencoTabelle{}

\setModifiche{
  v0.0.3 & Giorgia Paparazzo & Redattore & 2020-12-30 & Modifica capitolo \S2,  \S3 \\
  v0.0.2 & Giorgia Paparazzo & Redattore & 2020-12-29 & Continuo capitolo \S2, stesura capitolo \S3, \S4 \\
  v0.0.1 & Giorgia Paparazzo & Redattore & 2020-12-23 & Inizio stesura capitolo \S2
}

\begin{document}

\pagenumbering{gobble}

\newif\iffirstpage
\firstpagetrue

\backgroundsetup{
  scale=1,
  opacity=0.2,
  angle=0,
  placement=top,
  contents={%
    \iffirstpage
      \includegraphics[width=\paperwidth]{datascience-og_colori.png}%
      \global\firstpagefalse
    \fi
  }%
  }

\begin{titlepage}% per non stampare il numero della pagina
  
  


  \raggedright % allinea a destra la pagina
  %\rule{1pt}{\textheight}% linea verticale
  \hspace{0.05\textwidth}% spazio tra linea e testo
  % lasciare questa riga per il corretto funziomento di \parbox
  \parbox[b]{0.75\textwidth}{% paragrafo che tiene il testo a destra della riga cambiando la larghezza il testo si muove a destra o a sinistra
  {\hspace{0.07\textwidth}\includegraphics[width=3.5cm,height=3.5cm]{logo-nero.png}}\\[3\baselineskip] % logo
  {\Huge\bfseries SpaghettiCode}\\ [\baselineskip] %titolo
  {\texttt{spaghetti.code.g6@gmail.com}}\\[\baselineskip]\\[4\baselineskip] % 
  {\Large\textsc{\placeholderTitle{}}}\\[4\baselineskip] % nome del documento
  {\begin{tabular}{r|l}
    \hline \\
    % testo in grassetto
    \textbf{Versione}     & \versione{}               \\
    \rule{0pt}{3ex}%  EXTRA vertical height 
    \textbf{Approvazione} & \responsabile{}           \\
    \rule{0pt}{3ex}%  EXTRA vertical height 
    \textbf{Redazione}    & \redattori{}              \\
    \rule{0pt}{3ex}%  EXTRA vertical height 
    \textbf{Verifica}     & \verificatori{}           \\
    \rule{0pt}{3ex}%  EXTRA vertical height 
    \textbf{Uso}          & \uso{}                    \\
    \rule{0pt}{3ex}%  EXTRA vertical height 
    \textbf{Destinato a}  & \destinatari{}            \\
    \rule{0pt}{3ex}%  EXTRA vertical height 
    \ifthenelse{\equal{\uso}{Esterno}}{
                          & Zucchetti S.p.A.       \\
    }{}
  \end{tabular}}\\[4\baselineskip]

  {\bfseries Descrizione}\\
  {\descrizione{}}\\[1\baselineskip]
  }

\end{titlepage}

\newgeometry{textheight=660pt, lmargin=2cm, tmargin=2cm, rmargin=2cm}

% setup di header e footer nelle pagine senza numero
\fancypagestyle{nopage}{%
  \fancyhf{}%
  \fancyhead[R]{\includegraphics[width=1.3cm]{logo-nero.png}}%
  \fancyhead[L]{\emph{SpaghettiCode}\\\placeholderTitle{}}%
}
% setup di header e footer nelle pagine col numero
\fancypagestyle{usual}{%
  \fancyhf{}%
  \fancyhead[R]{\includegraphics[width=1.3cm]{logo-nero.png}}%
  \fancyhead[L]{\emph{SpaghettiCode}\\\placeholderTitle{}}%
  \fancyfoot[R]{\thepage\ di~\pageref{LastPage}}%
}
\setlength{\headheight}{1.8cm}

\newpage
\pagestyle{nopage}

\setcounter{table}{-1}


%REGISTRO DELLE MODIFICHE

\section*{Registro delle modifiche}%
\label{sec:registro_delle_modifiche}

\rowcolors{2}{white!80!lightgray!90}{white}
\renewcommand{\arraystretch}{2} % allarga le righe con dello spazio sotto e sopra
\begin{longtable}[H]{>{\centering\bfseries}m{2cm} >{\centering}m{3.5cm} >{\centering}m{2.5cm} >{\centering}m{3cm} >{\centering\arraybackslash}m{5cm}}
  \rowcolor{lightgray}
  {\textbf{Versione}} & {\textbf{Nominativo}} & {\textbf{Ruolo}} & {\textbf{Data}} & {\textbf{Descrizione}}  \\
  \endfirsthead%
  \rowcolor{lightgray}
  {\textbf{Versione}} & {\textbf{Nominativo}}  & {\textbf{Ruolo}} & {\textbf{Data}} & {\textbf{Descrizione}}  \\
  \endhead%
  \modifiche{}%
\end{longtable}
% section registro_delle_modifiche (end)

\newpage
\thispagestyle{nopage}
\pagenumbering{roman}
\tableofcontents

\elencoFigure{}%

\elencoTabelle{}%

\newpage

\pagestyle{usual}
\pagenumbering{arabic}


\section{Introduzione}%
\label{sec:introduzione}

\subsection{Scopo del documento}%
\label{sub:scopo_del_documento}
Il documento ha come scopo presentare i metodi di verifica e validazione adottati dal gruppo SpaghettiCode per garantire qualità nel prodotto e nel processo. Per poterle garantire verrà usato un sistema di verifica continua che permetta l'individuazione di errori nel minor tempo possibile e con estrema facilità, al fine di risolverli rapidamente ed evitare sprechi di tempo.\\
Questo documento verrà redatto seguendo una filosofia incrementale: i contenuti iniziali sono da considerarsi incompleti, verranno sottoposti a significative modifiche durante lo svolgimento del progetto. 

\subsection{Scopo del prodotto}%
\label{sub:scopo_prodotto}
Il capitolato C4 - HD Viz nasce dalla necessità di trasformare grosse moli di dati multidimensionali in grafici che diano la possibilità di interpretare le informazioni o apprenderne di nuove. Il gruppo SpaghettiCode si offre quindi di sviluppare la web-application commissionata dall’azienda Zucchetti S.p.A. seguendo le tecnologie richieste dal proponente. 


\subsection{Glossario}%
\label{sub:glossario}
Per aiutare il lettore nella comprensione di tale documento verrà allegato un Glossario. Ogni parola contenuta in esso verrà qui indicata con una G a pedice.


\subsection{Riferimenti}%
\label{sub:riferimenti}
\begin{itemize}
\item Norme di progetto: Norme di progetto v1.0.0
\item Regolamento organigramma e specifica tecnico-economica: \url{https://www.math.unipd.it/~tullio/IS-1/2020/Progetto/RO.html };
\item Capitolato d’appalto C4: \url{https://www.math.unipd.it/~tullio/IS-1/2020/Progetto/C4.pdf};
\item Slide "Gestione di progetto": \url{https://www.math.unipd.it/~tullio/IS-1/2020/Dispense/L06.pdf};
\end{itemize}

\subsection{Informativi}%
\label{sub:info}
\begin{itemize}
\item ISO/IEC 9126 \url{https://en.wikipedia.org/wiki/ISO/IEC_9126}
\item ISO/IEC 12207: \url{math.unipd.it/~tullio/IS-1/2009/Approfondimenti/ISO_12207-1995.pdf}
\item Slide del corso di Ingegneria del Software, qualità software: \url{https://www.math.unipd.it/~tullio/IS-1/2020/Dispense/L12.pdf}
\item Slide del corso di Ingegneria del Software, qualità di processo: \url{https://www.math.unipd.it/~tullio/IS-1/2020/Dispense/L13.pdf}
\item Slide del corso di Ingegneria del Software, verifica e validazione: introduzione \url{https://www.math.unipd.it/~tullio/IS-1/2020/Dispense/L14.pdf}
\item Indice Gulpease: \url{https://it.wikipedia.org/wiki/Indice_Gulpease}
\end{itemize}


\newpage
\section{Qualità del processo}
\label{sec:qproc}
\subfile{componenti/2-qualita_processo.tex}

\newpage
\section{Qualità del prodotto}
\label{sec:qprod}
\subfile{componenti/3-qualita_prodotto.tex}

\newpage
\usepackage[toc,page]{appendix} 
\begin{appendices}
\section{Specifica dei test}
\label{sec:spectest}
\subfile{componenti/4-specifica_test.tex}

\newpage
\section{Resoconto di attività di verifica}
\label{sec:resoconto}
%\subfile{componenti/resoconto_attivita.tex}

\newpage
\section{Lista di Controllo}
\label{sec:lista}
%\subfile{componenti/lista_controllo.tex}

\newpage
\section{Valutazioni per il miglioramento}
\label{sec:valutazioni}
\end{appendices}
%\subfile{componenti/valutazioni_miglioramento.tex}


\end{document}

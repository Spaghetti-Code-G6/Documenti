\documentclass[../piano_di_qualifica.tex]{subfiles}

\begin{document}
Nello svolgimento del progetto i processi faranno uso di criteri di qualità ispirati allo standard ISO/IEC/IEEE 12207:1995. Attraverso essi è possibile garantire che lo svolgimento dei processi sia migliorativo e che il cliente ottenga un prodotto di qualità. \\
Durante il periodo di sviluppo si dovranno rispettare i vari parametri relativi al:
\begin{itemize}
\item Budget: per evitare differenze eccessive rispetto al costo preventivato si seguiranno le metriche di seguito descritte;
\item Calendario: assicurare la pianificazione adatta ai compiti da svolgere
\end{itemize}

\subsection{Metriche di budget}%
\label{sub:metr_bud}
Il budget totale non deve superare del 5\% il preventivo stabilito.

\subsection{Metriche di pianificazione} %
\label{sub: metr_pianif}
Nella pianificazione descritta nel Piano di Progetto si è tenuto conto di un ampio margine per far fronte ad eventuali ritardi. Per ogni step del progetto vi sarà una pianificazione fatta a priori dove verranno stabiliti micro-processi da svolgere e le tempistiche entro cui svolgerli. Il ritardo aggiuntivo per ogni compito assegnato non deve essere superiore a 4 giorni. 

%\subsection{Produzione dei documenti}
%\label{sub:prod_doc}
%Ogni documento deve attestare determinate fasi del ciclo di vita; sarà quindi compito del verificatore accertarsi che esse vengano rispettate. Inoltre il verificatore dovrà controllare frequentemente i prodotti al fine di individuare il prima possibile eventuali errori e ridurre eventuali rischi. 


\end{document}
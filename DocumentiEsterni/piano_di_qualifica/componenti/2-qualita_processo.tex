\documentclass[../piano_di_qualifica.tex]{subfiles}

\begin{document}
Nello svolgimento del progetto i processi faranno uso di criteri di qualità ispirati allo standard ISO/IEC/IEEE 12207:1995. Attraverso essi è possibile garantire che lo svolgimento dei processi sia migliorativo e che il cliente ottenga un prodotto di qualità. In questa sezione verranno illustrati i livelli di qualità accettabili e ottimali su base delle metriche scelte nel documento Norme di Progetto v1.0.0.



\subsection{Pianificazione del progetto, organizzazione e struttura}%
\label{sub:pianif_org_str}
\subsubsection{Varianza della pianificazione}
Nella pianificazione descritta nel Piano di Progetto si è tenuto conto di un ampio margine per far fronte ad eventuali ritardi. Per ogni fase del progetto vi sarà una pianificazione dove verranno stabiliti micro-processi da svolgere, contrassegnati da Issue. Si avrà traccia dell’andamento di questi micro-processi tramite la chiusura delle Issue. 

\subsubsection{Varianza dei costi}
Ad ogni componente è assegnata una tariffa oraria entro la quale deve stare. Questa tariffa prevede un piccolo margine in caso di ritardo; qualora invece il ritardo fosse superiore a quello preventivato ogni membro non deve superare i 200€ di tariffa oraria.

\subsection{Analisi}%
\label{sub:analisi_pq}
\subsubsection{Requisiti obbligatori}
Tutti i requisiti obbligatori devono essere soddisfatti entro la consegna del progetto

\subsubsection{Requisiti desiderabili}
I requisiti desiderabili, non essendo obbligatori, possono non essere soddisfatti. Tuttavia il gruppo si impegna a rispettare gli accordi presi con il proponente. 

\subsection{Produzione dei documenti}
\label{sub:prod_doc}
Ogni documento deve attestare determinate fasi del ciclo di vita; sarà quindi compito del verificatore accertarsi che esse vengano rispettate. Inoltre il verificatore dovrà controllare frequentemente i prodotti al fine di individuare il prima possibile eventuali errori e ridurre eventuali rischi. 


\end{document}
\documentclass[../piano_di_qualifica.tex]{subfiles}

\begin{document}

Per garantire la qualità di prodotto è necessario stabilire delle metriche per l’esecuzione e il soddisfacimento dei test. Tuttavia in questa fase di del progetto è prematuro stabilire delle metriche precise e complete. 


\begin{table}[!ht]
\centering
\begin{tabular}{|c|c|c|c|}
		\hline
		\rowcolor{lightgray}
		\textbf{Metrica} & \textbf{Nome} &  \textbf{Valore accettabile} & \textbf{Valore preferibile} \\
		\hline 
		MTS1 & Test eseguiti in rapporto ai requisiti & 100\% & 100\% \\
 		\hline
		MTS2 & Percentuale test passati & 85\% & 100\% \\
		\hline
\end{tabular}
\caption{Metriche dei test}
\end{table}

\subsection{Test di accettazione}%
\label{sub:test_accett}
Sono test che dimostrano che il prodotto realizzato soddisfa i requisiti concordati con il proponente. 
Questi test si dividono in test di sistema, test di integrazione e test di unità.

\subsection{Test di sistema}%
\label{sub:test_sist}
Per assicurare che il progetto rispetti i requisiti identificati nel documento Analisi dei Requisiti verranno eseguiti i seguenti test: \\
(inserire test relativi ai requisiti)

\subsection{Test di integrazione}%
\label{sub:test_int}
Le specifiche di questi test verranno scritte successivamente.

\subsection{Test di unità}%
\label{sub:test_unit}
Le specifiche di questi test verranno scritte successivamente.


\end{document}
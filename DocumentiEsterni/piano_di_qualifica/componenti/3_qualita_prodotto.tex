\documentclass[../piano_di_qualifica.tex]{subfiles}

\begin{document}

Per garantire la qualità del prodotto il gruppo ha deciso di fare riferimento allo \glossario{standard
ISO/IEC 9126}, il quale regolamenta le modalità con cui produrre un software di buona qualità. Di questo standard il gruppo sceglie di seguire alcune delle metriche esposte, e sceglie di stabilirne altre da seguire per rendere il prodotto qualitativamente valido; qui di seguito vengono elencate.

\subsection{Modello di qualità}
Il gruppo intende seguire a pieno le caratteristiche descritte nello standard adottato circa il modello di qualità.\par
Il prodotto che SpaghettiCode intende rilasciare deve essere:\par
\begin{enumerate}
\item Funzionale: è la capacità di un prodotto software di fornire funzioni che soddisfino esigenze stabilite, necessarie per operare sotto condizioni specifiche.
	\begin{itemize}
	\item Il prodotto software sarà in grado di fornire un appropriato insieme di funzioni per i compiti prefissati dall’utente; sarà in grado di fornire i risultati concordati con il proponente; sarà in grado di interagire ed operare con più sistemi specificati dal proponente; sarà in grado di proteggere informazioni negando accessi non autorizzati, in base agli accordi presi con il proponente.
	\end{itemize}
\item Affidabile: è la capacità del prodotto software di mantenere uno specificato livello di prestazioni quando usato in date condizioni e per un dato periodo.
	\begin{itemize}
	\item Il prodotto software sarà in grado di evitare errori e malfunzionamenti; in caso questi dovessero presentarsi sarà possibile recuperare i dati su cui si stava lavorando; sarà in grado di aderire agli standard delineati con il proponente.
	\end{itemize}
\item Efficiente: è la capacità di fornire appropriate prestazioni relativamente alla quantità di risorse usate
	\begin{itemize}
	\item Il prodotto software sarà in grado di fornire adeguati tempi di risposta, elaborazione e velocità di attraversamento; sarà in grado di sfruttare adeguatamente le risorse e sarà efficiente.
	\end{itemize}
\item Usabile: è la capacità del prodotto software di essere capito, appreso, usato dall'utente
	\begin{itemize}
	\item Il prodotto software sarà di facile comprensione per l’utente; assieme ad esso verrà consegnato il manuale d’utente per consentire a chiunque il suo utilizzo; 
	\end{itemize}
\item Manutenibile è la capacità del software di essere modificato, includendo correzioni, miglioramenti o adattamenti.
	\begin{itemize}
	\item Il prodotto software sarà in grado di essere analizzato lato codice; sarà in grado di essere modificato e soggetto ad evoluzioni; sarà facilmente testabile per verificare le modifiche apportate
	\end{itemize}
\item Portabile: è la capacità del software di essere trasportato da un ambiente di lavoro ad un altro.
	\begin{itemize}
	\item il prodotto software sarà in grado di essere adattato a diversi ambienti sulla base degli accordi presi con il proponente
	\end{itemize}
\end{enumerate}

\emph{Poichè è ancora prematuro definire delle metriche relative al codice scritto, il gruppo si riserva di integrarle in futuro.}

\subsubsection{Completezza dell'implementazione}
Si intende la percentuale di requisiti implementati rispetto a quelli dichiarati .
% C=  (1- (Nfni / Nfi))*100

dove
\begin{itemize}
\item N\textsubscript{FnI} è il numero di funzionalità non implementate ;
\item N\textsubscript{FI} è il numero di funzionalità implementate ;
\end{itemize}

Per quanto riguarda i requisiti obbligatori il gruppo intende implementare il 100\% dei requisiti concordati. \\
Per quanto riguarda i requisiti opzionali il gruppo intende implementare il 100\% dei requisiti concordati, ma si pone come soglia di accettazione l’80\%. 

\subsubsection{Densità di errori}
La densità di errori è l'abilità del prodotto di resistere a malfunzionamenti. Si calcola così:\par

% M= (Ner / Nte ) * 100

dove
\begin{itemize}
\item N\textsubscript{ER} indica il numero di errori rilevati durante il testing;
\item N\textsubscript{TE} indica il numero di test eseguiti;
\end{itemize}

Il valore preferibile è lo 0\%; il valore accettabile è inferiore al 10\%.

%\subsubsection{Metriche di usabilità}
%Facilità di utilizzo: è la facilità con cui l’utente raggiunge ciò che vuole; viene rappresentata tramite il numero di click necessari per arrivare al contenuto desiderato.\\
%\begin{itemize}
%\item misurazione: click per raggiungere l'obiettivo;
%\item valore preferibile: inferiore a 6;
%\item valore accettabile: inferiore a 10;
%\end{itemize}

\subsubsection{Qualità dei documenti}

Tutti i documenti devono essere leggibili e comprensibili, e per renderli tali devono essere corretti dal punto di vista lessicografico, grammaticale e semantico.\par
Per misurare la leggibilità dei documenti si è deciso di usare l’indice di Gulpease. L'Indice Gulpease è un indice di leggibilità di un testo, è tarato sulla lingua italiana, e considera due variabili linguistiche: la lunghezza della parola e la lunghezza della frase rispetto al numero delle lettere.\par
La formula è la seguente:\par
inserire equazione corretta:\par

%IG= $$ 89 +\frac{300 $\cdot$ $|$frasi$|$ - 10 $\cdot$ $|$lettere$|$} {$|$parole$|$} $$

\begin{itemize}
\item QM-PROD-1 INDICE DI GULPEASE;
\item QM-PROD-2 Correttezza ortografica.
\end{itemize}


\begin{table}[!ht]
\centering
\begin{tabular}{|c|c|c|}
		\hline
		\rowcolor{lightgray}
		\textbf{ID metrica} & \textbf{Valore preferibile} & \textbf{Valore accettabile} \\
		\hline 
		QM-PROD-1 (GULP) & \(\ge 80\) & \(\ge 60\) \\
 		\hline
		QM-PROD-2 (CORT) & \(= 0\) & \(= 0\) \\
		\hline
\end{tabular}
\caption{Indici di qualità per le metriche di comprensione del prodotto}
\end{table}



\end{document}
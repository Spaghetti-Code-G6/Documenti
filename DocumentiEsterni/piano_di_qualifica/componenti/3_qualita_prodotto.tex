\documentclass[../piano_di_qualifica.tex]{subfiles}

\begin{document}

Per garantire la qualità del prodotto il gruppo ha deciso di fare riferimento allo \glossario{standard
	ISO/IEC 9126}, il quale regolamenta le modalità con cui produrre un software di buona qualità. Di questo standard il gruppo sceglie di seguire alcune delle metriche esposte, e sceglie di stabilirne altre da seguire per rendere il prodotto qualitativamente valido; qui di seguito vengono elencate. \\
Saranno adottati i seguenti codici identificativi:\par

\begin{center}
	\textbf{M[Metrica]QP[Qualità Prodotto]ID[Indice]} \\
	\textbf{O[Obiettivo]QP[Qualità Prodotto]ID[Indice]}
\end{center}

\subsection{Modello di qualità}
Il gruppo intende seguire a pieno le caratteristiche descritte nello standard adottato circa il modello di qualità.\par
Il prodotto che \emph{SpaghettiCode} intende rilasciare deve essere:\par
\begin{enumerate}
	\item Funzionale: è la capacità di un prodotto software di fornire funzioni che soddisfino esigenze stabilite, necessarie per operare sotto condizioni specifiche.
	      \begin{itemize}
		      \item Il prodotto software sarà in grado di fornire un appropriato insieme di funzioni per i compiti prefissati dall’utente; sarà in grado di fornire i risultati concordati con il proponente; sarà in grado di interagire ed operare con più sistemi specificati dal proponente; sarà in grado di proteggere informazioni negando accessi non autorizzati, in base agli accordi presi con il proponente.
	      \end{itemize}
	\item Affidabile: è la capacità del prodotto software di mantenere uno specificato livello di prestazioni quando usato in date condizioni e per un dato periodo.
	      \begin{itemize}
		      \item Il prodotto software sarà in grado di evitare errori e malfunzionamenti; in caso questi dovessero presentarsi sarà possibile recuperare i dati su cui si stava lavorando; sarà in grado di aderire agli standard definiti con il proponente.
	      \end{itemize}
	\item Efficiente: è la capacità di fornire appropriate prestazioni relativamente alla quantità di risorse usate
	      \begin{itemize}
		      \item Il prodotto software sarà in grado di fornire adeguati tempi di risposta, elaborazione e velocità di attraversamento; sarà in grado di sfruttare adeguatamente le risorse e sarà efficiente.
	      \end{itemize}
	\item Usabile: è la capacità del prodotto software di essere capito, appreso, usato dall'utente
	      \begin{itemize}
		      \item Il prodotto software sarà di facile comprensione per l’utente; assieme ad esso verrà consegnato il manuale d’utente per consentire a chiunque il suo utilizzo;
	      \end{itemize}
	\item Manutenibile: è la capacità del software di essere modificato, includendo correzioni, miglioramenti o adattamenti.
	      \begin{itemize}
		      \item Il prodotto software sarà in grado di essere analizzato lato codice; sarà in grado di essere modificato e soggetto ad evoluzioni; sarà facilmente testabile per verificare le modifiche apportate
	      \end{itemize}
	\item Portabile: è la capacità del software di essere trasportato da un ambiente di lavoro ad un altro.
	      \begin{itemize}
		      \item il prodotto software sarà in grado di essere adattato a diversi ambienti sulla base degli accordi presi con il proponente
	      \end{itemize}
\end{enumerate}

\subsection{Prodotti}
Per prodotto si intende ciò che è concretamente utilizzabile, consultabile o eseguibile. In questo caso si parla dei documenti e del software.

\subsubsection{Documenti}
Come trattato nel precedente capitolo i documenti devono essere letti e capiti da chiunque abbia almeno la licenza media.\\
Gli obiettivi per un documento sono quindi: \\

\setlength{\parindent}{0pt}\textbf{Obiettivi:}
\smallbreak
\begin{itemize}
	\item \textbf{OQP01 Leggibilità del testo}: i documenti devono essere letti in modo fluido evitando periodi troppo lunghi \\
	\item \textbf{OQP02 Correttezza ortografica}: i documenti non devono presentare errori
\end{itemize}

\textbf{Metriche:}
\smallbreak
\begin{itemize}
	\item \textbf{MQP01 Indice di Gulpease};
	\item \textbf{MQP02 Errori ortografici};
\end{itemize}

Per verificare questi requisiti si farà ricorso all’indice di Gulpease. L'Indice Gulpease è un indice di leggibilità di un testo, è tarato sulla lingua italiana, e considera due variabili linguistiche: la lunghezza della parola e la lunghezza della frase rispetto al numero delle lettere.\\
La formula è la seguente:\par
\begin{center}
	$G = 89\ +\ \frac{300 * (numero\ di\ frasi) - 10 * (numero\ di\ lettere)}{numero\ di\ parole} $
\end{center}
Come descritto nella seguente tabella, ci aspettiamo che i documenti siano entro uno specifico range di valori, e che siano correttezza, quindi che la quantità di errori presenti sia 0. \par

\begin{table}[!ht]
	\centering
	\begin{tabular}{|c|c|c|}
		\hline
		\rowcolor{lightgray}
		\textbf{ID metrica}      & \textbf{Valore preferibile} & \textbf{Valore accettabile} \\
		\hline
		MQP01 Indice di Gulpease & \(\ge 80\)                  & \(\ge 60\)                  \\
		\hline
		MQP02 Errori ortografici & \(= 0\)                     & \(= 0\)                     \\
		\hline
	\end{tabular}
	\caption{Indici di qualità per le metriche di comprensione del prodotto}
\end{table}

\paragraph{Software}
Il software rilasciato deve essere di qualità, e per renderlo tale si farà riferimento al \emph{Modello di qualità}. \\
Di seguito presentiamo alcune delle metriche che intendiamo adottare. \\
\emph{Questo paragrafo verrà ampliato in futuro, in quanto attualmente è ancora prematuro definire delle metriche più complete.} \\

\smallbreak
\textbf{Obiettivi:}
\smallbreak
\begin{itemize}
	\item \textbf{OQP03 Assenza di bug}: il prodotto non deve presentare bug;
	\item \textbf{OQP04 Assenza di errori}: il prodotto deve presentare meno errori possibile;
	\item \textbf{OQP05 Assenza di file senza intestazione}: non ci devono essere file senza intestazione;
	\item \textbf{OQP05 Usabilità del prodotto}: il prodotto deve essere facilmente usabile;
\end{itemize}

\textbf{Metriche:}
\smallbreak
\begin{itemize}
	\item \textbf{MQP03 Presenza di bug};
	\item \textbf{MQP04 Densità di errori};
	\item \textbf{MQP05 File senza intestazione};
	\item \textbf{MQP06 Usabilità del prodotto};
\end{itemize}


\paragraph{MQP04 - Densità di errori}
La densità di errori è l'abilità del prodotto di resistere a malfunzionamenti. Si calcola così:\par

\begin{center}
	$M = \frac{N_{er}}{N_{te}} * 100$
\end{center}

dove
\smallbreak
\begin{itemize}
	\item N\textsubscript{ER} indica il numero di errori rilevati durante il testing;
	\item N\textsubscript{TE} indica il numero di test effettuati;
\end{itemize}

Il valore preferibile è lo 0\%; il valore accettabile è inferiore al 10\%.

\paragraph{MQP06 Usabilità del prodotto}
L'usabilità del prodotto è la facilità con cui l’utente raggiunge ciò che vuole; viene rappresentata tramite il numero di click necessari per arrivare al contenuto desiderato, infatti un software poco annidato è più usabile per l'utente medio.
\smallbreak
\begin{itemize}
	\item misurazione: quantità di click per raggiungere l'obiettivo;
	\item valore preferibile: inferiore a 6;
	\item valore accettabile: inferiore a 10;
\end{itemize}

\subsection{Tabella qualità di prodotto}
La seguente tabella indica gli obiettivi di qualità che i prodotti devono possedere.\\
Ogni obiettivo di qualità è indicato con:
\smallbreak
\begin{itemize}
	\item \textbf{Obiettivo}: contrassegnato dal suo codice identificativo;
	\item \textbf{Metrica}: contrassegnato dal suo codice identificativo, indica come sarà garantita la qualità dell'obiettivo;
	\item \textbf{Accettabile}: indica il valore sotto il quale non sarà garantita la qualità;
	\item \textbf{Preferibile}: indica il valore a cui si punta per avere la massima qualità;
\end{itemize}

\begin{table}[!ht]
	\centering
	\begin{tabular}{|l|l|c|c|}
		\hline
		\rowcolor{lightgray}
		\textbf{Obiettivo}                    & \textbf{Metrica}              & \textbf{Accettabile} & \textbf{Preferibile} \\
		\hline
		OQP01 Leggibilità del testo           & MQP01 Indice di Gulpease      & \(\ge 60\)           & \(\ge 80\)           \\
		\hline
		OQP02 Correttezza ortografica         & MQP02 Errori ortografici      & 0                    & 0                    \\
		\hline
		OQP03 Assenza di bug                  & MQP03 Presenza di bug         & 0                    & 0                    \\
		\hline
		OQP04 Assenza di errori               & MQP04 Densità di errori       & 10\%                 & 0\%                  \\
		\hline
		OQP05 Assenza file senza intestazione & MQP05 File senza intestazione & 0                    & 0                    \\
		\hline
		OQP06 Usabilità del prodotto          & MQP06 Usabilità del prodotto  & \(\leq10\) click     & \(\leq6\) click      \\
		\hline
	\end{tabular}
	\caption{Metriche qualità di prodotto}
\end{table}



\end{document}
\documentclass[../piano_di_qualifica.tex]{subfiles}

\begin{document}

Per garantire la qualità di prodotto è necessario stabilire delle metriche per l’esecuzione e il soddisfacimento dei test. Tuttavia in questa fase di del progetto è prematuro stabilire delle metriche precise e complete.


\begin{table}[!ht]
	\centering
	\begin{tabular}{|c|c|c|c|}
		\hline
		\rowcolor{lightgray}
		\textbf{Metrica} & \textbf{Nome}                          & \textbf{Valore accettabile} & \textbf{Valore preferibile} \\
		\hline
		MTS1             & Test eseguiti in rapporto ai requisiti & 100\%                       & 100\%                       \\
		\hline
		MTS2             & Percentuale test passati               & 85\%                        & 100\%                       \\
		\hline
	\end{tabular}
	\caption{Metriche dei test}
\end{table}

\subsection{Test di accettazione}%
\label{sub:test_accett}
Sono test che dimostrano che il prodotto realizzato soddisfa i requisiti concordati con il proponente.
Questi test si dividono in test di sistema, test di integrazione e test di unità.

\subsection{Test di sistema}%
\label{sub:test_sist}
Per assicurare che il progetto rispetti i requisiti identificati nel documento Analisi dei Requisiti verranno eseguiti i seguenti test. Attualmente questo documento deve considerarsi incompleto, altri test verranno preventivati in futuro. \par

\begin{center}
	\begin{longtable}{|c|c|p{8.5cm}|c|}
		\hline
		\rowcolor{lightgray}
		{\textbf{Codice}}   & {\textbf{Riferimento}} & {\textbf{Descrizione}}  & {\textbf{implementazione}} \\
		\hline
		TS0    & RF01             & Si verifichi che l'utente possa inserire dati nel sistema                   & NI   \\ \hline
		TS1    & UC1 RFO2      & Si verifichi che l'utente possa importare dati tramite file .csv           & NI   \\ \hline
		TS2    & UC1.1 RFO3   & Si verifichi che l'utente possa importare dati tramite un database esterno        & NI   \\ \hline
		TS3    & UC1.3 RF04    & Si verifichi che l'utente possa aprire un collegamento con un server dati per accedere ad uno dei suoi database  & NI  \\ \hline
		TS4    & UC1.3.1 RFO5 & Si verifichi che l’utente possa poter immettere l’indirizzo del server & NI \\ \hline
		TS5    & UC1.3.1.1 RFO6 & Si verifichi che l’utente possa poter immettere il nome di accesso al server & NI \\ \hline
		TS6    & UC1.3.1.2 RFO7 & Si verifichi che l’utente possa poter immettere la password di accesso al server & NI \\ \hline
		TS7    & UC1.3.1.3 RFO8 & Si verifichi che l’utente possa poter importare i dati nel sistema mediante la ricerca su un database tra quelli disponibili & NI \\ \hline
		TS8    & UC1.3.2 RFO9 & Si verifichi che l’utente possa inserire il metadato relativo alla categoria del dato delle dimensioni del dataset. & NI \\ \hline
		TS9    & UC1.4 RFO10 & Si verichi che l'utente possa creare un grafico di sua scelta. & NI \\ \hline
		TS10  & UC2.1 RFO11& Si verichi che l'utente possa  selezionare l’opzione di costruzione di un grafico di tipo Scatter plot matrix & NI \\ \hline
		TS11  & UC2.2 RFO12 & Si verichi che l'utente possa selezionare l’opzione di costruzione di un grafico di tipo Force Field & NI \\ \hline
		TS12  & UC2.3 RFD13 &Si verichi che l'utente possa selezionare l’opzione di costruzione di un grafico di tipo Heat Map & NI \\ \hline
		TS13  & UC2.4 RFO14 & Si verichi che l'utente possa selezionare l’opzione di costruzione di un grafico di tipo PLMA & NI \\ \hline
		TS14  & UC2.5 RFO15 & Si verichi che l'utente possa selezionare l’opzione di costruzione di un grafico di tipo Distance Map & NI \\ \hline
		TS15 & UC2.6 RFO16 & Si verichi che l'utente possa modificare le proprietà dei metadati associati al dataset  & NI \\ \hline
		TS16 & UC3.1 RFO17 & Si verichi che l'utente possa modificare il tipo dei metadati associati al dataset & NI \\ \hline
		TS17 & UC3.2 RFO18 & Si verichi che l'utente possa modificare il metadato relativo alla visibilità di una dimensione del dataset & NI \\ \hline
		TS18  & RQO1  & Si verifichi che l'applicativo sia accompaganto dalla documentazione richiesta   & NI \\ \hline
      		TS19  & RQO2  & Si verifichi che l'applicativo sia accompagnato da un manualte di utilizzo  & NI \\ \hline
      		TS20  & RQO3  & Si verifichi che l'applicativo sia accompagnato da un manualte tecnico    & NI \\ \hline
      		TS21  & RQO4  & Si verifichi che il manuale di utlizzo sia scritto in italiano e in formato pdf     & NI \\ \hline
      		TS22  & RQO5  & Si verifichi che il manuale tecnico sia scritto in italiano e in formato pdf  & NI \\ \hline
      		TS23  & RQO7  & Si verifichi che il prodotto rispetti le norme di progetto    & NI \\   \hline
			TS24   & RV01  & Si verifichi che l'applicativo sia sviluppato con le tecnologie HTML/CSS/JavaScript  & NI \\ \hline	
      		TS25   & RV02   & Si verifichi che l'applicativo sia sviluppato con la libreria D3.js   & NI \\ \hline
      		TS26   & RV03    & Si verifichi che l'applicativo abbia una parte di server sviluppata in Java o Node.js   & NI \\ \hline
		TS27  & RVO4 & Si verichi che l’applicativo deve poter visualizzare dati ad almeno 15 dimensioni & NI \\ \hline
		TS28  & RVO5 & Si verichi che l’applicativo deve presentare la modalità di visualizzazione Scatter Plot Matrix & NI \\ \hline
		TS29  & RVO6 & Si verichi che l’applicativo deve presentare la modalità di visualizzazione Force Field & NI \\ \hline
		TS30  & RVD7 & Si verichi che l’applicativo deve presentare la modalità di visualizzazione Heat Map & NI \\ \hline
		TS31  & RVO8 & Si verichi che l’applicativo deve presentare la modalità di visualizzazione Proiezione Lineare Multi Asse & NI \\ \hline
		TS32  & RVO9 & Si verichi che l’applicativo deve presentare la modalità di visualizzazione Distance Map & NI \\ \hline
		TS33  & RVO10 &  Si verifichi che è possibile ordinare tramite clustering gerarchico gli elemnti nella visualizzazione "heat map" & NI \\ \hline
		\hline
		\rowcolor{white}
		\caption{Tabella contenente un riepilogo dei test di sistema}
	\end{longtable}

\end{center}


\subsection{Test di integrazione}%
\label{sub:test_int}
Le specifiche di questi test verranno redatte successivamente.

\subsection{Test di unità}%
\label{sub:test_unit}
Le specifiche di questi test verranno redatte successivamente.


\end{document}
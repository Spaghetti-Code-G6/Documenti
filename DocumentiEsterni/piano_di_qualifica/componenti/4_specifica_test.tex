\documentclass[../piano_di_qualifica.tex]{subfiles}

\begin{document}

Per garantire la qualità di prodotto è necessario stabilire delle metriche per l’esecuzione e il soddisfacimento dei test. Tuttavia in questa fase di del progetto è prematuro stabilire delle metriche precise e complete.


\begin{table}[!ht]
	\centering
	\begin{tabular}{|c|c|c|c|}
		\hline
		\rowcolor{lightgray}
		\textbf{Metrica} & \textbf{Nome}                          & \textbf{Valore accettabile} & \textbf{Valore preferibile} \\
		\hline
		MTS1             & Test eseguiti in rapporto ai requisiti & 100\%                       & 100\%                       \\
		\hline
		MTS2             & Percentuale test passati               & 85\%                        & 100\%                       \\
		\hline
	\end{tabular}
	\caption{Metriche dei test}
\end{table}

\subsection{Test di accettazione}%
\label{sub:test_accett}
Sono test che dimostrano che il prodotto realizzato soddisfa i requisiti concordati con il proponente.
Questi test si dividono in test di sistema, test di integrazione e test di unità.

\subsection{Test di sistema}%
\label{sub:test_sist}
Per assicurare che il progetto rispetti i requisiti identificati nel documento Analisi dei Requisiti verranno eseguiti i seguenti test. Attualmente questo documento deve considerarsi incompleto, altri test verranno preventivati in futuro. \par

\begin{center}
	\begin{longtable}{|c|c|p{8.5cm}|c|}
		\hline
		\rowcolor{lightgray}
		{\textbf{Codice}}                & {\textbf{Riferimento}} & {\textbf{Descrizione}}                                                                                            & {\textbf{implementazione}} \\
		\hline
		TS0                              & RF01                   & Si verifichi che l'utente deve poter inserire dati nel sistema                                                    & NI                         \\ \hline
		TS1                              & RFO1.1                 & Si verifichi che l'utente possa importare dati tramite file .csv                                                  & NI                         \\ \hline
		TS2                              & RF01.2                 & Si verifichi che l'utente possa importare dati tramite un database esterno                                        & NI                         \\ \hline
		TS3                              & RF04                   & Si verifichi che l'utente possa scegliere qualche grafico visualizzare                                            & NI                         \\ \hline
		TS4                              & RF04.1                 & Si verifichi che l'utente possa selezionare il grafico Scatter Plot Matrix                                        & NI                         \\ \hline
		TS5                              & RF04.2                 & Si verifichi che l'utente possa selezionare il grafico Force Field                                                & NI                         \\ \hline
		TS6                              & RF04.3                 & Si verifichi che l'utente possa selezionare il grafico Heat Map                                                   & NI                         \\ \hline
		TS7                              & RF04.4                 & Si verifichi che l'utente possa selezionare il grafico Proiezione multi-asse                                      & NI                         \\ \hline
		TS8                              & RFO1                   & Si verifichi che l'utente possa importare dati tramite file .csv                                                  & NI                         \\ \hline
		TS9                              & RFD3                   & Si verifichi che l'utente possa modificare metadati precedentemente inseriti                                      & NI                         \\ \hline
		TS10                             & RFD5                   & Si verifichi che l'utente possa modificare il grafico che sta visualizzando                                       & NI                         \\ \hline
		TS11                             & RDF5.2.1               & Si verifichi che l'utente possa spostare i nodi del grafico Force Field                                           & NI                         \\ \hline
		TS12                             & RDF5.2.2               & Si verifichi che l'utente possa modificare la tipologia di distanza utilizzata per il calcolo della matrice delle
		distanze nel grafico Force Field & NI                                                                                                                                                                      \\ \hline
		TS13                             & RDF5.3.1               & Si verifichi che l'utente possa modificare l'intervallo di colori utilizzato nel grafico Heat Map                 & NI                         \\ \hline
		TS14                             & RDF6                   & Si verifichi che l'utente venga notificato quando si verifica un errore                                           & NI                         \\
		\hline
		\rowcolor{white}
		\caption{Tabella contenente un riepilogo dei test di sistema}
	\end{longtable}

\end{center}


\subsection{Test di integrazione}%
\label{sub:test_int}
Le specifiche di questi test verranno redatte successivamente.

\subsection{Test di unità}%
\label{sub:test_unit}
Le specifiche di questi test verranno redatte successivamente.


\end{document}
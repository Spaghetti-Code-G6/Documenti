\documentclass[../piano_di_qualifica.tex]{subfiles}

\begin{document}

Per garantire la qualità di prodotto è necessario stabilire delle metriche per l’esecuzione e il soddisfacimento dei test. Tuttavia in questa fase di del progetto è prematuro stabilire delle metriche precise e complete.


\begin{table}[!ht]
	\centering
	\begin{tabular}{|c|c|c|c|}
		\hline
		\rowcolor{lightgray}
		\textbf{Metrica} & \textbf{Nome}                          & \textbf{Valore accettabile} & \textbf{Valore preferibile} \\
		\hline
		MTS1             & Test eseguiti in rapporto ai requisiti & 100\%                       & 100\%                       \\
		MTS2             & Percentuale test passati               & 85\%                        & 100\%                       \\
		\hline
	\end{tabular}
	\caption{Metriche dei test}
\end{table}

\subsection{Test di accettazione}%
\label{sub:test_accett}
Sono test che dimostrano che il prodotto realizzato soddisfa i requisiti concordati con il proponente.
Questi test si dividono in test di sistema, test di integrazione e test di unità.

\subsection{Test di sistema}%
\label{sub:test_sist}
Per assicurare che il progetto rispetti i requisiti identificati nel documento \textsc{Analisi dei Requisiti} verranno eseguiti i seguenti test. Attualmente questo documento deve considerarsi incompleto, altri test verranno preventivati in futuro.

\begin{center}
	\begin{longtable}{|c|c|p{8cm}|c|}
		\hline
		\rowcolor{lightgray}
		{\textbf{Codice}} & {\textbf{Riferimento}} & {\textbf{Descrizione}}                                                                                                & {\textbf{implementazione}} \\
		\hline
		\endfirsthead
	
		\hline
		\rowcolor{lightgray}
		{\textbf{Codice}} & {\textbf{Riferimento}} & {\textbf{Descrizione}}                                                                                                & {\textbf{implementazione}} \\
		\hline
		\endhead
		
		\hline
		\multicolumn{4}{|c|}{\emph{Continua alla pagina successiva...}}\\
		\hline
		\endfoot

		\endlastfoot
		TS0               & RFO1                   & Si verifichi che l'utente possa creare un ambiente di lavoro     & NI                         \\
		TS0               & RFO1.1                   & Si verifichi che l'utente possa inserire dati nel sistema & NI                         \\
		TS1               & RFO1.2              & Si verifichi che l'utente possa importare dati tramite file CSV         & NI                         \\
		TS2               & RFO1.3            & Si verifichi che l'utente possa importare dati tramite un database esterno   & NI                  \\
		TS3               & RFO1.3.1      & Si verifichi che l'utente possa aprire un collegamento con un server per poter accedere ad uno dei suoi database & NI                         \\
		TS4               & RFO1.3.1.1                   & Si verifichi che l'utente possa immettere l'indirizzo del server & NI                         \\
		TS5               & RFO1.3.1.2                   & Si verifichi che l'utente possa immettere il nome di accesso al server & NI                         \\
		TS6               & RFO1.3.1.3      & Si verifichi che l'utente possa immettere la password di accesso al server & NI  \\
		TS7               & RFO1.3.2      & Si verifichi che l'utente possa importare i dati nel sistema mediante la ricerca su un databae tra quelli disponibili & NI  \\
		TS8               & RFO1.4      & Si verifichi che l'utente possa inserire i metadati di tipo delle demensioni del dataset & NI  \\
		TS9               & RFO2      & Si verifichi che l'utente possa creare  un grafico & NI  \\
		TS10               & RFO2.1      & Si verifichi che l'utente possa selezionare la costruzione di un grafico di sua scelta & NI  \\
		TS11               & RFO2.2      & Si verifichi che l'utente possa selezionare l'opzione di costruzione di un grafico di tipo Scatter Plot Matrix & NI  \\
		TS12               & RFO2.3     & Si verifichi che l'utente possa selezionare l'opzione di costruzione di un grafico di tipo Force Field & NI  \\
		TS13               & RFO2.4     & Si verifichi che l'utente possa selezionare l'opzione di costruzione di un grafico di tipo Heat Map & NI  \\
		TS14 &  RFO2.5     & Si verifichi che l'utente possa selezionare l'opzione di costruzione di un grafico di tipo PLMA & NI \\
		TS15 &  RFO2.6     & Si verifichi che l'utente possa selezionare l'opzione di costruzione di un grafico di tipo Distance Map & NI \\
		TS16 & RFO3     & Si verifichi che l'utente possa modificare i metadati associati al dataset & NI \\
		TS17 & RFO3.1     & Si verifichi che l'utente possa selezionare di quale dimensione desidera modificare i metadati & NI \\
		TS18 & RFO3.2     & Si verifichi che l'utente possa modificare i metadati di tipo associati al dataset & NI \\
		TS19 & RFO3.3     & Si verifichi che l'utente possa modificare i metadati di visibilità delle dimensioni del dataset & NI \\
		TS17 & RFO4.5.2     & Si verifichi che l'utente possa ordinare la Distance Map & NI \\
		TS18 & RFO4.5.3.1     & Si verifichi che l'utente possa ordinare la Distance Map mediante clustering gerarchico & NI \\
		
		TS19 & RFO4.5.3.2   & Si verifichi che l'utente possa associare un dendrogramma al clustering gerarchico nella Distance Map & NI \\
		TS20 & RFO5     & Si verifichi che l'utente possa essere informato in caso di errori & NI \\
		TS21 & RFO6     & Si verifichi che l'utente possa essere informato in caso di errore durante l'inserimento di un file & NI \\
		TS22 & RFO7     & Si verifichi che l'utente possa essere informato in caso di errore durante l'accesso al database & NI \\
		TS23 & RFO8    & Si verifichi che l'utente possa essere informato in caso di errore se il database non contiene alcun dato & NI \\
		TS24 & RFO9   & Si verifichi che l'utente possa essere informato in caso la query usata per recuperare i dati non ritorni nessun dato & NI \\
		TS25 & RFO10   & Si verifichi che l'utente possa essere informato in caso provi a modificare la visibilità di un dato quando non può farlo & NI \\
		TS27 &  RVO4    & Si verifichi che l'applicativo possa visualizzare dati ad almeno 15 dimensioni & NI \\
		TS28 & RVO5    & Si verifichi che l'applicativo possa presentare la modalità di visualizzazione Scatter Plot Matrix & NI \\
		TS29 & RVO6        & Si verifichi che l'applicativo possa presentare la modalità di visualizzazione Force Field & NI \\
		TS30 & RVO7        & Si verichi che l'applicativo possa presentare la modalità di visualizzazione Heat Map & NI \\
		TS31 & RVO8      & Si verifichi che l'applicativo possa presentare la modalità di visualizzazione Proiezione Lineare Multi Asse & NI \\
		TS32 & RVO9        & Si verifichi che l'applicativo possa presentare la modalità di visualizzazione Distance Map & NI \\
		TS33 & RVO10      & Si verifichi che l'applicativo possa fornire l'ordinamento dei punti mediante clustering gerarchico nel grafico Heat Map & NI \\
		\hline
		\rowcolor{white}
		\caption{Riepilogo dei test di sistema}
	\end{longtable}

\end{center}

\subsection{Test di integrazione}%
\label{sub:test_int}
Le specifiche di questi test verranno redatte successivamente.

\subsection{Test di unità}%
\label{sub:test_unit}
Le specifiche di questi test verranno redatte successivamente.

\end{document}
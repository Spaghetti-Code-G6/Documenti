\documentclass[../piano_di_qualifica.tex]{subfiles}
\begin{document}

Qui di seguito viene presentata la valutazione fatta dai membri del gruppo SpaghettiCode circa il lavoro svolto durante l'attività conclusa.\\
I problemi analizzati riguardano:
\begin{itemize}
    \item l'organizzazione: ovvero problemi relativi alla comunicazione e organizzazione
    \item ruoli: problemi relativi al corretto svolgimento dei ruoli
    \item strumenti di lavoro: problemi relativi all'impiego di strumenti scelti
\end{itemize}

\emph{Questa sezione è attualmente incompleta, verrà integrata col proseguimento del progetto}.

\subsection{Valutazioni sull'organizzazione}
\label{sub:valut_org}
Questo anno particolare ha richiesto molti cambiamenti e flessibilità da parte degli studenti. Sebbene ad oggi la maggior parte dei compiti e 
degli incontri è possibile svolgerla online, non sono mancate alcune difficoltà.

\begin{center}
\begin{longtable}{|p{3cm}|p{4.5cm}|p{4.5cm}|}
		\hline
		\rowcolor{lightgray}
            \textbf{Problema} & \textbf{Descrizione} &  \textbf{Soluzione} \\ 
            \hline 
            Incontri di gruppo & 
            Non sempre tutti sono disponibili per un certo momento concordato &
            Si è deciso di fare una riunione con almeno 5 membri del gruppo \\
            \hline
\caption{Tabella contente le difficoltà sorte nell'organizzazione}
\end{longtable}
\end{center}


\subsection{Valutazioni dei ruoli}
\label{sub:valut_ruoli}
Tutti i ruoli sono e saranno sempre assunti su base volontaria. Essendo la prima volta per ciascun membro, sono emerse alcune difficoltà che tuttavia sono state superate.\par

\begin{center}
\begin{longtable}{|p{3cm}|p{4.5cm}|p{4.5cm}|}
		\hline
		\rowcolor{lightgray}
            \textbf{Ruolo} & \textbf{Problema rilevato} &  \textbf{Contromisura} \\
            \hline 
            Responsabile &
            Per mancata esperienza precedente in questo ambito è stato necessario un supporto  &
            Tutto il gruppo è stato in grado di autogestirsi; ognuno ha scelto attivamente che ruolo assumere e di conseguenza che documenti redigere\\
            \hline
            Analista & 
            Il ruolo di analista è fondamentale e allo stesso tempo delicato. Delineare attualmente i requisiti richiesti è un lavoro difficile in quanto si inizia ad avere consapevolezza della portata del progetto &
            Si è deciso di affidare il ruolo di Analista a più componenti in quanto lavorare in team favorisce la ricerca e la comprensione dei vari aspetti del progetto \\
            \hline
            Verificatore &
            Partire con il piede giusto è fondamentale, quindi si vede necessario l'impiego di più verificatori &
            Si è scelto di affidare il ruolo di Verificatore a più componenti del gruppo in quanto ciascuno riesce a notare aspetti che qualcun altro non percepisce \\
            \hline

\caption{Tabella contenente le valutazioni dei ruoli }
\end{longtable}
\end{center}

\subsection{Valutazioni degli strumenti}
\label{sub:valut_strumenti}

\begin{center}
	\begin{longtable}{|p{3cm}|p{4.5cm}|p{4.5cm}|}
		\hline
		\rowcolor{lightgray}
            \textbf{Strumento} & \textbf{Problema rilevato} &  \textbf{Contromisura} \\
            \hline 
            Version Control System & 
            Non tutti i componenti avevano confidenza con strumenti di versionamento &
            Alcuni colleghi hanno fatto una sorta di lezione per aiutare i meno esperti ad allinearsi al gruppo \\
            \hline
            \LaTeX &
            Quasi nessuno aveva esperienza con questo strumento di scrittura &
            Un membro che aveva già avuto esperienze precedenti ha preparato e fornito al gruppo un template pronto, e ha tracciato una 
            guida di utilizzo \\
            \hline

\caption{Tabella contenente le valutazioni degli strumenti}
\end{longtable}
\end{center}

\end{document}
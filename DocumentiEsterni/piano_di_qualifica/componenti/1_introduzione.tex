\documentclass[../piano_di_qualifica.tex]{subfiles}
\begin{document}

\subsection{Scopo del documento}
Il documento ha come scopo quello di presentare i metodi di \glossario{verifica} e \glossario{validazione} adottati per garantire qualità del prodotto e del processo durante tutta la durata del progetto. Per poter garantire ciò verrà usato un sistema di verifica continua che permetta l'individuazione di eventuali problematiche in modo automatico e facile, al fine di risolverli rapidamente ed ottimizzare al massimo la risorse tempo.

Tale attività è resa possibile attraverso l’impiego di metriche empiriche riproducibili, permettendo così di ottenere risultati quantificabili, oggettivi e misurabili. Infatti, questo documento potrà essere impiegato dal committente per verificare il lavoro svolto dal gruppo, con valutazione oggettiva sul prodotto consegnato.

Questo documento verrà redatto seguendo una \glossario{filosofia incrementale}: i contenuti iniziali sono da considerarsi incompleti, verranno sottoposti a significative modifiche \glossario{just in time} durante lo svolgimento del progetto.

\subsection{Scopo del prodotto}
Il capitolato C4 - HD Viz nasce dalla necessità di trasformare grosse moli di dati multidimensionali in grafici che diano la possibilità di interpretare le informazioni o apprenderne di nuove. Il gruppo SpaghettiCode si offre quindi di sviluppare la web-application commissionata dall’azienda Zucchetti S.p.A. seguendo le tecnologie richieste dal proponente.

\subsection{Glossario}
Per aiutare il lettore nella comprensione di tale documento verrà allegato un Glossario. Ogni parola contenuta in esso verrà qui indicata con una G a pedice.

All'interno del documento ci sono termini che potrebbero presentare significati ambigui o incongruenti a seconda del contesto. Per evitare incomprensioni viene fornito un documento Glossario v1.0.0 contenente la spiegazione dei termini.

Nella seguente documentazione queste parole sono contrasegnate con una "G" a pedice ad ogni prima occorrenza del termine per ogni sezione.

\subsection{Riferimenti}

\subsubsection{Riferimenti normativi}
\begin{itemize}
	\item Norme di Progetto v1.0.0;
	\item Regolamento organigramma e specifica tecnico-economica: \url{https://www.math.unipd.it/~tullio/IS-1/2020/Progetto/RO.html};
	\item Capitolato d’appalto C4: \url{https://www.math.unipd.it/~tullio/IS-1/2020/Progetto/C4.pdf};
	\item Slide "Gestione di progetto": \url{https://www.math.unipd.it/~tullio/IS-1/2020/Dispense/L06.pdf}.
\end{itemize}

\subsubsection{Riferimenti informativi}

\begin{itemize}
	%\item ISO/IEC 9126 \url{https://en.wikipedia.org/wiki/ISO/IEC_9126}
	%\item ISO/IEC 12207: \url{math.unipd.it/~tullio/IS-1/2009/Approfondimenti/ISO_12207-1995.pdf}
	\item Slide del corso di Ingegneria del Software, qualità software: \url{https://www.math.unipd.it/~tullio/IS-1/2020/Dispense/L12.pdf}
	\item Slide del corso di Ingegneria del Software, qualità di processo: \url{https://www.math.unipd.it/~tullio/IS-1/2020/Dispense/L13.pdf}
	\item Slide del corso di Ingegneria del Software, verifica e validazione: introduzione \url{https://www.math.unipd.it/~tullio/IS-1/2020/Dispense/L14.pdf}
	\item Indice Gulpease: \url{https://it.wikipedia.org/wiki/Indice_Gulpease}
\end{itemize}

\end{document}
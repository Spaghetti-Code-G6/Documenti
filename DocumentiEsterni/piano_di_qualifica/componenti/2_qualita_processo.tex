\documentclass[../piano_di_qualifica.tex]{subfiles}

\begin{document}
Nello svolgimento del progetto i processi faranno uso di criteri di qualità ispirati allo \glossario{standard ISO IEC 90003 2014}. Attraverso essi è possibile garantire che lo svolgimento dei processi sia migliorativo e che il cliente ottenga un prodotto di qualità. \\
Durante il periodo di sviluppo si dovranno rispettare i vari parametri relativi al:
\begin{itemize}
\item Budget: per evitare differenze eccessive rispetto al costo preventivato si seguiranno le metriche di seguito descritte;
\item Calendario: assicurare la pianificazione adatta ai compiti da svolgere
\end{itemize}
%
%\subsection{Processi primari}
%\label{sub:proc_prim}
%I processi primari sono quelli relativi all'analisi dei requisiti, progettazione dell'architettura e progettazione di dettaglio.\\
%Durante l'analisi dei requisiti le informazioni sono riportate nel documento Analisi dei requisiti, dove è possibile avere una descrizione dettagliata del sistema, le sue caratteristiche e il suo funzionamento. In questo documento i requisiti sono divisi in desiderabili e opzionali, verranno sottoposti all'attenzione del proponente e verrà stabilito quali verranno portati a termine. Nel successivo capitolo verrà esposta la percentuale di requisiti che si intende soddisfare. \\
%Nella fase di progettazione dell'architettura i requisiti verranno tradotti in un modello architetturale. Tale modello sarà ancora acerbo, tuttavia darà indicazioni su quali possono essere i test che dovranno essere effettuati, e dei quali si parlerà nel capitolo \S4.\\
%Nella fase di progettazione di dettaglio verrà concretizzato il modelllo precedentemente delineato. Gli obiettivi verranno tradotti in codice e dovranno essere rispettate le metriche ad essi relativi. Saranno delineate nuove metriche di standard da seguire al fine di realizzare un prodotto soddisfacente. \\
%Durante ciascuna di queste fasi si dovrà tenere d'occhio il budget preventivato e le pianificazioni realizzate di volta in volta. Di seguito vengono esplicitate le metriche a cui il gruppo si atterrà. 

\subsection{Introduzione}
Durante lo svolgimento di questo progetto, verrano usati i processi, che permetteranno di raggiungere la soddisfazione dei criteri di qualità prefissando un continuo miglioramento. Si è scelto di usare lo standard ISO/IEC 15504 (SPICE) come riferimento per una best practice e di fare uso del metodo PDCA. Così facendo, è possibile garantire l'evoluzione dei processi che, attraverso l'esperienza, si migliorano e assicurano un prodotto di qualità. Vengono di seguito esposti i parametri di qualità tollerabili e preferibili sulla base delle metriche scelte nel documento Norme di Progetto.

\subsection{Metriche di budget}
\label{sub:metr_bud}
Si cercherà di rispettare a pieno il budget preventivato, tuttavia ci si è posto come valore tollerabile una differenza di ${\pm 5\%}$ il totale stabilito.

\subsection{Metriche di pianificazione}
\label{sub: metr_pianif}
Nella \glossario{pianificazione} descritta nel Piano di Progetto si è tenuto conto di un ampio margine per far fronte ad eventuali ritardi. Per ogni step del progetto vi sarà una pianificazione fatta a priori dove verranno stabiliti micro-processi da svolgere e le tempistiche entro cui svolgerli. Il ritardo aggiuntivo per ogni compito assegnato non deve essere superiore a 4 giorni. 

%\subsection{Produzione dei documenti}
%\label{sub:prod_doc}
%Ogni documento deve attestare determinate fasi del ciclo di vita; sarà quindi compito del verificatore accertarsi che esse vengano rispettate. Inoltre il verificatore dovrà controllare frequentemente i prodotti al fine di individuare il prima possibile eventuali errori e ridurre eventuali rischi. 


\end{document}
\documentclass[../piano_di_qualifica.tex]{subfiles}

\begin{document}

\subsection{Scopo}
In questa sezione saranno presentati gli obiettivi di qualità che si vogliono raggiungere, dei processi implementati e dei loro prodotti, così da rendere efficace la valutazione della qualità e di selezionare le migliori metriche.
La qualità di un prodotto è fortemente influenzata dai processi utilizzati, per questo motivo si è deciso di usare come riferimento lo standard ISO/IEC 15504 (SPICE) per la qualità di processo e il metodo PDCA (o ciclo di Deming) per il miglioramento continuo della qualità durante tutto l'arco del progetto.\\
Saranno adottati i seguenti codici identificativi:\par

\begin{center}
	      \textbf{QPR[Qualità di processo]ID[Indice]N[Nome]}\\
		\textbf{MPR[Metrica di processo]ID[Indice]N[Nome]}
\end{center}

Per ogni processo sono elencate le sue funzioni principali, gli obiettivi prefissati per ottenere la qualità desiderata e le metriche adottate per raggiungere quell’obiettivo.

\begin{center}
	\textbf{PROC[Processo]ID[Indice]N [Nome]}
\end{center}

\subsubsection{PROC01 Gestione progetto}
Lo scopo di questo macro-processo è quello di pianificare ed organizzare il lavoro da svolgere per adempiere ai requisiti trovati. L'esito dell'intero progetto è dipendente da questo processo, che deve strutturare il way of working del gruppo.

\textbf{Funzioni:}
\smallbreak
\begin{itemize}
	\item \textbf{Scelta standard}: trovare gli standard che si adattano meglio allo sviluppo di questo progetto;
	\item \textbf{Sviluppo sotto-processi}: i processi devo poter essere associati ad azioni precise;
	\item \textbf{Suddivisione lavoro}: assegnazione di compiti ai membri del gruppo;
	\item \textbf{Programmare scadenze}: stabilire delle baseline;
	\item \textbf{Prevedere formazione}: prevedere dei periodi di autoformazione;
	\item \textbf{Rispettare budget}: rimanere nei limiti di budget stimati durante il preventivo.
\end{itemize}

\textbf{Obiettivi:}
\smallbreak
\begin{itemize}
	\item \textbf{QPR01 Scadenze}: rispettare il più possibile le date di scadenza presenti all'interno del Piano di Progetto;
	\item \textbf{QPR02 Budget}: mantenere le risorse messe a disposizione ad inzio progetto per tutta la sua durata;
	\item \textbf{QPR03 Miglioramento}: i processi devono seguire le fasi del ciclo di vita di Deming per migliorare la qualità;
	\item \textbf{QPR04 Versionamento}: i prodotti devono essere versionati per poter conoscere lo storico delle modifiche e stabilire quale di esse li ha generati.
\end{itemize}

\textbf{Metriche:}
\smallbreak
\begin{itemize}
	\item \textbf{MPR01 Rispetto degli standard};
	\item \textbf{MPR02 Varianza della pianificazione};
	\item \textbf{MPR03 Varianza dei costi}.
\end{itemize}

\subsubsection{PROC02 Analisi}
Questo macro-processo si riferisce ad ogni tipo di analisi presente in questo progetto, non solo l'analisi dei requisiti.

\textbf{Funzioni:}
\smallbreak
\begin{itemize}
	\item \textbf{Individuare requisti}: individuare i requisiti espliciti e impliciti e classificarli;
	\item \textbf{Individuare rischi}: individuare i rischi e classificarli per poterli evitare o per mitigare i loro effetti;
	\item \textbf{Individuare obiettivi}: individuare gli obiettivi per poter perseguire la qualità nel progetto;
	\item \textbf{Individuare norme}: trovare delle norme e poi adattarle per miglioare il lavoro del gruppo.
\end{itemize}

\textbf{Obiettivi:}
\smallbreak
\begin{itemize}
	\item \textbf{QPR05 Adempimento requisiti obbligatori}: tutti i requisiti obbligatori devono essere soddisfatti;
	\item \textbf{QPR06 Adempimento requisiti non obbligatori}: i requisiti non obbligatori, cioè desiderabili e opzionali, potranno essere soddisfatti solo se tutti i requisiti obbligatori sono stati completati prima della fine del progetto;
	\item \textbf{QPR07 Rischi non previsti}: non dovranno verificarsi rischi non previsti che potrebbero portare rallentamenti;
	\item \textbf{QPR08 Rispetto obiettivi}: tutti gli obiettivi dovranno essere rispettati per avere un prodotto di qualità;
	\item \textbf{QPR09 Rispetto delle norme}: tutte le norme dovranno essere seguire, se si necessitasse di fare diversamente, le norme dovranno essere adattate.
\end{itemize}

\textbf{Metriche:}
\smallbreak
\begin{itemize}
	\item \textbf{MPR04 Requisiti obbligatori};
	\item \textbf{MPR05 Requisiti non obbligatori desiderabili};
	\item \textbf{MPR06 Obiettivi soddisfatti};
	\item \textbf{MPR07 Norme rispettate}.
\end{itemize}

Nello specifico il calcolo usato per verificare la percentuale di requisiti soddisfatti sarà il secuente: \par
\begin{center}
	$C = \frac{N_{fni}}{N_{fi}} * 100$
\end{center}

dove
\begin{itemize}
\item N\textsubscript{FnI} è il numero di funzionalità non implementate ;
\item N\textsubscript{FI} è il numero di funzionalità implementate ;
\end{itemize}

\subsubsection{PROC03 Documentazione}
Questo processo ha lo scopo di produrre documenti che riportino le scelte effettuate, gli strumenti utilizzati e le modifiche attuate durante l'intero progetto.

\textbf{Funzioni:}
\smallbreak
\begin{itemize}
	\item \textbf{Redazione documenti}: scrittura dei documenti;
	\item \textbf{Verifica documenti}: controllo dei documenti da parte dei verificatori;
	\item \textbf{Approvazione documenti}: controllo e approvazione finale da parte del responsabile.
\end{itemize}

\textbf{Obiettivi:}
\smallbreak
\begin{itemize}
	\item \textbf{QPR10 Ciclo di vita dei documenti}: i documenti devono rispettare ogni fase del loro ciclo di vita e le scadenze prefissate;
	\item \textbf{QPR11 Leggibilità}: i documenti devono poter essere letti e capiti da chiunque abbia almeno una licenza media.
\end{itemize}

\subsubsection{PROC04 Verifica}
Questo processo ha lo scopo di valutare i prodotti per stabilire se presentano errori, se rispettano gli obiettivi di qualità e se sono corretti nella loro forma e contenuto.

\textbf{Funzioni:}
\smallbreak
\begin{itemize}
	\item \textbf{Verificare funzionalità}:  i prodotti devono risepttare i requisiti prefissati e quindi bisogna controllare che il loro output sia quello atteso;
	\item \textbf{Verifica processi}:  controllare che i processi rispettino il loro ciclo di vita;
	\item \textbf{Verifica qualità}:  bisogna verificare che le metriche degli obliettivi di qualità siano entro limiti accettabili;
	\item \textbf{Verifica rispetto norme}: bisogna verificare constantemente che le norme prefissate, per il way of working del gruppo, siano sempre rispettate.
\end{itemize}

\textbf{Obiettivi:}
\smallbreak
\begin{itemize}
	\item \textbf{QPR12 Verifica costante}: tutti i prodotti devono essere constantemente sotto verifica e le verifiche devono essere sempre le medesime, per avere consistenza con i risultati e poter apportare miglioramenti di qualità.
\end{itemize}

\textbf{Metriche:}
\smallbreak
\begin{itemize}
	\item \textbf{MPR8 Frequenza di controllo};
\end{itemize}

\subsection{Tabella qualità di processo}
La seguente tabella indica gli obiettivi di qualità che i processi devono possedere.\\
Ogni obiettivo di qualità è indicato con:
\smallbreak
\begin{itemize}
	\item \textbf{Obiettivo}: contrasegnato dal suo codice identificativo;
	\item \textbf{Metrica}: contrasegnato dal suo codice identificativo, indica come sarà garantita la qualità dell'obiettivo;
	\item \textbf{Accettabile}: indica il valore sotto il quale non sarà garantita la qualità;
	\item \textbf{Preferibile}: indica il valore a cui si punta per avere la massima qualità;
\end{itemize}

\begin{table}[!ht]
	\centering
	\begin{tabular}{|p{3.5cm}|p{3.5cm}|c|c|}
		\hline
		\rowcolor{lightgray}
		\textbf{Obiettivo}  & \textbf{Metrica} & \textbf{Accettabile} & \textbf{Preferibile} \\
		\hline
		QPR01 Scadenze & MPR02 Varianza della pianificazione  & 4 giorni  & 0 giorni   \\
		\hline
		QPR02 Budget   & MPR03 Varianza dei costi   &  $\pm$5\%    & 0\%  \\
		\hline
		QPR05 Adempimento requisiti obbligatori  & MPR04 Requisiti obbligatori     & 100\%  & 100\%   \\
		\hline
		QPR06 Adempimento requisiti non obbligatori & MPR05 Requisiti non obbligatori & 75\%   & 0\%     \\
		\hline
		QPR08 Rispetto obiettivi   & MPR07 Obiettivi soddisfatti     & 100\%   & 100\%   \\
		\hline
		QPR09 Rispetto delle norme  & MPR08 Norme rispettate   & 100\%  & 100\%   \\
		\hline
		QPR12 Verifica costante   & MPR10 Frequenza di controllo    & Ad ogni Milestone    & Ad ogni modifica     \\
		\hline
	\end{tabular}
	\caption{Metriche qualità di processo}
\end{table}

\end{document}



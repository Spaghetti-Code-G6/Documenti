\documentclass[../piano_di_qualifica.tex]{subfiles}

\begin{document}

\subsection{Obiettivi di qualità}
In questa sezione saranno presentati gli obiettivi di qualità che si vogliono raggiungere, dei processi implementati e i dei loro prodotti, così da rendere efficace la valutazione della qualità e di selezionare le metriche migliori.

\subsubsection{Obiettivi di qualità di processo}
Si è deciso di usare come riferimento lo standard ISO/IEC 15504 (SPICE) per la qualità di processo e il metodo PDCA (o ciclo di Deming) per il miglioramento continuo della qualità durante tutto l'arco del progetto.
%Per una trattazione esaustiva di tale standard si rimanda all’appendice §A delle Norme di Progetto 2.0.0. 
%Per una trattazione esaustiva di tale modello si rimanda all’appendice §B delle Norme di Progetto 2.0.0.

\begin{itemize}
	\item OQPC01 Monitoraggio risorse: tenere sotto osservazione il consumo delle risorse;
	\item OQPC02 Rispetto pianificazione: rispettare le scadenze prefissate nel Piano di Progetto;
	\item OQPC03 Gestione rischi: tenere sempre sotto controllo i rischi per poterne evitare o mitigare gli effetti;
	\item OQPC04 Miglioramento continuo: seguendo il metodo DPCA, migliorare sempre il processo.
\end{itemize}

\subsubsection{Obiettivi di qualità di prodotto}
Si è deciso di usare come riferimento lo standard ISO/IEC 9126 per la qualità di prodotto. 
%Per una trattazione esaustiva di tale modello si rimanda all’appendice §C delle Norme di Progetto 2.0.0.

\begin{itemize}
	\item OQPD01 Correttezza documenti: tutti gli errori grammaticali e lessicali devono essere corretti;
	\item OQPD02 Leggibilità documenti: chiunque abbia una licenza media deve poter comprendere i documenti (secondo l'indice Gulpease);
	\item OQPD03 Coerenza documenti: tutte le inconsistenze tra i documenti devono essere corrette.
\end{itemize}

\subsection{Metriche di qualità}












%\subsection{Introduzione}
%Durante lo svolgimento di questo progetto, verranno usati i processi, che permetteranno di raggiungere la soddisfazione dei criteri di qualità prefissando un continuo miglioramento. Si è scelto di usare lo \glossario{standard ISO/IEC 12207:1995} come riferimento per una \glossario{best practice} e di fare uso del \glossario{metodo PDCA}. Così facendo, è possibile garantire l'evoluzione dei processi che, attraverso l'esperienza, migliora e assicura un prodotto di qualità. Vengono di seguito esposti i parametri di qualità tollerabili e preferibili sulla base delle metriche scelte nel documento Norme di Progetto. %I parametri riguardano:
%\begin{itemize}
%\item Budget: per evitare differenze eccessive rispetto al costo preventivato si seguiranno le metriche di seguito descritte;
%\item Calendario: assicurare la pianificazione adatta ai compiti da svolgere.
%\end{itemize}

%\subsection{Metriche di budget}
%\label{sub:metr_bud}
%Si cercherà di rispettare a pieno il budget preventivato, tuttavia ci si è posto come valore tollerabile una differenza di ${\pm 5\%}$ il totale stabilito.

%\subsection{Metriche di pianificazione}
%\label{sub: metr_pianif}
%Nella \glossario{pianificazione} descritta nel Piano di Progetto si è tenuto conto di un ampio margine per far fronte ad eventuali ritardi. Per ogni step del progetto vi sarà una pianificazione fatta a priori dove verranno stabiliti micro-processi da svolgere e le tempistiche entro cui svolgerli. Il ritardo aggiuntivo per ogni compito assegnato non deve essere superiore a 4 giorni. 

\end{document}
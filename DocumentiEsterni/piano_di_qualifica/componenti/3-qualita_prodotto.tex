\documentclass[../piano_di_qualifica.tex]{subfiles}

\begin{document}

Per garantire la qualità del prodotto il gruppo ha deciso di fare riferimento allo standard
ISO/IEC 9126, il quale regolamenta le modalità con cui produrre un prodotto di buona qualità. Di questo standard il gruppo sceglie di seguire alcune delle metriche esposte, e sceglie di stabilirne altre da seguire per rendere il prodotto qualitativamente valido; qui di seguito vengono elencate.

\subsection{Qualità dei documenti}
\label{sub:qual_doc}
Tutti i documenti devono essere leggibili e comprensibili, e per renderli tali devono essere corretti dal punto di vista lessicografico, grammaticale e semantico.
Per garantire la leggibilità dei documenti si è deciso di usare l’indice di Gulpease come indicatore di questa caratteristica. La comprensione di essi verrà valutata dai seguenti criteri:
\begin{itemize}
\item QM-PROD-1 INDICE DI GULPEASE;
\item QM-PROD-2 Correttezza ortografica.
\end{itemize}


\begin{table}[!ht]
\centering
\begin{tabular}{|c|c|c|}
		\hline
		\rowcolor{lightgray}
		\textbf{ID metrica} & \textbf{Valore preferibile} & \textbf{Valore accettabile} \\
		\hline 
		QM-PROD-1 (GULP) & \(\ge 80\) & \(\ge 60\) \\
 		\hline
		QM-PROD-2 (CORT) & \(= 0\) & \(= 0\) \\
		\hline
\end{tabular}
\caption{Indici di qualità per le metriche di comprensione del prodotto}
\end{table}


\end{document}
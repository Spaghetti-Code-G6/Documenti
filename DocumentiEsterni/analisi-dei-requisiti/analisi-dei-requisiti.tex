\documentclass{article}

\input{../../risorse/config}
%aggiungere percorsi dai quali il documento prende le immagini
\appendToGraphicspath{../../risorse/img/}

\setTitle{Analisi dei Requisiti}

\setVersione{v0.0.1}

\setResponsabile{
  Paparazzo Giorgia
}

\setRedattori{
  Rizzo Stefano \\ &
  Contro Daniel Eduardo \\ &
  Fichera Jacopo \\ &
  Pagotto Manuel 
}

\setVerificatori{
  XX
}

\setUso{Interno}

\setDestinatari{
  prof. Vardanega Tullio \\ &
  prof. Cardin Riccardo \\ &
  SpaghettiCode
}

\setDescrizione{Questo documento ha lo scopo di descrivere l'analisi dei capitolati d'appalto realizzata dal gruppo al fine di
valutarne la fattibilità}

%\disabilitaElencoFigure{}
%\disabilitaElencoTabelle{}

\setModifiche{
  v0.0.4 & YYY & Analista & 2020-12-X & ZZZ \\
  v0.0.3 & YYY & Analista & 2020-12-X & ZZZ \\
  v0.0.2 & YYY & Analista & 2020-12-X & ZZZ \\
  v0.0.1 & Manuel Pagotto & Analista & 2020-12-12 & Creazione del documento 
}

\begin{document}

\pagenumbering{gobble}

\newif\iffirstpage
\firstpagetrue

\backgroundsetup{
  scale=1,
  opacity=0.2,
  angle=0,
  placement=top,
  contents={%
    \iffirstpage
      \includegraphics[width=\paperwidth]{datascience-og_colori.png}%
      \global\firstpagefalse
    \fi
  }%
  }

\begin{titlepage}% per non stampare il numero della pagina
  
  


  \raggedright % allinea a destra la pagina
  %\rule{1pt}{\textheight}% linea verticale
  \hspace{0.05\textwidth}% spazio tra linea e testo
  % lasciare questa riga per il corretto funziomento di \parbox
  \parbox[b]{0.75\textwidth}{% paragrafo che tiene il testo a destra della riga cambiando la larghezza il testo si muove a destra o a sinistra
  {\hspace{0.07\textwidth}\includegraphics[width=3.5cm,height=3.5cm]{logo-nero.png}}\\[3\baselineskip] % logo
  {\Huge\bfseries SpaghettiCode}\\ [\baselineskip] %titolo
  {\texttt{spaghetti.code.g6@gmail.com}}\\[\baselineskip]\\[4\baselineskip] % 
  {\Large\textsc{\placeholderTitle{}}}\\[4\baselineskip] % nome del documento
  {\begin{tabular}{r|l}
    \hline \\
    % testo in grassetto
    \textbf{Versione}     & \versione{}               \\
    \rule{0pt}{3ex}%  EXTRA vertical height 
    \textbf{Approvazione} & \responsabile{}           \\
    \rule{0pt}{3ex}%  EXTRA vertical height 
    \textbf{Redazione}    & \redattori{}              \\
    \rule{0pt}{3ex}%  EXTRA vertical height 
    \textbf{Verifica}     & \verificatori{}           \\
    \rule{0pt}{3ex}%  EXTRA vertical height 
    \textbf{Uso}          & \uso{}                    \\
    \rule{0pt}{3ex}%  EXTRA vertical height 
    \textbf{Destinato a}  & \destinatari{}            \\
    \rule{0pt}{3ex}%  EXTRA vertical height 
    \ifthenelse{\equal{\uso}{Esterno}}{
                          & Zucchetti S.p.A.       \\
    }{}
  \end{tabular}}\\[4\baselineskip]

  {\bfseries Descrizione}\\
  {\descrizione{}}\\[1\baselineskip]
  }

\end{titlepage}

\newgeometry{textheight=660pt, lmargin=2cm, tmargin=2cm, rmargin=2cm}

% setup di header e footer nelle pagine senza numero
\fancypagestyle{nopage}{%
  \fancyhf{}%
  \fancyhead[R]{\includegraphics[width=1.3cm]{logo-nero.png}}%
  \fancyhead[L]{\emph{SpaghettiCode}\\\placeholderTitle{}}%
}
% setup di header e footer nelle pagine col numero
\fancypagestyle{usual}{%
  \fancyhf{}%
  \fancyhead[R]{\includegraphics[width=1.3cm]{logo-nero.png}}%
  \fancyhead[L]{\emph{SpaghettiCode}\\\placeholderTitle{}}%
  \fancyfoot[R]{\thepage\ di~\pageref{LastPage}}%
}
\setlength{\headheight}{1.8cm}

\newpage
\pagestyle{nopage}

\setcounter{table}{-1}


%REGISTRO DELLE MODIFICHE

\section*{Registro delle modifiche}%
\label{sec:registro_delle_modifiche}

\rowcolors{2}{white!80!lightgray!90}{white}
\renewcommand{\arraystretch}{2} % allarga le righe con dello spazio sotto e sopra
\begin{longtable}[H]{>{\centering\bfseries}m{2cm} >{\centering}m{3.5cm} >{\centering}m{2.5cm} >{\centering}m{3cm} >{\centering\arraybackslash}m{5cm}}
  \rowcolor{lightgray}
  {\textbf{Versione}} & {\textbf{Nominativo}} & {\textbf{Ruolo}} & {\textbf{Data}} & {\textbf{Descrizione}}  \\
  \endfirsthead%
  \rowcolor{lightgray}
  {\textbf{Versione}} & {\textbf{Nominativo}}  & {\textbf{Ruolo}} & {\textbf{Data}} & {\textbf{Descrizione}}  \\
  \endhead%
  \modifiche{}%
\end{longtable}
% section registro_delle_modifiche (end)

\newpage
\thispagestyle{nopage}
\pagenumbering{roman}
\tableofcontents

\elencoFigure{}%

\elencoTabelle{}%

\newpage

\pagestyle{usual}
\pagenumbering{arabic}


\section{Introduzione}%
\label{sec:introduzione}

\subsection{Scopo del documento}%
\label{sub:scopo_del_documento}
Con il presente documento si vuole analizziare in modo dettagliato i requisiti e i casi d'uso 

\subsection{Glossario}%
\label{sub:glossario}
Lorem ipsum dolor sit amet, consectetur adipiscing elit. Aenean eget eros quis lorem iaculis varius. Suspendisse mauris eros, convallis id sem quis, condimentum laoreet sem. Duis scelerisque, dolor ut pretium iaculis, mauris dui gravida nibh, ac porttitor eros purus quis leo. Sed in fermentum sem. Nulla vitae pellentesque quam.

\subsection{Riferimenti}%
\label{sub:riferimenti}

\subsubsection{Normativi}%
\label{subs:normativi}

    \begin{itemize}
    \item \textbf{Norme di progetto}: \textsc{Norme di progetto vX.X.X}
    \item \textbf{Capitolato d’appalto C4}: \url{https://www.math.unipd.it/~tullio/IS-1/2020/Progetto/C4.pdf};
    \end{itemize}

\subsubsection{Informativi}%
\label{subs:informativi}
    \begin{itemize}
    \item \textbf{Studio di Fattibilità}: \textsc{Studio di Fattibilità v0.0.1}
    \end{itemize}

\end{document}

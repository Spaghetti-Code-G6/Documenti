\subsection{Requisiti di qualità}
\label{sub:requisiti_di_qualita}

\rowcolors{2}{white!80!lightgray!90}{white}
\renewcommand{\arraystretch}{2} %
\begin{longtable}[H]{>{\centering\bfseries}m{2cm} >{\centering}m{9cm} >{\centering}m{2.5cm} >{\centering\arraybackslash}m{2.5cm}}
    \caption{Requisiti di qualità}%
    \label{tab:requisiti_di_qualita} \\
    \rowcolor{lightgray}
    {\textbf{Requisito}} & {\textbf{Descrizione}} & {\textbf{Classificazione}} & {\textbf{Fonte}}  \\
    \endfirsthead%
    \rowcolor{lightgray}
    {\textbf{Requisito}} & {\textbf{Descrizione}} & {\textbf{Classificazione}} & {\textbf{Fonte}}  \\
    \endhead%
    \rowcolor{white}
    \multicolumn{4}{c}{\textit{Continua alla pagina successiva}}
    \endfoot%
    \endlastfoot%

    %CORPO TABELLA
    RQO1
        & L'applicativo deve essere accompagnato dalla documentazione minima richiesta per il corso di Ingegneria del software
        & Obbligatorio
        & Capitolato \\

    RQO2
        & L'applicativo dovrà essere accompagnato da un manuale di utilizzo
        & Obbligatorio
        & Capitolato \\

    RQO3
        & L'applicativo dovrà essere accompagnato da un manuale tecnico per inddicare come estendere l'applicazione
        & Obbligatorio
        & Capitolato \\

    RQO4
        & Il manuale di utilizzo dovrà essere fornito in formato pdf ed in lingua italiana
        & Obbligatorio
        & Decisione interna \\

    RQO5
        & Il manuale tecnico dovrà essere fornito in formato pdf ed in lingua italiana
        & Obbligatorio
        & Decisione interna \\

    RQD6
        & Il codice sorgente dovrà essere disponibile su una repository pubblica su Github
        & Desiderabile
        & Capitolato \\

    RQO7
        & L'applicativo dovrà essere sviluppato seguendo quanto stabilito nel documento Norme di Progetto v1.0.0
        & Obbligatorio
        & Decisione interna
        
\end{longtable}

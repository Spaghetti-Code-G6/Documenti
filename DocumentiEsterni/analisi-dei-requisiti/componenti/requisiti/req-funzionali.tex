\subsection{Requisiti di funzionalità}
\label{sub:requisiti_di_funzionalita}

\rowcolors{2}{white!80!lightgray!90}{white}
\renewcommand{\arraystretch}{2} %
\begin{longtable}[H]{>{\centering\bfseries}m{2cm} >{\centering}m{9cm} >{\centering}m{2.5cm} >{\centering\arraybackslash}m{2.5cm}}
    \caption{Requisiti funzionali}%
    \label{tab:requisiti_funzionali} \\
    \rowcolor{lightgray}
    {\textbf{Requisito}} & {\textbf{Descrizione}} & {\textbf{Classificazione}} & {\textbf{Fonte}}  \\
    \endfirsthead%
    \rowcolor{lightgray}
    {\textbf{Requisito}} & {\textbf{Descrizione}} & {\textbf{Classificazione}} & {\textbf{Fonte}}  \\
    \endhead%
    \rowcolor{white}
    \multicolumn{4}{c}{\textit{Continua alla pagina successiva}}
    \endfoot%
    \endlastfoot%

    %CORPO TABELLA
    RFO1
    & L'utente deve poter inserire dati nel sistema
    & Obbligatorio
    & Capitolato \hyperref[sub:uc1]{UC1} \\

    RFO2
    & L'utente deve poter effettuare l'inserimento dati da file csv
    & Obbligatorio
    & Capitolato  \hyperref[ssub:uc1.1]{UC1.1} \\

    RFO3
    & L'utente deve poter effettuare l'inserimento dati da database
    & Obbligatorio
    & Capitolato \hyperref[ssub:uc1.3]{UC1.3}\\

    RFO4
    &   L'utente deve poter aprire un collegamento con un server di dati per
        accedere ad uno dei suoi database
    & Obbligatorio
    & Interno  \hyperref[par:uc1.3.1]{UC1.3.1}\\

    RFO5
    &   L'utente deve poter immettere l'indirizzo del server
    & Obbligatorio
    & Interno  \hyperref[spar:uc1.3.1.1]{UC1.3.1.1}\\

    RFO6
    &   L'utente deve poter immettere il nome di accesso al server
    & Obbligatorio
    & Interno  \hyperref[spar:uc1.3.1.2]{UC1.3.1.2}\\

    RFO7
    &   L'utente deve poter immettere la password di accesso al server
    & Obbligatorio
    & Interno  \hyperref[spar:uc1.3.1.3]{UC1.3.1.3}\\

    RFO8
    &   L'utente deve poter importare i dati nel sistema mediante la ricerca
        su un database tra quelli disponibili
    & Obbligatorio
    & Interno \hyperref[par:uc1.3.2]{UC1.3.2}\\

    RFO9
    &   L'utente deve poter inserire il metadato relativo alla categoria
        del dato delle dimensioni del dataset.
    & Obbligatorio
    & Interno \hyperref[ssub:uc1.4]{UC1.4}\\

    RFO10
    & L'utente deve poter creare un grafico di sua scelta.
    & Obbligatorio
    & Capitolato \hyperref[ssub:uc2.1]{UC2.1}\\

    RFO11
    & L'utente deve poter seleziona l'opzione di costruzione di un grafico di tipo Scatter plot matrix
    & Obbligatorio
    & Capitolato   \hyperref[ssub:uc2.2]{UC2.2}\\

    RFO12
    & L'utente deve poter seleziona l'opzione di costruzione di un grafico di tipo Force Field
    & Obbligatorio
    & Capitolato  \hyperref[ssub:uc2.3]{UC2.3}\\

    RFD13
    & L'utente deve poter seleziona l'opzione di costruzione di un grafico di tipo Heat Map
    & Obbligatorio
    & Interno  \hyperref[ssub:uc2.4]{UC2.4}\\

    RFO14
    & L'utente deve poter seleziona l'opzione di costruzione di un grafico di tipo PLMA
    & Obbligatorio
    & Captiolato   \hyperref[ssub:uc2.5]{UC2.5}\\

    RFO15
    & L'utente deve poter seleziona l'opzione di costruzione di un grafico di tipo Distance Map
    & Obbligatorio
    & Captiolato  \hyperref[ssub:UC2.6]{UC2.6}\\

    RFO16
    & L'utente deve poter modificare le proprietà dei metadati associati al dataset
    & Obbligatorio
    & Interno  \hyperref[ssub:uc3.1]{UC3.1} \\

    RFO17
    & L'utente deve poter modificare il tipo dei metadati associati al dataset
    & Obbligatorio
    & Interno\hyperref[sub:uc3.2]{UC3.2} \\

    RFO18
    & L'utente deve poter modificare il metadato relativo alla visibilità di una dimensione del dataset
    & Obbligatorio
    & Interno  \hyperref[sub:uc3.3]{UC3.3} \\

    RFO19
    & L'utente deve poter annullare le modifiche fatte ai metadati
    & Desiderabile
    & Interno  \hyperref[ssub:uc3.4]{UC3.4} \\

    RFD20
    & L'utente deve poter modificare ed eventualmete annullare le modifiche effettuate su un grafico
    & Desiderabile
    & Capitolato \hyperref[ssub:uc4]{UC4} \\

    RFD21
    & L'utente deve poter modificare le proprietà del grafico
    & Desiderabile
    & Capitolato \hyperref[ssub:uc4.1]{UC4.1} \\

    RFD22
    & L'utente deve poter modificare le proprietà della visualizzazione scatter plot matrix
    & Desiderabile
    & Capitolato \hyperref[ssub:uc4.2]{UC4.2} \\

    RFD23
    & L'utente deve poter modificare il numero di dimensioni di rappresentabili dalla matrice di scatterplot
    & Desiderabile
    & Capitolato \hyperref[ssub:uc4.2.1]{UC4.2.1} \\

    RFD24
    & l'utente deve poter aggiungere una ulteriore dimensione del dato alla matrice di scatterplot mediante colore
    & Desiderabile
    & Verbale \hyperref[ssub:uc4.2.2]{UC4.2.2} \\

    RFD25
    & l'utente deve poter aggiungere una ulteriore dimensione del dato alla matrice di scatterplot mediante brillanza
    & Desiderabile
    & Verbale \hyperref[ssub:uc4.2.3]{UC4.2.3} \\

    RFD26
    & l'utente deve poter selezionare un punto in uno scatter plot della matrice e visualizzarlo negli altri scatter plot.
    & Desiderabile
    & Interno \hyperref[ssub:uc4.2.4]{UC4.2.4} \\

    RFD27
    & l'utente deve poter selezionare un insieme di punti in uno scatter plot della matrice e visualizzarlo negli altri scatter plot.
    & Desiderabile
    & Interno \hyperref[ssub:uc4.2.5]{UC4.2.5} \\

    RFD28
    & l'utente deve poter selezionare un punto rappresentante un dato in uno scatter plot della matrice e visualizzarlo negli altri scatter plot.
    & Desiderabile
    & Interno \hyperref[ssub:uc4.2.3]{UC4.2.3} \\

    RFD29
    & l'utente deve poter eseguire modifiche sul modo di calcolare la matrice delle distanze
    & Desiderabile
    & Verbale \hyperref[ssub:uc4.3]{UC4.3} \\

    RFD30
    & l'utente deve poter scegliere l'algoritmo di calcolo delle distanze tra quelli implementati in HD Viz
    & Desiderabile
    & Interno \hyperref[ssub:uc4.3.1]{UC4.3.1} \\

    RFD31
    & l'utente deve poter scegliere se normalizzare, standardizzare o non eseguire alcun tipo di preprocessing sui dati
    & Desiderabile
    & Verbale \hyperref[ssub:uc4.3.2]{UC4.3.2} \\

    RFD32
    & l'utente deve poter modificare l'influenza di una dimensione
    & Desiderabile
    & Interno \hyperref[ssub:uc4.3.3]{UC4.3.3} \\

    RFD33
    & l'utente deve poter scegliere quali dimensioni del dataset considerare per il calcolo delle distanze
    & Desiderabile
    & Interno \hyperref[ssub:uc4.3.2]{UC4.3.4} \\

    RFD34
    & L'utente deve poter modificare le proprietà della visualizzazione Force Field
    & Desiderabile
    & Capitolato \hyperref[ssub:uc4.4]{UC4.4} \\

    RFD35
    & L'utente deve poter trascinare i nodi visualizzati dalla visualizzazione force field
    & Desiderabile
    & Capitolato \hyperref[ssub:uc4.4.1]{UC4.4.1} \\

    RFD36
    & L'utente deve poter eliminare gli archi che collegano i nodi collegati da forze con valori al di fuori di una certa soglia nella visualizzazione force field
    & Desiderabile
    & Verbale \hyperref[ssub:uc4.4.2]{UC4.4.2} \\

    RFD37
    & L'utente deve poter eliminare gli archi tra nodi collegati tra di loro con forza inferiori ad una certa soglia
    & Desiderabile
    & Verbale \hyperref[ssub:uc4.4.3]{UC4.4.3} \\

    RFD38
    & L'utente deve poter eliminare gli archi tra nodi collegati tra di loro con forza superiori ad una certa soglia
    & Desiderabile
    & Verbale \hyperref[ssub:uc4.4.4]{UC4.4.4} \\

    RFD39
    & L'utente deve poter scalare la forza di attrattività
    & Desiderabile
    & Interno \hyperref[ssub:uc4.4.5]{UC4.4.5} \\

    RFD40
    & L'utente deve poter modificare le proprietà della visualizzazione distance map
    & Desiderabile
    & Capitolato \hyperref[ssub:uc4.5]{UC4.5} \\

    RFD41
    & L'utente deve poter modificare il gradiente di colori tra quelli offerti da HD Viz
    & Desiderabile
    & Interno \hyperref[ssub:uc4.5.1]{UC4.5.1} \\

    RFO42
    & L'utente deve poter ordinare la distance map
    & Obbligatorio
    & Capitolato \hyperref[ssub:uc4.5.2]{UC4.5.2} \\

    RFO43
    & L'utente deve poter ordinare la distance map mediante clustering gerarchico
    & Obbligatorio
    & Capitolato \hyperref[ssub:uc4.5.3]{UC4.5.3} \\

    RFO44
    & L'utente deve poter associare un dendrogramma alla distance map
    & Obbligatorio
    & Capitolato \hyperref[ssub:uc4.5.2]{UC4.5.2} \\

    RFD45
    & L'utente deve poter ripristinare l'ordinamento originario
    & Desiderabile
    & Interno \hyperref[ssub:uc4.5.4]{UC4.5.4} \\

    RFF46
    & L'utente deve poter ordinare le dimensioni rappresentate nella distance map dei dati per valore
    & Facoltativo
    & Interno \hyperref[ssub:uc4.5.5]{UC4.5.5} \\

    RFD47
    & L'utente deve poter modificare le etichette associate alla distance Map
    & Desiderabile
    & Interno \hyperref[ssub:uc4.5.6]{UC4.5.6} \\

    RFD48
    & L'utente deve poter le proprietà della visualizzazione Proiezione Lineare Multi Asse
    & Desiderabile
    & Interno  \hyperref[ssub:uc4.6.1]{UC4.6} \\

    RFD49
    & L'utente deve poter aggiungere una dimensione dalla visualizzazione Proiezione Lineare Multi Asse
    & Desiderabile
    & Interno  \hyperref[par:uc4.6.1]{UC4.6.1} \\

    RFD50
    & L'utente deve poter rimuovere una dimensione dalla visualizzazione Proiezione Lineare Multi Asse
    & Desiderabile
    & Interno  \hyperref[par:uc4.6.2]{UC4.6.2} \\

    RFD51
    & L'utente deve poter spostare gli asi per visualizzare diverse proiezioni dello stesso dataset nella visualizzazione Proiezione Lineare Multi Asse
    & Desiderabile
    & Interno  \hyperref[par:uc4.6.3]{UC4.6.3} \\

    RFD52
    & L'utente deve poter modificare la visualizzazione dell'Heat Map
    & Desiderabile
    & Interno  \hyperref[ssub:uc4.7]{UC4.7} \\

    RFD53
    & L'utente deve poter modificare la visualizzazione dell'Heat Map
    & Desiderabile
    & Interno  \hyperref[ssub:uc4.7]{UC4.7} \\

    RFD54
    & L'utente deve poter modificare il gradiente dei colori nell'Heat Map scegliendo tra quelli proposti
    & Desiderabile
    & Interno  \hyperref[par:uc4.7.1]{UC4.7.1} \\

    RFD55
    & L'utente deve poter modificare le proprprietà dell'asse nell'Heat Map
    & Desiderabile
    & Interno  \hyperref[par:uc4.7.2]{UC4.7.2} \\

    RFD56
    & L'utente deve poter modificare la categoria di etichette assocciata all'asse nell'Heat Map
    & Desiderabile
    & Interno  \hyperref[spar:uc4.7.2.1]{UC4.7.2.1} \\

    RFD57
    & L'utente deve poter ordinare gli elementi dell'asse nell'Heat Map con ordinamento
    & Desiderabile
    & Interno  \hyperref[spar:uc4.7.2.2]{UC4.7.2.2} \\

    RFD58
    & L'utente deve poter ordinare gli elementi dell'asse nell'Heat Map con ordinamento clustering gerarchico
    & Desiderabile
    & Interno  \hyperref[spar:uc4.7.2.3]{UC4.7.2.3} \\

    RFD59
    & L'utente deve poter modificare l'ordinamento degli elementi ritornando all'ordine orginale del dataset
    & Desiderabile
    & Interno  \hyperref[spar:uc4.7.2.4]{UC4.7.2.4} \\

    RFD60
    & L'utente deve poter modificare le proprietà delle righe nell'Heat Map
    & Desiderabile
    & Interno  \hyperref[par:uc4.7.3]{UC4.7.3} \\

    RFD61
    & L'utente deve poter modificare le etichette associate alle righe nell'Heat Map
    & Desiderabile
    & Interno  \hyperref[spar:uc4.7.3.1]{UC4.7.3.1} \\

    RFD62
    & L'utente deve poter modificare l'ordinamento delle righe nella visualizzazione Heat Map
    & Desiderabile
    & Interno  \hyperref[spar:uc4.7.3.2]{UC4.7.3.2} \\

    RFD63
    & L'utente deve poter ordinare le righe nell'Heat Map attraverso l'ordinamento clustering gerarchico
    & Desiderabile
    & Interno  \hyperref[spar:uc4.7.3.3]{UC4.7.3.3} \\

    RFD64
    & L'utente deve poter ordinare le righe nell'Heat Map nell'ordine originario del dataset
    & Desiderabile
    & Interno  \hyperref[spar:uc4.7.3.4]{UC4.7.3.4} \\

    RFD65
    & L'utente deve poter modificare l'ordinamento delle righe nell'Heat Map secondo dei valori di una dimensione selezionata
    & Desiderabile
    & Interno  \hyperref[spar:uc4.7.3.5]{UC4.7.3.5} \\

    RFD66
    & L'utente deve poter modificare le proprietà delle colonne nell'Heat Map
    & Desiderabile
    & Interno  \hyperref[par:uc4.7.4]{UC4.7.4} \\

    RFD67
    & L'utente deve poter modificare le etichette associate alle colonne nell'Heat Map
    & Desiderabile
    & Interno  \hyperref[spar:uc4.7.4.1]{UC4.7.4.1} \\


    RFD68
    & L'utente deve poter modificare l'ordinamento delle colonne nella visualizzazione Heat Map
    & Desiderabile
    & Interno  \hyperref[spar:uc4.7.4.2]{UC4.7.4.2} \\

    RFD69
    & L'utente deve poter ordinare le colonne nell'Heat Map attraverso l'ordinamento clustering gerarchico
    & Desiderabile
    & Interno  \hyperref[spar:uc4.7.4.3]{UC4.7.4.3} \\

    RFD70
    & L'utente deve poter ordinare le colonne nell'Heat Map nell'ordine originario del dataset
    & Desiderabile
    & Interno  \hyperref[spar:uc4.7.4.4]{UC4.7.4.4} \\

    RFD71
    & L'utente deve poter aggiungere o rimuovere dimensioni alla visulizzazione tramite Heat Map
    & Desiderabile
    & Interno  \hyperref[par:uc4.7.5]{UC4.7.5} \\

    RFD72
    & L'utente deve poter scartare le modifiche fatte nella selezione corrente
    & Desiderabile
    & Interno  \hyperref[ssub:uc4.8]{UC4.8} \\

\end{longtable}

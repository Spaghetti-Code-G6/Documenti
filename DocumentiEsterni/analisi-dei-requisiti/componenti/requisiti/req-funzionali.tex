\subsection{Requisiti di funzionalità}
\label{sub:requisiti_di_funzionalita}

\rowcolors{2}{white!80!lightgray!90}{white}
\renewcommand{\arraystretch}{2} %
\begin{longtable}[H]{>{\raggedright\bfseries}m{20mm} >{\raggedright}m{90mm} >{\raggedright}m{28mm} >{\raggedright\arraybackslash}m{30mm}}
    \caption{Requisiti funzionali}%
    \label{tab:requisiti_funzionali} \\
    \rowcolor{lightgray}
    \multicolumn{1} {>{\centering\bfseries}m{20mm}} {\textbf{Requisito}}
    & \multicolumn{1} {>{\centering}m{90mm}} {\textbf{Descrizione}}
    & \multicolumn{1} {>{\centering}m{28mm}} {\textbf{Classificazione}}
    & \multicolumn{1} {>{\centering\arraybackslash}m{30mm}} {\textbf{Fonte}} \\

    \endfirsthead%
    \rowcolor{lightgray}
    \multicolumn{1} {>{\centering\bfseries}m{20mm}} {\textbf{Requisito}}
    & \multicolumn{1} {>{\centering}m{90mm}} {\textbf{Descrizione}}
    & \multicolumn{1} {>{\centering}m{28mm}} {\textbf{Classificazione}}
    & \multicolumn{1} {>{\centering\arraybackslash}m{30mm}} {\textbf{Fonte}} \\
    \endhead%
    \rowcolor{lightgray!40}
    \multicolumn{4}{c}{\textit{Continua alla pagina successiva}} \\
    \endfoot%
    \endlastfoot%

    %CORPO TABELLA
    RFO1
    & L'utente deve poter inserire dati nel sistema
    & Obbligatorio
    & \makecell{
        Capitolato \\
        \hyperref[sub:uc1]{UC1}} \\

    RFO1.1
    & L'utente deve poter effettuare l'inserimento dei dati da file CSV
    & Obbligatorio
    & \makecell{
        Capitolato  \\
        \hyperref[ssub:uc1.1]{UC1.1}} \\

    RFO1.2
    & L'utente deve poter effettuare l'inserimento dei dati da database esterno
    & Obbligatorio
    & \makecell{
        Capitolato \\
        \hyperref[ssub:uc1.3]{UC1.3}} \\

    RFO1.3
    &   L'utente deve poter aprire un collegamento con un server per poter
        accedere ad uno dei suoi database
    & Obbligatorio
    & \makecell{ Interno \\  \hyperref[par:uc1.3.1]{UC1.3.1}}\\

    RFO1.4
    &   L'utente deve poter immettere l'indirizzo del server
    & Obbligatorio
    & \makecell{ Interno \\  \hyperref[spar:uc1.3.1.1]{UC1.3.1.1}}\\

    RFO1.5
    &   L'utente deve poter immettere il nome di accesso al server
    & Obbligatorio
    & \makecell{ Interno \\  \hyperref[spar:uc1.3.1.2]{UC1.3.1.2}}\\

    RFO1.6
    &   L'utente deve poter immettere la password di accesso al server
    & Obbligatorio
    & \makecell{ Interno \\  \hyperref[spar:uc1.3.1.3]{UC1.3.1.3}}\\

    RFO1.7
    &   L'utente deve poter importare i dati nel sistema mediante la ricerca
        su un database tra quelli disponibili
    & Obbligatorio
    & \makecell{ Interno \\ \hyperref[par:uc1.3.2]{UC1.3.2}}\\

    RFO1.8
    &   L'utente deve poter inserire il metadato relativo alla categoria
        del dato delle dimensioni del dataset.
    & Obbligatorio
    & \makecell{ Interno \\ \hyperref[ssub:uc1.4]{UC1.4}}\\

    RFO2
    & L'utente deve poter creare un grafico di sua scelta.
    & Obbligatorio
    & \makecell{ Capitolato \\ \hyperref[ssub:uc2.1]{UC2.1}}\\

    RFO2.1
    & L'utente deve poter seleziona l'opzione di costruzione di un grafico di tipo Scatter Plot
    Matrix
    & Obbligatorio
    & \makecell{ Capitolato \\   \hyperref[ssub:uc2.2]{UC2.2}}\\

    RFO2.2
    & L'utente deve poter seleziona l'opzione di costruzione di un grafico di tipo Force Field
    & Obbligatorio
    & \makecell{ Capitolato \\  \hyperref[ssub:uc2.3]{UC2.3}}\\

    RFD2.3
    & L'utente deve poter seleziona l'opzione di costruzione di un grafico di tipo Heat Map
    & Obbligatorio
    & \makecell{ Interno \\  \hyperref[ssub:uc2.4]{UC2.4}}\\

    RFO2.4
    & L'utente deve poter seleziona l'opzione di costruzione di un grafico di tipo PLMA
    & Obbligatorio
    & \makecell{Capitolato \\ \hyperref[ssub:uc2.5]{UC2.5}}\\

    RFO2.5
    & L'utente deve poter seleziona l'opzione di costruzione di un grafico di tipo Distance Map
    & Obbligatorio
    & \makecell{Capitolato \\ \hyperref[ssub:UC2.6]{UC2.6}}\\

    RFO3
    & L'utente deve poter modificare le proprietà dei metadati associati al dataset
    & Obbligatorio
    & \makecell{ Interno \\  \hyperref[ssub:uc3.1]{UC3.1} }\\

    RFO3.1
    & L'utente deve poter modificare il tipo dei metadati associati al dataset
    & Obbligatorio
    & \makecell{ Interno \\\hyperref[ssub:uc3.2]{UC3.2} }\\

    RFO3.2
    & L'utente deve poter modificare il metadato relativo alla visibilità di una dimensione del dataset
    & Obbligatorio
    & \makecell{ Interno \\  \hyperref[ssub:uc3.3]{UC3.3} }\\

    RFO3.3
    & L'utente deve poter annullare le modifiche fatte ai metadati
    & Desiderabile
    & \makecell{ Interno \\  \hyperref[ssub:uc3.4]{UC3.4} }\\

    RFD4
    & L'utente deve poter modificare ed eventualmete annullare le modifiche effettuate su un grafico
    & Desiderabile
    & \makecell{ Capitolato \\ \hyperref[sub:uc4]{UC4} }\\

    RFD4.1
    & L'utente deve poter modificare le proprietà del grafico
    & Desiderabile
    & \makecell{ Capitolato \\ \hyperref[ssub:uc4.1]{UC4.1} }\\

    RFD4.2
    & L'utente deve poter modificare le proprietà della visualizzazione scatter plot matrix
    & Desiderabile
    & \makecell{ Capitolato \\ \hyperref[ssub:uc4.2]{UC4.2} }\\

    RFD4.3
    & L'utente deve poter modificare il numero di dimensioni di rappresentabili dalla matrice di scatterplot
    & Desiderabile
    & \makecell{ Capitolato \\ \hyperref[par:uc4.2.1]{UC4.2.1} }\\

    RFD4.4
    & l'utente deve poter aggiungere una ulteriore dimensione del dato alla matrice di scatterplot mediante colore
    & Desiderabile
    & \makecell{ Verbale \\ \hyperref[par:uc4.2.2]{UC4.2.2} }\\

    RFD4.5
    & l'utente deve poter aggiungere una ulteriore dimensione del dato alla matrice di scatterplot mediante brillanza
    & Desiderabile
    & \makecell{ Verbale \\ \hyperref[par:uc4.2.3]{UC4.2.3} }\\

    RFD4.6
    & l'utente deve poter selezionare un punto in uno scatter plot della matrice e visualizzarlo negli altri scatter plot.
    & Desiderabile
    & \makecell{ Interno \\ \hyperref[par:uc4.2.4]{UC4.2.4} }\\

    RFD4.7
    & l'utente deve poter selezionare un insieme di punti in uno scatter plot della matrice e visualizzarlo negli altri scatter plot.
    & Desiderabile
    & \makecell{ Interno \\ \hyperref[par:uc4.2.5]{UC4.2.5} }\\

    RFD4.8
    & l'utente deve poter selezionare un punto rappresentante un dato in uno scatter plot della matrice e visualizzarlo negli altri scatter plot.
    & Desiderabile
    & \makecell{ Interno \\ \hyperref[par:uc4.2.3]{UC4.2.3} }\\

    RFD4.9
    & l'utente deve poter eseguire modifiche sul modo di calcolare la matrice delle distanze
    & Desiderabile
    & \makecell{ Verbale \\ \hyperref[ssub:uc4.3]{UC4.3} }\\

    RFD4.10
    & l'utente deve poter scegliere l'algoritmo di calcolo delle distanze tra quelli implementati in HD Viz
    & Desiderabile
    & \makecell{ Interno \\ \hyperref[par:uc4.3.1]{UC4.3.1} }\\

    RFD4.11
    & l'utente deve poter scegliere se normalizzare, standardizzare o non eseguire alcun tipo di preprocessing sui dati
    & Desiderabile
    & \makecell{ Verbale \\ \hyperref[par:uc4.3.2]{UC4.3.2} }\\

    RFD4.12
    & l'utente deve poter modificare l'influenza di una dimensione
    & Desiderabile
    & \makecell{ Interno \\ \hyperref[par:uc4.3.3]{UC4.3.3} }\\

    RFD4.13
    & l'utente deve poter scegliere quali dimensioni del dataset considerare per il calcolo delle distanze
    & Desiderabile
    & \makecell{ Interno \\ \hyperref[par:uc4.3.2]{UC4.3.4} }\\

    RFD4.14
    & L'utente deve poter modificare le proprietà della visualizzazione Force Field
    & Desiderabile
    & \makecell{ Capitolato \\ \hyperref[ssub:uc4.4]{UC4.4} }\\

    RFD4.15
    & L'utente deve poter trascinare i nodi visualizzati dalla visualizzazione force field
    & Desiderabile
    & \makecell{ Capitolato \\ \hyperref[par:uc4.4.1]{UC4.4.1} }\\

    RFD4.16
    & L'utente deve poter eliminare gli archi che collegano i nodi collegati da forze con valori al di fuori di una certa soglia nella visualizzazione force field
    & Desiderabile
    & \makecell{ Verbale \\ \hyperref[par:uc4.4.2]{UC4.4.2} }\\

    RFD4.17
    & L'utente deve poter eliminare gli archi tra nodi collegati tra di loro con forza inferiori ad una certa soglia
    & Desiderabile
    & \makecell{ Verbale \\ \hyperref[par:uc4.4.3]{UC4.4.3} }\\

    RFD4.18
    & L'utente deve poter eliminare gli archi tra nodi collegati tra di loro con forza superiori ad una certa soglia
    & Desiderabile
    & \makecell{ Verbale \\ \hyperref[par:uc4.4.4]{UC4.4.4} }\\

    RFD4.19
    & L'utente deve poter scalare la forza di attrattività
    & Desiderabile
    & \makecell{ Interno \\ \hyperref[par:uc4.4.5]{UC4.4.5} }\\

    % NICE
    RFD4.20
    & L'utente deve poter modificare le proprietà della visualizzazione distance map
    & Desiderabile
    & \makecell{ Capitolato \\ \hyperref[ssub:uc4.5]{UC4.5} }\\

    RFD4.21
    & L'utente deve poter modificare il gradiente di colori tra quelli offerti da HD Viz
    & Desiderabile
    & \makecell{ Interno \\ \hyperref[par:uc4.5.1]{UC4.5.1} }\\

    RFO4.22
    & L'utente deve poter ordinare la distance map
    & Obbligatorio
    & \makecell{ Capitolato \\ \hyperref[par:uc4.5.2]{UC4.5.2} }\\

    RFO4.23
    & L'utente deve poter ordinare la distance map mediante clustering gerarchico
    & Obbligatorio
    & \makecell{ Capitolato \\ \hyperref[par:uc4.5.3]{UC4.5.3} }\\

    RFO4.24
    & L'utente deve poter associare un dendrogramma alla distance map
    & Obbligatorio
    & \makecell{ Capitolato \\ \hyperref[par:uc4.5.2]{UC4.5.2} }\\

    RFD4.25
    & L'utente deve poter ripristinare l'ordinamento originario
    & Desiderabile
    & \makecell{ Interno \\ \hyperref[par:uc4.5.4]{UC4.5.4} }\\

    RFF4.26
    & L'utente deve poter ordinare le dimensioni rappresentate nella distance map dei dati per valore
    & Facoltativo
    & \makecell{ Interno \\ \hyperref[par:uc4.5.5]{UC4.5.5} }\\

    RFD4.27
    & L'utente deve poter modificare le etichette associate alla distance Map
    & Desiderabile
    & \makecell{ Interno \\ \hyperref[par:uc4.5.6]{UC4.5.6} }\\

    RFD4.28
    & L'utente deve poter le proprietà della visualizzazione Proiezione Lineare Multi Asse
    & Desiderabile
    & \makecell{ Interno \\  \hyperref[par:uc4.6.1]{UC4.6} }\\

    RFD4.29
    & L'utente deve poter aggiungere una dimensione dalla visualizzazione Proiezione Lineare Multi Asse
    & Desiderabile
    & \makecell{ Interno \\  \hyperref[par:uc4.6.1]{UC4.6.1} }\\

    RFD4.30
    & L'utente deve poter rimuovere una dimensione dalla visualizzazione Proiezione Lineare Multi Asse
    & Desiderabile
    & \makecell{ Interno \\  \hyperref[par:uc4.6.2]{UC4.6.2} }\\

    RFD4.31
    & L'utente deve poter spostare gli asi per visualizzare diverse proiezioni dello stesso dataset nella visualizzazione Proiezione Lineare Multi Asse
    & Desiderabile
    & \makecell{ Interno \\  \hyperref[par:uc4.6.3]{UC4.6.3} }\\

    RFD4.32
    & L'utente deve poter modificare la visualizzazione dell'Heat Map
    & Desiderabile
    & \makecell{ Interno \\  \hyperref[ssub:uc4.7]{UC4.7} }\\

    RFD4.33
    & L'utente deve poter modificare il gradiente dei colori nell'Heat Map scegliendo tra quelli proposti
    & Desiderabile
    & \makecell{ Interno \\  \hyperref[par:uc4.7.1]{UC4.7.1} }\\

    RFD4.34
    & L'utente deve poter modificare le etichette assocciate agli assi nell'Heat Map
    & Desiderabile
    & \makecell{ Interno \\  \hyperref[spar:uc4.7.2]{UC4.7.2} }\\

    RFD4.35
    & L'utente deve poter modificare le etichette associate alle righe nell'Heat Map
    & Desiderabile
    & \makecell{ Interno \\  \hyperref[spar:uc4.7.3]{UC4.7.3} }\\

    RFD4.36
    & L'utente deve poter modificare le etichette associate alle colonne nell'Heat Map
    & Desiderabile
    & \makecell{ Interno \\  \hyperref[spar:uc4.7.4]{UC4.7.4} }\\

    RFD4.37
    & L'utente deve poter ordinare gli elementi dell'Heat
    & Desiderabile
    & \makecell{ Interno \\  \hyperref[spar:uc4.7.5]{UC4.7.5} }\\

    RFD4.38
    & L'utente deve poter ordinare gli elementi dell'Heat Map mediante clustering gerarchico
    & Desiderabile
    & \makecell{ Interno \\  \hyperref[spar:uc4.7.6]{UC4.7.6} }\\

    RFD4.39
    & L'utente deve poter modificare l'ordinamento degli elementi ritornando all'ordine oroginale del dataset
    & Desiderabile
    & \makecell{ Interno \\  \hyperref[spar:uc4.7.7]{UC4.7.7} }\\

    RFD4.40
    & L'utente deve poter scartare le modifiche fatte nella selezione corrente
    & Desiderabile
    & \makecell{ Interno \\  \hyperref[ssub:uc4.8]{UC4.8} }\\

\end{longtable}

\subsection{Requisiti funzionali}
\label{sub:requisiti_funzionali}

\rowcolors{2}{white}{white!80!lightgray!90}
\renewcommand{\arraystretch}{2} %
\begin{longtable}[H]{|>{\raggedright\arraybackslash}p{20mm} | p{90mm} | p{22mm} | p{30mm} |}
    \caption{Requisiti funzionali}%
    \label{tab:requisiti_funzionali} \\
    \hline
    \rowcolor{lightgray}
    \multicolumn{1}{| >{\centering\bfseries}m{20mm} |}{\textbf{Requisito}} 
    & \multicolumn{1}{ >{\centering}m{90mm} |}{\textbf{Descrizione}} 
    & \multicolumn{1}{ >{\centering}m{22mm} |}{\textbf{Rilevanza}} 
    & \multicolumn{1}{ >{\centering\arraybackslash}m{30mm} |}{\textbf{Fonte}} \\

    \endfirsthead%
    \hline
    \rowcolor{lightgray}
    \multicolumn{1}{| >{\centering\bfseries}m{20mm} |}{\textbf{Requisito}} 
    & \multicolumn{1}{ >{\centering}m{90mm} |}{\textbf{Descrizione}} 
    & \multicolumn{1}{ >{\centering}m{22mm} |}{\textbf{Rilevanza}} 
    & \multicolumn{1}{ >{\centering\arraybackslash}m{30mm} |}{\textbf{Fonte}} \\
    \hline
    \endhead%
    \hline
    \rowcolor{lightgray!40}
    \multicolumn{4}{|c|}{\textit{Continua alla pagina successiva}} \\
    \hline
    \endfoot%
    \hline
    \endlastfoot%

    %CORPO TABELLA
    
    % Creazione ambiente
    RFO1
        & L'utente deve poter inserire dati nel sistema 
        & Obbligatorio 
        & \\
        % \hyperref[UC1]{sub:uc1} \\
    RFO1.1 
        & L'utente deve poter inserire dati nel sistema a partire da un file .csv 
        & Obbligatorio 
        & \\
        % & \hyperref[UC1.1]{ssub:uc1.1} \\
    RFO1.2 
        & L'utente deve poter inserire dati nel sistema da database esterno 
        & Obbligatorio 
        &  \\
        % & \hyperref[UC1.2]{ssub:uc1.2} \\
    % RFD1 
    %     & L'utente deve poter inserire dati nel sistema da file di progetto 
    %     & Desiderabile 
    %     & \hyperref[UC1.3]{ssub:uc1.3} \\

    RFD2 
        & L'utente deve poter inserire metadati inerenti ai dati precedentemente inseriti 
        & Desiderabile 
        & \\
        % & \hyperref[UC]{ssub:uc.} \\
    RFD3 
        & L'utente deve poter modificare metadati precedentemente inseriti 
        & Desiderabile 
        & \\
        % & \hyperref[UC]{ssub:uc.} \\
    RFO4 
        & L'utente deve poter selezionare la tipologia di grafico da visualizzare 
        & Obbligatorio 
        & \\
        % & \hyperref[UC]{ssub:uc.} \\
    RFO4.1 
        & L'utente deve poter selezionare Scatter Plot Matrix come tipologia di grafico da visualizzare 
        & Obbligatorio 
        & \\
        % & \hyperref[UC]{ssub:uc.} \\
    RFO4.2 
        & L'utente deve poter selezionare Force Field come tipologia di grafico da visualizzare 
        & Obbligatorio 
        & \\
        % & \hyperref[UC]{ssub:uc.} \\
    RFO4.3 
        & L'utente deve poter selezionare Heat Map come tipologia di grafico da visualizzare 
        & Obbligatorio 
        & \\
        % & \hyperref[UC]{ssub:uc.} \\
    RFO4.4 
        & L'utente deve poter selezionare Proiezione Lineare Multi Asse come tipologia di grafico da visualizzare 
        & Obbligatorio 
        & \\
        % & \hyperref[UC]{ssub:uc.} \\
    RFD5 
        & L'utente deve poter modificare il grafico che sta visualizzando 
        & Desiderabile 
        & \\
        % & \hyperref[UC]{ssub:uc.} \\
    
    % RF Scatter Plot to be decided con Zucchetti

    % Force Field
    RFD5.2.1 
        & L'utente deve poter spostare i nodi del grafico Force Field 
        & Desiderabile 
        & \\
        % & \hyperref[UC]{ssub:uc.} \\

    % Forse é necessario specificare quali distanze proponiamo
    RFD5.2.2 
        & L'utente deve poter modificare la tipologia di distanza utilizzata per il calcolo della matrice delle 
        distanze nel grafico Force Field 
        & Desiderabile 
        & \\
        % & \hyperref[UC]{ssub:uc.} \\

    RFD5.2.3 
        & L'utente deve poter 
        & 
        & \\

    % Heat Map
    RFD5.3.1 
        & L'utente deve poter modificare l'intervallo di colori utilizzato nel grafico Heat Map
        & Desiderabile
        & \\
        % & \hyperref[UC]{ssub:uc.} \\

    RFD5.4.1
        & L'utente deve poter aggiungere una delle dimensione dei dati non presenti nel grafico PLMA
        & Desiderabile
        & \\
        % & \hyperref[UC]{ssub:uc.} \\
    
    RFD5.4.2
        & L'utente deve poter rimuovere una delle dimensioni dei dati presenti nel grafico PLMA
        & Desiderabile
        & \\
        % & \hyperref[UC]{ssub:uc.} \\

    RFD5.4.3
        % Specificare quali proprietà in requisiti diversi?
        & L'utente deve poter modificare le proprietà grafiche delle dimensioni
        & Desiderabile
        & \\
        % & \hyperref[UC]{ssub:uc.} \\

    RFD6
        & L'utente deve essere notificato quando si verifica un errore
        & Desiderabile
        & \\
        % & \hyperref[UC]{ssub:uc.} \\

    % Requisiti diversi per scenari alternativi (?)
    RFD6.1
        & L'utente deve essere notificato quando avviene un errore durante 
        l'inserimento dei dati da file
        & Desiderabile
        & \\
        % & \hyperref[UC]{ssub:uc.} \\

    % Requisiti diversi per scenari alternativi (?)
    RFD6.2
        & L'utente deve essere notificato quando avviene un errore durante 
        l'inserimento di dati da database esterno
        & Desiderabile
        & \\
        % & \hyperref[UC]{ssub:uc.} \\

    % Aggiungere altro codice puntato per errori di modifica grafico(?)
    % Eventuale errore scatter plot
    RFD6.3
        & L'utente deve essere notificato quando viene scelta la visualizzazione di un grafico Force Field e nessuna delle dimensioni dei dati caricati é categorica
        & Desiderabile
        & \\
        % & \hyperref[UC]{ssub:uc.} \\

    RFD6.4
        & L'utente deve essere notificato quando viene scelta la visualizzazione di un grafico Heat Map e nessuna delle dimensioni dei dati caricati é numerica
        & Desiderabile
        & \\
        % & \hyperref[UC]{ssub:uc.} \\
    
\end{longtable}
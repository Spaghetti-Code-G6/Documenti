\subsection{Requisiti di funzionalità}
\label{sub:requisiti_di_funzionalita}

\rowcolors{2}{white!80!lightgray!90}{white}
\renewcommand{\arraystretch}{2} %
\begin{longtable}[H]{>{\centering\bfseries}m{2cm} >{\centering}m{9cm} >{\centering}m{2.5cm} >{\centering\arraybackslash}m{2.5cm}}
    \caption{Requisiti funzionali}%
    \label{tab:requisiti_funzionali} \\
    \rowcolor{lightgray}
    {\textbf{Requisito}} & {\textbf{Descrizione}} & {\textbf{Classificazione}} & {\textbf{Fonte}}  \\
    \endfirsthead%
    \rowcolor{lightgray}
    {\textbf{Requisito}} & {\textbf{Descrizione}} & {\textbf{Classificazione}} & {\textbf{Fonte}}  \\
    \endhead%
    \rowcolor{white}
    \multicolumn{4}{c}{\textit{Continua alla pagina successiva}}
    \endfoot%
    \endlastfoot%

    %CORPO TABELLA
    RFO1
    & L'utente deve poter inserire dati nel sistema 
    & Obbligatorio
    & Capitolato \hyperref[sub:uc1]{UC1} \\

    RFO2
    & L'utente deve poter effettuare l'inserimento dati da file csv
    & Obbligatorio
    & Capitolato  \hyperref[ssub:uc1.1]{UC1.1} \\

    RFO3
    & L'utente deve poter effettuare l'inserimento dati da database
    & Obbligatorio
    & Capitolato \hyperref[ssub:uc1.3]{UC1.3}\\

    RFO4
    &   L'utente deve poter aprire un collegamento con un server di dati per 
        accedere ad uno dei suoi database
    & Obbligatorio
    & Interno  \hyperref[par:uc1.3.1]{UC1.3.1}\\

    RFO5
    &   L'utente deve poter immettere l'indirizzo del server
    & Obbligatorio
    & Interno  \hyperref[spar:uc1.3.1.1]{UC1.3.1.1}\\

    RFO6
    &   L'utente deve poter immettere il nome di accesso al server
    & Obbligatorio
    & Interno  \hyperref[spar:uc1.3.1.2]{UC1.3.1.2}\\

    RFO7
    &   L'utente deve poter immettere la password di accesso al server
    & Obbligatorio
    & Interno  \hyperref[spar:uc1.3.1.3]{UC1.3.1.3}\\

    RFO8
    &   L'utente deve poter importare i dati nel sistema mediante la ricerca
        su un database tra quelli disponibili
    & Obbligatorio
    & Interno \hyperref[par:uc1.3.2]{UC1.3.2}\\

    RFO9
    &   L'utente deve poter inserire il metadato relativo alla categoria
        del dato delle dimensioni del dataset.
    & Obbligatorio
    & Interno \hyperref[ssub:uc1.4]{UC1.4}\\

    RFO10
    & L'utente deve poter creare un grafico di sua scelta.
    & Obbligatorio
    & Capitolato \hyperref[ssub:uc2.1]{UC2.1}\\

    RFO11
    & L'utente deve poter seleziona l'opzione di costruzione di un grafico di tipo Scatter plot matrix
    & Obbligatorio
    & Capitolato   \hyperref[ssub:uc2.2]{UC2.2}\\

    RFO12
    & L'utente deve poter seleziona l'opzione di costruzione di un grafico di tipo Force Field
    & Obbligatorio
    & Capitolato  \hyperref[ssub:uc2.3]{UC2.3}\\
    
    RFD13
    & L'utente deve poter seleziona l'opzione di costruzione di un grafico di tipo Heat Map
    & Obbligatorio
    & Interno  \hyperref[ssub:uc2.4]{UC2.4}\\

    RFO14
    & L'utente deve poter seleziona l'opzione di costruzione di un grafico di tipo PLMA
    & Obbligatorio
    & Captiolato   \hyperref[ssub:uc2.5]{UC2.5}\\

    RFO15
    & L'utente deve poter seleziona l'opzione di costruzione di un grafico di tipo Distance Map
    & Obbligatorio
    & Captiolato  \hyperref[ssub:UC2.6]{UC2.6}\\

    RFO16
    & L'utente deve poter modificare le proprietà dei metadati associati al dataset
    & Obbligatorio
    & Interno  \hyperref[ssub:uc3.1]{UC3.1} \\

    RFO17
    & L'utente deve poter modificare il tipo dei metadati associati al dataset
    & Obbligatorio
    & Interno\hyperref[sub:uc3.2]{UC3.2} \\

    RFO18
    & L'utente deve poter modificare il metadato relativo alla visibilità di una dimensione del dataset
    & Obbligatorio
    & Interno  \hyperref[sub:uc3.3]{UC3.3} \\

    RFO19
    & L'utente deve poter annullare le modifiche fatte ai metadati
    & Desiderabile
    & Interno  \hyperref[ssub:uc3.4]{UC3.4} \\

\end{longtable}
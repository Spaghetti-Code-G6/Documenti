\documentclass[../analisi-dei-requisiti.tex]{subfiles}

\begin{document}

\subsection{Scopo del documento}%
\label{subs:scopo_del_documento}
Questo documento ha lo scopo di descrivere i requisiti e i casi d'uso individuati in seguito allo studio del \glossario{capitolato} \emph{HD Viz} proposto da \emph{Zucchetti S.p.A.}. 

\subsection{Scopo del prodotto}%
\label{subs:scopo_del_prodotto}
Il capitolato richiede lo sviluppo di una \glossario{web application} che abbia come scopo la 
traduzione di dati con molte dimensioni in grafici che aiutino l’utente a trarre delle interpretazioni e conclusioni sugli stessi. 
Questi dati dovranno essere inseriti tramite file \glossario{CSV} oppure ottenuti tramite \glossario{query} da un \glossario{database}.
Verrà utilizzata la libreria \glossario{JavaScript} \glossario{D3.js} per creare le visualizzazioni dei dati in modo dinamico ed interattivo.
Il back end verrà sviluppato in JavaScript. 


\subsection{Glossario}
\label{subs:glossario}
Alcuni termini all'interno di questo documento possono risultare ambigui a seconda del contesto in cui sono utilizzati.
La prima occorrenza nel documento di questi termini é segnalata con una 'G' a pedice, e nel documento 
\textsc{Glossario v1.0.0} é esplicitato il loro significato specifico.


\subsection{Riferimenti}
\label{subs:riferimenti}

\subsubsection{Normativi}%
\label{sssec:normativi}

%TODO Sistemare versioni
\begin{itemize}
  \item \textbf{Norme di progetto}: \textsc{Norme di progetto v1.0.0};
  \item \textbf{Capitolato d'appalto C4 - HD Viz}: \url{https://www.math.unipd.it/~tullio/IS-1/2020/Progetto/C4.pdf};
  \item \textbf{Verbale esterno}: \textsc{Verbale Esterno 2020-12-17 v1.0.0}.
\end{itemize}

\subsubsection{Informativi}%
\label{sssec:informativi}
%TODO aggiungere i vari riferimenti usati nel documento
\begin{itemize}
  \item \textbf{Studio di fattibilità}: \textsc{Studio di fattibilità v1.0.0};
  \item \textbf{Capitolato d'appalto C4 - HD Viz}: \url{https://www.math.unipd.it/~tullio/IS-1/2020/Progetto/C4.pdf};
  \item \textbf{Documentazione libreria D3.js}: \url{https://github.com/d3/d3/wiki}
\end{itemize}

\end{document}

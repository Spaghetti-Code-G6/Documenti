\documentclass[../analisi-dei-requisiti.tex]{subfiles}

\begin{document}

\subsection{Scopo del documento}%
\label{subs:scopo_del_documento}
Questo documento ha lo scopo di descrivere i requisiti e i casi d'uso individuati in seguito allo studio del \glossario{capitolato} \emph{HD Viz} proposto da \emph{Zucchetti S.p.A.}. 

\subsection{Scopo del prodotto}%
\label{subs:scopo_del_prodotto}
Il capitolato richiede lo sviluppo di una \glossario{web application} che abbia come scopo la 
traduzione di dati con molte dimensioni in grafici che aiutino l’utente a trarre delle interpretazioni e conclusioni. Questi dati dovranno essere inseriti tramite file \glossario{CSV} oppure ottenuti tramite \glossario{query} da un \glossario{database}.
Verrà utilizzata la librerira \glossario{JavaScript} \glossario{D3.js} per creare le visualizzazioni dei dati in modo dinamico ed interattvo.
Il back end verrà scritto utilizzando \glossario{JavaScript}. 


\subsection{Glossario}
\label{subs:glossario}
Alcuni termini all'interno di questo documento possono risultare ambigui a secondo del contesto in cui sono utlizzati.
Questi termini sono segnalati con un 'G' a pedice del termine ambiguo; nel documento \textsc{Glossario vX.X.X} sono presenti questi termini con il loro significato specifico.


\subsection{Riferimenti}
\label{subs:riferimenti}

\subsubsection{Normativi}%
\label{sssec:normativi}

%TODO Sistemare versioni
\begin{itemize}
  \item \textbf{Norme di progetto}: \textsc{Norme di progetto vX.X.X};
  \item \textbf{Capitolato d'appalto C4 - HD Viz}: \url{https://www.math.unipd.it/~tullio/IS-1/2020/Progetto/C4.pdf};
  \item \textbf{Verbale esterno}: \textsc{Verbale Esterno 2020-12-17 vX.X.X}.
\end{itemize}

\subsubsection{Informativi}%
\label{sssec:informativi}
%TODO aggiungere i vari riferimenti usati nel documento
\begin{itemize}
  \item \textbf{Studio di fattibilità}: \textsc{Studio di fattibilità v1.0.0};
  \item \textbf{Capitolato d'appalto C4 - HD Viz}: \url{https://www.math.unipd.it/~tullio/IS-1/2020/Progetto/C4.pdf};
  \item \textbf{Documentazione libreria D3.js}: \url{https://github.com/d3/d3/wiki}
\end{itemize}

\end{document}

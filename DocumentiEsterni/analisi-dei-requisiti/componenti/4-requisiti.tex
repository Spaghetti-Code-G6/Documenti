\documentclass[../analisi-dei-requisiti.tex]{subfiles}

\begin{document}
I requisiti sono riportati in una tabella i cui campi sono:
\begin{itemize}
    \item\emph{Identificativo}: I requisiti si identificano mediante l'espressione: R[Tipoliga][Vincolo][CodiceProgressivo].
    
    Ogni voce tra parentesi quadre rappresenta un valore relativo alle proprietà di un requisito:
    \begin{itemize}
       
        \item \emph{Tipologia}: La tipologia che può assumere, il cui valore che varia tra: Funzionale, Qualità, Vincoolo.
        
        \item \emph{Vincolo}: Il vincolo di obbligo che si ha su di esso, che varia tra: Obbligatorio, Desiderabile e Facoltativo.
        \item \emph{CodiceProgressivo}: Un numero progressivo per identificarlo univocamente.
    \end{itemize} 

    \item \emph{Descrizione}: Una breve descrizione della funzionalità.
    \item \emph{Classificazione}: Vincolo di obbligo che si ha sulla produzione del requisito.
    \item \emph{Fonti}:   Ogni requisito ha una fonte che è descritta dal metodo di individuamento tra
                    Capitolato, Interno e Verbale e l'eventuale UC che ne descrive la funzionalità.

\end{itemize}

\subsection{Requisiti di funzionalità}
\label{sub:requisiti_di_funzionalita}

\rowcolors{2}{white!80!lightgray!90}{white}
\renewcommand{\arraystretch}{2} %
\begin{longtable}[H]{>{\centering\bfseries}m{2cm} >{\centering}m{9cm} >{\centering}m{2.5cm} >{\centering\arraybackslash}m{2.5cm}}
    \caption{Requisiti funzionali}%
    \label{tab:requisiti_funzionali} \\
    \rowcolor{lightgray}
    {\textbf{Requisito}} & {\textbf{Descrizione}} & {\textbf{Classificazione}} & {\textbf{Fonte}}  \\
    \endfirsthead%
    \rowcolor{lightgray}
    {\textbf{Requisito}} & {\textbf{Descrizione}} & {\textbf{Classificazione}} & {\textbf{Fonte}}  \\
    \endhead%
    \rowcolor{white}
    \multicolumn{4}{c}{\textit{Continua alla pagina successiva}}
    \endfoot%
    \endlastfoot%

    %CORPO TABELLA

\end{longtable}

\newpage

\subsection{Requisiti di qualità}
\label{sub:requisiti_di_qualita}

\rowcolors{2}{white!80!lightgray!90}{white}
\renewcommand{\arraystretch}{2} %
\begin{longtable}[H]{>{\centering\bfseries}m{2cm} >{\centering}m{9cm} >{\centering}m{2.5cm} >{\centering\arraybackslash}m{2.5cm}}
    \caption{Requisiti di qualità}%
    \label{tab:requisiti_di_qualita} \\
    \rowcolor{lightgray}
    {\textbf{Requisito}} & {\textbf{Descrizione}} & {\textbf{Importanza}} & {\textbf{Fonte}}  \\
    \endfirsthead%
    \rowcolor{lightgray}
    {\textbf{Requisito}} & {\textbf{Descrizione}} & {\textbf{Importanza}} & {\textbf{Fonte}}  \\
    \endhead%
    \rowcolor{white}
    \multicolumn{4}{c}{\textit{Continua alla pagina successiva}}
    \endfoot%
    \endlastfoot%

    %CORPO TABELLA
    RQO1
        & L'applicativo deve essere accompagnato dalla documentazione minima richiesta per il corso di Ingegneria del software
        & Obbligatorio
        & Capitolato \\

    RQO2
        & L'applicativo dovrà essere accompagnato da un manuale di utilizzo
        & Obbligatorio
        & Capitolato \\

    RQO3
        & L'applicativo dovrà essere accompagnato da un manuale tecnico per inddicare come estendere l'applicazione
        & Obbligatorio
        & Capitolato \\

    RQO4
        & Il manuale di utilizzo dovrà essere fornito in formato pdf ed in lingua italiana
        & Obbligatorio
        & Interno \\

    RQO5
        & Il manuale tecnico dovrà essere fornito in formato pdf ed in lingua italiana
        & Obbligatorio
        & Interno \\

    RQD6
        & Il codice sorgente dovrà essere disponibile su una repository pubblica su Github
        & Desiderabile
        & Capitolato \\

    RQO7
        & L'applicativo dovrà essere sviluppato seguendo quanto stabilito nel documento Norme di Progetto v1.0.0
        & Obbligatorio
        & Interno

\end{longtable}


\newpage

\subsection{Requisiti di vincolo}
\label{sub:requisiti_di_vincolo}

\rowcolors{2}{white!80!lightgray!90}{white}
\renewcommand{\arraystretch}{2}
\begin{longtable}[H]{>{\centering\bfseries}m{2cm} >{\centering}m{9cm} >{\centering}m{2.5cm} >{\centering\arraybackslash}m{2.5cm}}
    \caption{Requisiti di vincolo}%
    \label{tab:requisiti_di_vincolo} \\
    \rowcolor{lightgray}
    {\textbf{Requisito}} & {\textbf{Descrizione}} & {\textbf{Classificazione}} & {\textbf{Fonte}}  \\
    \endfirsthead%
    \rowcolor{lightgray}
    {\textbf{Requisito}} & {\textbf{Descrizione}} & {\textbf{Classificazione}} & {\textbf{Fonte}}  \\
    \endhead%
    \rowcolor{white}
    \multicolumn{4}{c}{\textit{Continua alla pagina successiva}}
    \endfoot%
    \endlastfoot%

    %CORPO TABELLA


\end{longtable}

\newpage

\subsection{Requisiti prestazionali}
\label{sub:requisiti_prestazionali}
% Quale algoritmo sarebbe?
Non sono stati individuati requisiti prestazionali obbligatori. Nel caso si decidesse di sviluppare il requisito RVD10, 
bisognerebbe fare ulteriori valutazioni sull'algoritmo utilizzato per il calcolo della correlazione tra dimensioni. Nel 
caso si importino dati con molte dimensioni i tempi di calcolo della correlazione aumentano di svariati ordini di 
grandezza, rendendo così necessario imporre vincoli di tipo prestazionale.

\newpage

\subsection{Tracciamento}
\label{sub:tracciamento}

\subsubsection{Fonte - Requisiti}
\label{sssec:fonte_requisiti}

\rowcolors{2}{white!80!lightgray!90}{white}
\renewcommand{\arraystretch}{2} % 
\begin{longtable}[H]{| >{\centering\bfseries}p{8cm} | >{\centering\arraybackslash}p{8cm} |}
    \caption{Fonte - Requisiti}
    \label{tab:fonte_requisiti}\\
    \hline
    \rowcolor{lightgray}
    \multicolumn{1}{| >{\centering\bfseries}m{8cm} |}{\textbf{Fonte}} 
    & \multicolumn{1}{>{\centering\arraybackslash}m{8cm} |}{\textbf{Requisiti}}  \\
    \hline
    \endfirsthead%
    \hline
    \rowcolor{lightgray}
    \multicolumn{1}{| >{\centering\bfseries}m{8cm} |}{\textbf{Fonte}} 
    & \multicolumn{1}{>{\centering\arraybackslash}m{8cm} |}{\textbf{Requisiti}}  \\
    \hline
    \endhead%
    \hline
    \rowcolor{lightgray!40}
    \multicolumn{2}{|c|}{\textit{Continua alla pagina successiva}} \\
    \hline
    \endfoot%
    \hline
    \endlastfoot%

    %CORPO DELLA TABELLA

    Capitolato 
        & \makecell{
            \rule{0pt}{2.6ex} \\
            RQO1 \\ 
            RQO2 \\
            RQO3 \\
            RQD6 \\
            RVO1 \\
            RVO2 \\
            RVO3 \\
            RVO4 \\
            RVO5 \\
            RVO6 \\
            RVO7 \\
            RVO8 \\
            RVO9 \\
            RVO10 \\
            \rule[-1.2ex]{0pt}{0pt}
        } \\

    Decisione interna 
        & \makecell{
            \rule{0pt}{2.6ex} \\
            RQO4 \\
            RQO5 \\
            RQO7 \\
            RVF11 \\
            \rule[-1.2ex]{0pt}{0pt}
        } \\



\end{longtable}

\newpage

\subsubsection{Requisito - Fonte}
\label{sssec:requisito_fonte}

\rowcolors{2}{white!80!lightgray!90}{white}
\renewcommand{\arraystretch}{2}
\begin{longtable}[H]{| >{\centering\bfseries}m{8cm} | >{\centering\arraybackslash}m{8cm} |}
    \caption{Requisito - Fonte}%
    \label{tab:requisito_fonte} \\
    \hline
    \rowcolor{lightgray}
    {\textbf{Requisiti}} & {\textbf{Fonte}}  \\
    \hline
    \endfirsthead%
    \hline
    \rowcolor{lightgray}
    {\textbf{Requisiti}} & {\textbf{Fonte}}  \\
    \hline
    \endhead%
    \hline
    \rowcolor{lightgray!40}
    \multicolumn{2}{|c|}{\textit{Continua alla pagina successiva}} \\
    \hline
    \endfoot%
    \hline
    \endlastfoot%

    %CORPO TABELLA

    RFO1 
        & \\
        % & \hyperref[UC]{ssub:uc.} \\

    RFO1.1
        & \\
        % & \hyperref[UC]{ssub:uc.} \\

    RFO1.2
    & \\
    % & \hyperref[UC]{ssub:uc.} \\

    RFD2
    & \\
    % & \hyperref[UC]{ssub:uc.} \\

    RFD3
    & \\
    % & \hyperref[UC]{ssub:uc.} \\

    RFO4
    & \\
    % & \hyperref[UC]{ssub:uc.} \\

    RFO4.1
    & \\
    % & \hyperref[UC]{ssub:uc.} \\

    RFO4.2
    & \\
    % & \hyperref[UC]{ssub:uc.} \\

    RFO4.3
    & \\
    % & \hyperref[UC]{ssub:uc.} \\

    RFO4.4
    & \\
    % & \hyperref[UC]{ssub:uc.} \\

    RFO5
    & \\
    % & \hyperref[UC]{ssub:uc.} \\

    RFO5.2.1
    & \\
    % & \hyperref[UC]{ssub:uc.} \\

    RFO5.2.2
    & \\
    % & \hyperref[UC]{ssub:uc.} \\

    RFO5.2.3
    & \\
    % & \hyperref[UC]{ssub:uc.} \\

    RFO5.3.1
    & \\
    % & \hyperref[UC]{ssub:uc.} \\

    RFO5.4.1
    & \\
    % & \hyperref[UC]{ssub:uc.} \\

    RFD6
    & \\
    % & \hyperref[UC]{ssub:uc.} \\

    RFD6.1
    & \\
    % & \hyperref[UC]{ssub:uc.} \\

    RFD6.2
    & \\
    % & \hyperref[UC]{ssub:uc.} \\

    RFD6.3
    & \\
    % & \hyperref[UC]{ssub:uc.} \\

    RFD6.4
    & \\
    % & \hyperref[UC]{ssub:uc.} \\

    RQO1
    & Capitolato \\

    RQO2
    & Capitolato \\

    RQO3
    & Capitolato \\

    RQO4
    & Descrizione interna \\

    RQO5
    & Descrizione interna \\

    RQD6
    & Capitolato \\

    RQO7
    & Descrizione interna \\

    RVO1
    & Capitolato \\

    RVO2
    & Capitolato \\

    RVO3
    & Capitolato \\

    RVO4
    & Capitolato \\

    RVO5
    & Capitolato \\

    RVO6
    & Capitolato \\

    RVO7
    & Capitolato \\

    RVO8
    & Capitolato \\

    RVD9
    & Capitolato \\

    RVD10
    & Capitolato \\

    RVF11
    & Descrizione interna \\

\end{longtable}

\newpage

\subsection{Riepilogo}
\label{sub:riepilogo}

\rowcolors{2}{white!80!lightgray!90}{white}
\renewcommand{\arraystretch}{2}
\begin{longtable}[H]{>{\centering\bfseries}m{3cm} >{\centering}m{3cm} >{\centering}m{3cm} >{\centering}m{3cm} >{\centering\arraybackslash}m{3cm}}
  \caption{Riepilogo}%
  \label{tab:riepilogo}                                                    \\
  \rowcolor{lightgray}
  {\textbf{Tipologia}} & {\textbf{Obbligatorio}} & {\textbf{Facoltativo}} & {\textbf{Desiderabile}} & {\textbf{Totale}} \\
  \endfirsthead%
  \rowcolor{lightgray}
  {\textbf{Tipologia}} & {\textbf{Obbligatorio}} & {\textbf{Facoltativo}} & {\textbf{Desiderabile}} & {\textbf{Totale}}   \\
  \endhead%
  \rowcolor{white}
  \multicolumn{5}{c}{\textit{Continua alla pagina successiva}}
  \endfoot%
  \endlastfoot%
  \textbf{Funzionale} & 21 & 1 & 50 & 71 \\
  \textbf{Qualità} & 6 & 0 & 1 & 7 \\
  \textbf{Vincolo} & 10 & 0 & 1 & 11 \\
\end{longtable}

\end{document}

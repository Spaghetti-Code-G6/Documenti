\documentclass[../analisi-dei-requisiti.tex]{subfiles}

\begin{document}
I requisiti sono riportati in una tabella i cui campi sono:
\begin{itemize}
    \item\emph{Identificativo}: I requisiti si identificano mediante l'espressione: R[Tipoliga][Vincolo][CodiceProgressivo].
    
    Ogni voce tra parentesi quadre rappresenta un valore relativo alle proprietà di un requisito:
    \begin{itemize}
       
        \item \emph{Tipologia}: La tipologia che può assumere, il cui valore che varia tra: Funzionale, Qualità, Vincoolo.
        
        \item \emph{Vincolo}: Il vincolo di obbligo che si ha su di esso, che varia tra: Obbligatorio, Desiderabile e Facoltativo.
        \item \emph{CodiceProgressivo}: Un numero progressivo per identificarlo univocamente.
    \end{itemize} 

    \item \emph{Descrizione}: Una breve descrizione della funzionalità.
    \item \emph{Classificazione}: Vincolo di obbligo che si ha sulla produzione del requisito.
    \item \emph{Fonti}:   Ogni requisito ha una fonte che è descritta dal metodo di individuamento tra
                    Capitolato, Interno e Verbale e l'eventuale UC che ne descrive la funzionalità.

\end{itemize}

\subsection{Requisiti di funzionalità}
\label{sub:requisiti_di_funzionalita}

\rowcolors{2}{white!80!lightgray!90}{white}
\renewcommand{\arraystretch}{2} %
\begin{longtable}[H]{>{\raggedright\bfseries}m{20mm} >{\raggedright}m{90mm} >{\raggedright}m{28mm} >{\raggedright\arraybackslash}m{30mm}}
    \caption{Requisiti funzionali}%
    \label{tab:requisiti_funzionali} \\
    \rowcolor{lightgray}
    \multicolumn{1} {>{\centering\bfseries}m{20mm}} {\textbf{Requisito}}
    & \multicolumn{1} {>{\centering}m{90mm}} {\textbf{Descrizione}}
    & \multicolumn{1} {>{\centering}m{28mm}} {\textbf{Classificazione}}
    & \multicolumn{1} {>{\centering\arraybackslash}m{30mm}} {\textbf{Fonte}} \\

    \endfirsthead%
    \rowcolor{lightgray}
    \multicolumn{1} {>{\centering\bfseries}m{20mm}} {\textbf{Requisito}}
    & \multicolumn{1} {>{\centering}m{90mm}} {\textbf{Descrizione}}
    & \multicolumn{1} {>{\centering}m{28mm}} {\textbf{Classificazione}}
    & \multicolumn{1} {>{\centering\arraybackslash}m{30mm}} {\textbf{Fonte}} \\
    \endhead%
    \rowcolor{lightgray!40}
    \multicolumn{4}{c}{\textit{Continua alla pagina successiva}} \\
    \endfoot%
    \endlastfoot%

    %CORPO TABELLA
    RFO1
    & L'utente deve poter inserire dati nel sistema
    & Obbligatorio
    & \makecell{
        Capitolato \\
        \hyperref[sub:uc1]{UC1}} \\

    RFO1.1
    & L'utente deve poter effettuare l'inserimento dei dati da file CSV
    & Obbligatorio
    & \makecell{
        Capitolato  \\
        \hyperref[ssub:uc1.1]{UC1.1}} \\

    RFO1.2
    & L'utente deve poter effettuare l'inserimento dei dati da database esterno
    & Obbligatorio
    & \makecell{
        Capitolato \\
        \hyperref[ssub:uc1.3]{UC1.3}} \\

    RFO1.3
    &   L'utente deve poter aprire un collegamento con un server per poter
        accedere ad uno dei suoi database
    & Obbligatorio
    & \makecell{ Interno \\  \hyperref[par:uc1.3.1]{UC1.3.1}}\\

    RFO1.4
    &   L'utente deve poter immettere l'indirizzo del server
    & Obbligatorio
    & \makecell{ Interno \\  \hyperref[spar:uc1.3.1.1]{UC1.3.1.1}}\\

    RFO1.5
    &   L'utente deve poter immettere il nome di accesso al server
    & Obbligatorio
    & \makecell{ Interno \\  \hyperref[spar:uc1.3.1.2]{UC1.3.1.2}}\\

    RFO1.6
    &   L'utente deve poter immettere la password di accesso al server
    & Obbligatorio
    & \makecell{ Interno \\  \hyperref[spar:uc1.3.1.3]{UC1.3.1.3}}\\

    % TODO: rivedere/confronto
    RFO1.7
    &   L'utente deve poter importare i dati nel sistema mediante la ricerca
        su un database tra quelli disponibili
    & Obbligatorio
    & \makecell{ Interno \\ \hyperref[par:uc1.3.2]{UC1.3.2}}\\

    RFO1.8
    &   L'utente deve poter inserire i metadati di tipo delle dimensioni del dataset
    & Obbligatorio
    & \makecell{ Interno \\ \hyperref[ssub:uc1.4]{UC1.4}}\\

    RFO2
    & L'utente deve poter selezionare la costruzione di un grafico di sua scelta
    & Obbligatorio
    & \makecell{ Capitolato \\ \hyperref[ssub:uc2.1]{UC2.1}}\\

    RFO2.1
    & L'utente deve poter selezionare l'opzione di costruzione di un grafico di tipo Scatter Plot
    Matrix
    & Obbligatorio
    & \makecell{ Capitolato \\   \hyperref[ssub:uc2.2]{UC2.2}}\\

    RFO2.2
    & L'utente deve poter selezionare l'opzione di costruzione di un grafico di tipo Force Field
    & Obbligatorio
    & \makecell{ Capitolato \\  \hyperref[ssub:uc2.3]{UC2.3}}\\

    RFD2.3
    & L'utente deve poter selezionare l'opzione di costruzione di un grafico di tipo Heat Map
    & Obbligatorio
    & \makecell{ Interno \\  \hyperref[ssub:uc2.4]{UC2.4}}\\

    RFO2.4
    & L'utente deve poter selezionare l'opzione di costruzione di un grafico di tipo PLMA
    & Obbligatorio
    & \makecell{Capitolato \\ \hyperref[ssub:uc2.5]{UC2.5}}\\

    RFO2.5
    & L'utente deve poter selezionare l'opzione di costruzione di un grafico di tipo Distance Map
    & Obbligatorio
    & \makecell{Capitolato \\ \hyperref[ssub:uc2.6]{UC2.6}}\\

    RFO3
    & L'utente deve poter modificare i metadati associati al dataset
    & Obbligatorio
    & \makecell{ Interno \\  \hyperref[sub:uc3]{UC3} }\\

    RFO3.1
    & L'utente deve poter selezionare di quale dimensione desidera modificare i metadati
    & Obbligatorio
    & \makecell{ Interno \\\hyperref[ssub:uc3.1]{UC3.1} }\\

    RFO3.2
    & L'utente deve poter modificare i metadati di tipo associati al dataset
    & Obbligatorio
    & \makecell{ Interno \\\hyperref[ssub:uc3.2]{UC3.2} }\\

    RFO3.3
    & L'utente deve poter modificare i metadati di visibilità delle dimensioni del dataset
    & Obbligatorio
    & \makecell{ Interno \\  \hyperref[ssub:uc3.3]{UC3.3} }\\

    % TODO: rivedere
    RFO3.3
    & L'utente deve poter annullare le modifiche fatte ai metadati
    & Desiderabile
    & \makecell{ Interno \\  \hyperref[ssub:uc3.4]{UC3.4} }\\

    % TODO: rivedere
    RFD4
    & L'utente deve poter modificare il grafico visualizzato
    & Desiderabile
    & \makecell{ Capitolato \\ \hyperref[sub:uc4]{UC4} }\\

    % TODO: rivedere
    RFD4.1
    & L'utente deve poter modificare le proprietà del grafico
    & Desiderabile
    & \makecell{ Capitolato \\ \hyperref[ssub:uc4.1]{UC4.1} }\\

    RFD4.2
    & L'utente deve poter modificare le proprietà della visualizzazione Scatter Plot Matrix
    & Desiderabile
    & \makecell{ Capitolato \\ \hyperref[ssub:uc4.2]{UC4.2} }\\

    RFD4.3
    & L'utente deve poter modificare il numero di dimensioni di rappresentabili dalla matrice di Scatter Plot
    & Desiderabile
    & \makecell{ Capitolato \\ \hyperref[par:uc4.2.1]{UC4.2.1} }\\

    RFD4.4
    & L'utente deve poter aggiungere un'ulteriore dimensione del dato alla matrice di Scatter Plot mediante colore
    & Desiderabile
    & \makecell{ Verbale \\ \hyperref[par:uc4.2.2]{UC4.2.2} }\\

    RFD4.5
    & L'utente deve poter aggiungere un'ulteriore dimensione del dato alla matrice di Scatter Plot mediante brillanza
    & Desiderabile
    & \makecell{ Verbale \\ \hyperref[par:uc4.2.3]{UC4.2.3} }\\

    RFD4.6
    & L'utente deve poter selezionare un punto in uno Scatter Plot della matrice e visualizzarlo evidenziato negli 
    altri Scatter Plot
    & Desiderabile
    & \makecell{ Interno \\ \hyperref[par:uc4.2.4]{UC4.2.4} }\\

    RFD4.7
    & L'utente deve poter selezionare un insieme di punti in uno Scatter Plot della matrice e visualizzarlo evidenziato 
    negli altri Scatter Plot
    & Desiderabile
    & \makecell{ Interno \\ \hyperref[par:uc4.2.5]{UC4.2.5} }\\

    % TODO: duplicato? + ref sbagliato
    RFD4.8
    & L'utente deve poter selezionare un punto rappresentante un dato in uno Scatter Plot della matrice e visualizzarlo 
    negli altri scatter plot.
    & Desiderabile
    & \makecell{ Interno \\ \hyperref[par:uc4.2.3]{UC4.2.3} }\\

    RFD4.9
    & L'utente deve poter modificare i grafici con la matrice delle distanze
    & Desiderabile
    & \makecell{ Verbale \\ \hyperref[ssub:uc4.3]{UC4.3} }\\

    % TODO: duplicare il requisito per ogni algoritmo offerto almeno inizialmente
    RFD4.10
    & L'utente deve poter scegliere l'algoritmo di calcolo delle distanze tra quelli implementati in HD Viz
    & Desiderabile
    & \makecell{ Interno \\ \hyperref[par:uc4.3.1]{UC4.3.1} }\\

    % TODO: duplicare il requisito per ogni preprocessing offerto
    RFD4.11
    & L'utente deve poter scegliere se normalizzare, standardizzare o non eseguire alcun tipo di preprocessing sui dati
    & Desiderabile
    & \makecell{ Verbale \\ \hyperref[par:uc4.3.2]{UC4.3.2} }\\

    RFD4.12
    & L'utente deve poter modificare l'influenza di una dimensione
    & Desiderabile
    & \makecell{ Interno \\ \hyperref[par:uc4.3.3]{UC4.3.3} }\\

    % TODO: remove?
    RFD4.13
    & L'utente deve poter scegliere quali dimensioni del dataset considerare per il calcolo delle distanze
    & Desiderabile
    & \makecell{ Interno \\ \hyperref[par:uc4.3.2]{UC4.3.4} }\\

    RFD4.14
    & L'utente deve poter modificare le proprietà della visualizzazione Force Field
    & Desiderabile
    & \makecell{ Capitolato \\ \hyperref[ssub:uc4.4]{UC4.4} }\\

    RFD4.15
    & L'utente deve poter trascinare i nodi visualizzati nella visualizzazione Force field
    & Desiderabile
    & \makecell{ Capitolato \\ \hyperref[par:uc4.4.1]{UC4.4.1} }\\

    RFD4.schifo
    & L'utente deve poter impostare il valore di soglia per le forze associate agli archi
    & Desiderabile
    & \makecell{ Verbale \\ \hyperref[par:uc4.4.2]{UC4.4.2} }\\

    RFD4.16
    & L'utente deve poter eliminare gli archi ai quali sono associate forze con valori al di fuori di una certa soglia 
    nella visualizzazione Force Field
    & Desiderabile
    & \makecell{ Verbale \\ \hyperref[par:uc4.4.2]{UC4.4.2} }\\

    RFD4.schifo
    & L'utente deve poter impostare il valore di soglia minimo per le forze associate agli archi
    & Desiderabile
    & \makecell{ Verbale \\ \hyperref[par:uc4.4.2]{UC4.4.3} }\\

    RFD4.17
    & L'utente deve poter eliminare gli archi ai quali sono associate forze inferiori ad una certa soglia
    & Desiderabile
    & \makecell{ Verbale \\ \hyperref[par:uc4.4.3]{UC4.4.3} }\\

    RFD4.schifo
    & L'utente deve poter impostare il valore di soglia massimo per le forze associate agli archi
    & Desiderabile
    & \makecell{ Verbale \\ \hyperref[par:uc4.4.2]{UC4.4.4} }\\

    RFD4.18
    & L'utente deve poter eliminare gli archi ai quali sono associate forze superiori ad una certa soglia
    & Desiderabile
    & \makecell{ Verbale \\ \hyperref[par:uc4.4.4]{UC4.4.4} }\\

    RFD4.19
    & L'utente deve poter scalare le forze di attrazione
    & Desiderabile
    & \makecell{ Interno \\ \hyperref[par:uc4.4.5]{UC4.4.5} }\\

    % NICE
    RFD4.20
    & L'utente deve poter modificare le proprietà della visualizzazione Distance Map
    & Desiderabile
    & \makecell{ Capitolato \\ \hyperref[ssub:uc4.5]{UC4.5} }\\

    % TODO: duplicare per ogni gradiente(?)
    RFD4.21
    & L'utente deve poter modificare il gradiente di colori tra quelli offerti da HD Viz
    & Desiderabile
    & \makecell{ Interno \\ \hyperref[par:uc4.5.1]{UC4.5.1} }\\

    RFO4.22
    & L'utente deve poter ordinare la Distance Map
    & Obbligatorio
    & \makecell{ Capitolato \\ \hyperref[par:uc4.5.2]{UC4.5.2} }\\

    RFO4.23
    & L'utente deve poter ordinare la Distance Map mediante clustering gerarchico
    & Obbligatorio
    & \makecell{ Capitolato \\ \hyperref[par:uc4.5.3]{UC4.5.3} }\\

    RFO4.24
    & L'utente deve poter associare un dendrogramma al clustering gerarchico nella Distance Map
    & Obbligatorio
    & \makecell{ Capitolato \\ \hyperref[par:uc4.5.2]{UC4.5.3} }\\

    RFD4.25
    & L'utente deve poter ripristinare l'ordinamento originario
    & Desiderabile
    & \makecell{ Interno \\ \hyperref[par:uc4.5.4]{UC4.5.4} }\\

    RFF4.26
    & L'utente deve poter ordinare le dimensioni rappresentate nella Distance Map per valore
    & Facoltativo
    & \makecell{ Interno \\ \hyperref[par:uc4.5.5]{UC4.5.5} }\\

    RFD4.27
    & L'utente deve poter modificare le etichette associate alla Distance Map
    & Desiderabile
    & \makecell{ Interno \\ \hyperref[par:uc4.5.6]{UC4.5.6} }\\

    RFD4.28
    & L'utente deve poter modificare le proprietà della visualizzazione Proiezione Lineare Multi Asse
    & Desiderabile
    & \makecell{ Interno \\  \hyperref[par:uc4.6.1]{UC4.6} }\\

    % TODO: remove?
    RFD4.29
    & L'utente deve poter aggiungere una dimensione dalla visualizzazione Proiezione Lineare Multi Asse
    & Desiderabile
    & \makecell{ Interno \\  \hyperref[par:uc4.6.1]{UC4.6.1} }\\

    % TODO: remove?
    RFD4.30
    & L'utente deve poter rimuovere una dimensione dalla visualizzazione Proiezione Lineare Multi Asse
    & Desiderabile
    & \makecell{ Interno \\  \hyperref[par:uc4.6.2]{UC4.6.2} }\\

    RFD4.31
    & L'utente deve poter spostare gli assi nella visualizzazione Proiezione Lineare Multi Asse
    & Desiderabile
    & \makecell{ Interno \\  \hyperref[par:uc4.6.3]{UC4.6.3} }\\

    RFD4.32
    & L'utente deve poter modificare la visualizzazione dell'Heat Map
    & Desiderabile
    & \makecell{ Interno \\  \hyperref[ssub:uc4.7]{UC4.7} }\\

    % TODO: duplicare?
    RFD4.33
    & L'utente deve poter modificare il gradiente dei colori nell'Heat Map scegliendo tra quelli proposti
    & Desiderabile
    & \makecell{ Interno \\  \hyperref[par:uc4.7.1]{UC4.7.1} }\\

    RFD4.34
    & L'utente deve poter modificare le etichette assocciate agli assi nell'Heat Map
    & Desiderabile
    & \makecell{ Interno \\  \hyperref[par:uc4.7.2]{UC4.7.2} }\\

    RFD4.35
    & L'utente deve poter modificare le etichette associate alle righe nell'Heat Map
    & Desiderabile
    & \makecell{ Interno \\  \hyperref[par:uc4.7.3]{UC4.7.3} }\\

    RFD4.36
    & L'utente deve poter modificare le etichette associate alle colonne nell'Heat Map
    & Desiderabile
    & \makecell{ Interno \\  \hyperref[par:uc4.7.4]{UC4.7.4} }\\

    RFD4.37
    & L'utente deve poter ordinare gli elementi dell'Heat Map
    & Desiderabile
    & \makecell{ Interno \\  \hyperref[par:uc4.7.5]{UC4.7.5} }\\

    RFD4.38
    & L'utente deve poter ordinare gli elementi dell'Heat Map mediante clustering gerarchico
    & Desiderabile
    & \makecell{ Interno \\  \hyperref[spar:uc4.7.5.6]{UC4.7.5.6} }\\

    RFD4.39
    & L'utente deve poter modificare l'ordinamento degli elementi ritornando all'ordine originale del dataset
    & Desiderabile
    & \makecell{ Interno \\  \hyperref[spar:uc4.7.5.7]{UC4.7.5.7} }\\

    RFD4.40
    & L'utente deve poter ripristinare la visualizzazione corrente ad uno stato pari a quello di creazione
    & Desiderabile
    & \makecell{ Interno \\  \hyperref[ssub:uc4.8]{UC4.8} }\\

\end{longtable}


\newpage

\subsection{Requisiti di qualità}
\label{sub:requisiti_di_qualita}

\rowcolors{2}{white!80!lightgray!90}{white}
\renewcommand{\arraystretch}{2} %
\begin{longtable}[H]{| >{\raggedright\bfseries}m{20mm} | >{\raggedright}m{90mm} | >{\centering}m{25mm} | >{\centering\arraybackslash}m{30mm}|}

    \hline
    \rowcolor{lightgray}
    \multicolumn{1} {| >{\centering\bfseries}m{20mm}| } {\textbf{Requisito}}
    & \multicolumn{1} {>{\centering}m{90mm}| } {\textbf{Descrizione}}
    & \multicolumn{1} {>{\centering}m{25mm}| } {\textbf{Importanza}}
    & \multicolumn{1} {>{\centering\arraybackslash}m{30mm}| } {\textbf{Fonte}} \\
    \hline
    
    \endfirsthead%
    
    \hline
    \rowcolor{lightgray}
    \multicolumn{1} {>{\centering\bfseries}m{20mm}| } {\textbf{Requisito}}
    & \multicolumn{1} {>{\centering}m{90mm}| } {\textbf{Descrizione}}
    & \multicolumn{1} {>{\centering}m{25mm}| } {\textbf{Importanza}}
    & \multicolumn{1} {>{\centering\arraybackslash}m{30mm}| } {\textbf{Fonte}} \\
    \hline
    
    \endhead%
    
    \hline
    \rowcolor{lightgray!40}
    \multicolumn{4}{|c|}{\textit{Continua alla pagina successiva}} \\
    \hline
    
    \endfoot%
    
    \endlastfoot%

    %CORPO TABELLA

    RQO1
        & L'applicativo deve essere accompagnato dalla documentazione minima richiesta per il corso di Ingegneria del software
        & Obbligatorio
        & Capitolato \\

    RQO2
        & L'applicativo dovrà essere accompagnato da un manuale di utilizzo
        & Obbligatorio
        & Capitolato \\

    RQO3
        & L'applicativo dovrà essere accompagnato da un manuale tecnico per inddicare come estendere l'applicazione
        & Obbligatorio
        & Capitolato \\

    RQO4
        & Il manuale di utilizzo dovrà essere fornito in formato pdf ed in lingua italiana
        & Obbligatorio
        & Interno \\

    RQO5
        & Il manuale tecnico dovrà essere fornito in formato pdf ed in lingua italiana
        & Obbligatorio
        & Interno \\

    RQD6
        & Il codice sorgente dovrà essere disponibile su una repository pubblica su Github
        & Desiderabile
        & Capitolato \\

    RQO7
        & L'applicativo dovrà essere sviluppato seguendo quanto stabilito nel documento Norme di Progetto v1.0.0
        & Obbligatorio
        & Interno \\
    \hline
    \rowcolor{white}
    \caption{Requisiti di qualità}%
    \label{tab:requisiti_di_qualita}
\end{longtable}


\newpage

\subsection{Requisiti di vincolo}
\label{sub:requisiti_di_vincolo}

\rowcolors{2}{white}{white!80!lightgray!90}
\renewcommand{\arraystretch}{2} %
\begin{longtable}[H]{|>{\raggedright\arraybackslash}p{20mm} | p{90mm} | p{22mm} | p{30mm} |}
    \caption{Requisiti di vincolo}%
    \label{tab:requisiti_di_vincolo} \\
    \hline
    \rowcolor{lightgray}
    \multicolumn{1}{| >{\centering\bfseries}m{20mm} |}{\textbf{Requisito}} 
    & \multicolumn{1}{ >{\centering}m{90mm} |}{\textbf{Descrizione}} 
    & \multicolumn{1}{ >{\centering}m{22mm} |}{\textbf{Rilevanza}} 
    & \multicolumn{1}{ >{\centering\arraybackslash}m{30mm} |}{\textbf{Fonte}} \\

    \endfirsthead%
    \hline
    \rowcolor{lightgray}
    \multicolumn{1}{| >{\centering\bfseries}m{20mm} |}{\textbf{Requisito}} 
    & \multicolumn{1}{ >{\centering}m{90mm} |}{\textbf{Descrizione}} 
    & \multicolumn{1}{ >{\centering}m{22mm} |}{\textbf{Rilevanza}} 
    & \multicolumn{1}{ >{\centering\arraybackslash}m{30mm} |}{\textbf{Fonte}} \\
    \hline
    \endhead%
    \hline
    \rowcolor{lightgray!40}
    \multicolumn{4}{|c|}{\textit{Continua alla pagina successiva}} \\
    \hline
    \endfoot%
    \hline
    \endlastfoot%
    %CORPO TABELLA

    RVO1
        & L'applicativo deve essere sviluppato in tecnologia HTML/CSS/Javascript
        & Obbligatorio
        & Capitolato \\

    RVO2
        & L'applicativo deve essere sviluppato utilizzando la libreria D3.js
        & Obbligatorio
        & Capitolato \\

    RVO3
        & La parte server dell'applicativo deve essere sviluppata in Java o in Node.js
        & Obbligatorio
        & Capitolato \\

    RVO4
        & L'applicativo deve poter visualizzare dati ad almeno 15 dimensioni
        & Obbligatorio
        & Capitolato \\

    RVO5
        & L'applicativo deve presentare la modalità di visualizzazione Scatter Plot Matrix
        & Obbligatorio
        & Capitolato \\

    RVO6
        & L'applicativo deve presentare la modalità di visualizzazione Force Field
        & Obbligatorio
        & Capitolato \\
    
    RVD7
        & L'applicativo deve presentare la modalità di visualizzazione Heat Map
        & Obbligatorio
        & Capitolato \\
    
    RVO8
        & L'applicativo deve presentare la modalità di visualizzazione Proiezione Lineare Multi Asse
        & Obbligatorio
        & Capitolato \\

    RVO9
        & L'applicativo deve presentare la modalità di visualizzazione Distance Map
        & Obbligatorio
        & Capitolato \\

    RVO10
        & L'applicativo deve fornire l'ordinamento dei punti mediante clustering gerarchico nel grafico Heat Map
        & Desiderabile
        & Capitolato \\

    RVD10
        & L'applicativo deve fornire un metodo di analisi automatica che permetta di individuare le dimensioni di particolare interesses
        & Desiderabile
        & Capitolato \\

\end{longtable}


\newpage

\subsection{Requisiti prestazionali}
\label{sub:requisiti_prestazionali}
% Quale algoritmo sarebbe?
Non sono stati individuati requisiti prestazionali obbligatori. Nel caso si decidesse di sviluppare il requisito RVD10, 
bisognerebbe fare ulteriori valutazioni sull'algoritmo utilizzato per il calcolo della correlazione tra dimensioni. Nel 
caso si importino dati con molte dimensioni i tempi di calcolo della correlazione aumentano di svariati ordini di 
grandezza, rendendo così necessario imporre vincoli di tipo prestazionale.

\newpage

\subsection{Tracciamento}
\label{sub:tracciamento}

\subsubsection{Fonte - Requisiti}
\label{sssec:fonte_requisiti}

\rowcolors{2}{white!80!lightgray!90}{white}
\renewcommand{\arraystretch}{2} %
\begin{longtable}[H]{| >{\centering\bfseries}p{8cm} | >{\centering\arraybackslash}p{8cm} |}
    \hline
    \rowcolor{lightgray}
    \multicolumn{1}{| >{\centering\bfseries}m{8cm} |}{\textbf{Fonte}}
    & \multicolumn{1}{>{\centering\arraybackslash}m{8cm} |}{\textbf{Requisiti}}  \\
    \hline
    \endfirsthead%
    \hline
    \rowcolor{lightgray}
    \multicolumn{1}{| >{\centering\bfseries}m{8cm} |}{\textbf{Fonte}}
    & \multicolumn{1}{>{\centering\arraybackslash}m{8cm} |}{\textbf{Requisiti}}  \\
    \hline
    \endhead%
    \hline
    \rowcolor{lightgray!40}
    \multicolumn{2}{|c|}{\textit{Continua alla pagina successiva}} \\
    \hline
    \endfoot%
    \hline
    \endlastfoot%

    %CORPO DELLA TABELLA

    Capitolato &
        \makecell{
            RFO1 \\
            RFO1.1 \\
            RFO1.2 \\
            RFO1.3 \\
            RFO2.1 \\
            RFO2.2 \\
            RFO2.3 \\
            RFO2.5 \\
            RFD2.6 \\
            RFD4 \\
            RFD4.1 \\
            RFD4.2 \\
            RFD4.2.1 \\
            RFD4.4 \\
            RFD4.4.1 \\
            RFD4.4.22 \\
            RFD4.5 \\
            RFO4.5.2 \\
            RFO4.5.3.1 \\
            RFO4.5.3.2 \\
            RQO1 \\
            RQO2 \\
            RQO3 \\
            RQD6 \\
            RVO1 \\
            RVO2 \\
            RVO3 \\
            RVO4 \\
            RVO5 \\
            RVO6 \\
            RVO7 \\
            RVO8 \\
            RVO9 \\
            RVO10 \\
            RVD11
        } \\

    Interno &
        \makecell{
            RFO1.3.1 \\
            RFO1.3.1.1 \\
            RFO1.3.1.2 \\
            RFO1.3.1.3 \\
            RFO1.3.2 \\
            RFO1.4 \\
            RFO2 \\
            RFO2.4 \\
            RFO3 \\
            RFO3.1 \\
            RFO3.2 \\
            RFO3.3 \\
            RFD4.2.4 \\
            RFD4.2.5 \\
            RFD4.3.1 \\
            RFD4.3.1.1 \\
            RFD4.3.1.2 \\
            RFD4.3.1.3 \\
            RFD4.3.1.4 \\
            RFD4.3.3 \\
            RFD4.4.5 \\
            RFD4.5.1 \\
            RFD4.5.1.1 \\
            RFD4.5.1.2 \\
            RFD4.5.1.3 \\
            RFO4.5.4 \\
            RFF4.5.5 \\
            RFD4.5.6 \\
            RFD4.6 \\
            RFD4.6.1 \\
            RFD4.6.2 \\
            RFD4.6.3 \\
            RFD4.7 \\
            RFD4.7.1 \\
            RFD4.7.1.1\\
            RFD4.7.1.2 \\
            RFD4.7.1.3 \\
            RFD4.7.2.4 \\
            RFD4.7.3 \\
            RFD4.7.4 \\
            RFD4.5 \\
            RFD4.7.6 \\
            RFD4.7.7 \\
            RFD4.8 \\
            RFO5 \\
            RFO6 \\
            RFO7 \\
            RFO8 \\
            RFO9 \\
            RFO10 \\
            RQO4 \\
            RQO5 \\
            RQO7 \\
            RVO12
        } \\

    Verbale &
        \makecell{
            RFD4.2.2 \\
            RFD4.2.3 \\
            RFD4.3 \\
            RFD4.3.2\\
            RFD4.3.2.1 \\
            RFD4.3.2.2 \\
            RFD4.3.2.3 \\
            RFD4.4.3.1 \\
            RFD4.4.2.2 \\
            RFD4.4.2.3 \\
            RFD4.4.3.4
        } \\

    \hyperref[sub:uc1]{UC1} & RFO1 \\

    \hyperref[ssub:uc1.1]{UC1.1} & RFO1.1 \\

    \hyperref[ssub:uc1.2]{UC1.2} &  RFO1.2\\

    \hyperref[ssub:uc1.3]{UC1.3} & RFO1.3 \\

    \hyperref[par:uc1.3.1]{UC1.3.1} & RFO1.3.1 \\

    \hyperref[spar:uc1.3.1.1]{UC1.3.1.1} & RFO1.3.1.1 \\

    \hyperref[spar:uc1.3.1.2]{UC1.3.1.2} & RFO1.3.1.2 \\

    \hyperref[spar:uc1.3.1.3]{UC1.3.1.3} & RFO1.3.1.3 \\

    \hyperref[par:uc1.3.2]{UC1.3.2} & RFO1.3.2 \\

    \hyperref[ssub:uc1.4]{UC1.4} & RFO1.4 \\



    \hyperref[sub:uc2]{UC2} &  RFO2\\

    \hyperref[ssub:uc2.1]{UC2.1} & RFO2.1 \\

    \hyperref[ssub:uc2.2]{UC2.2} & RFO2.2 \\

    \hyperref[ssub:uc2.3]{UC2.3} & RFO2.3 \\

    \hyperref[ssub:uc2.4]{UC2.4} & RFD2.4 \\

    \hyperref[ssub:uc2.5]{UC2.5} & RFO2.5 \\

    \hyperref[ssub:uc2.6]{UC2.6} & RFO2.6 \\



    \hyperref[sub:uc3]{UC3} & RFO3 \\

    \hyperref[ssub:uc3.1]{UC3.1} & RFO3.1 \\

    \hyperref[ssub:uc3.2]{UC3.2} & RFO3.2 \\

    \hyperref[ssub:uc3.3]{UC3.3} & RFO3.3 \\




    \hyperref[sub:uc4]{UC4} & RFD4 \\

    \hyperref[ssub:uc4.1]{UC4.1} & RFD4.1 \\

    \hyperref[ssub:uc4.2]{UC4.2} & RFD4.2 \\

    \hyperref[par:uc4.2.1]{UC4.2.1} & RFD4.2.1 \\

    \hyperref[par:uc4.2.2]{UC4.2.2} & RFD4.2.2 \\

    \hyperref[par:uc4.2.3]{UC4.2.3}  & RFD4.2.3 \\

    \hyperref[par:uc4.2.4]{UC4.2.4} & RFD4.2.4 \\

    \hyperref[par:uc4.2.5]{UC4.2.5} & RFD4.2.5 \\

    \hyperref[ssub:uc4.3]{UC4.3} & RFD4.3 \\

    \hyperref[par:uc4.3.1]{UC4.3.1} & \makecell{
        RFD4.3.1 \\ 
        RFD4.3.1.1 \\ 
        RFD4.3.1.2 \\ 
        RFD4.3.1.3 \\ 
        RFD4.3.1.4} \\

    \hyperref[par:uc4.3.2]{UC4.3.2} & \makecell{
        RFD4.3.2 \\ 
        RFD4.3.2.1 \\
        RFD4.3.2.2 \\
        RFD4.3.2.3} \\

    \hyperref[par:uc4.3.3]{UC4.3.3} & RFD4.3.3 \\


    \hyperref[ssub:uc4.4]{UC4.4} & RFD4.4 \\

    \hyperref[par:uc4.4.1]{UC4.4.1} & RFD4.4.1 \\

    \hyperref[par:uc4.4.2]{UC4.4.2} & RFD4.4.2 \\

    \hyperref[par:uc4.4.3]{UC4.4.3}  & \makecell{
        RFD4.4.2.1 \\ 
        RFD4.4.2.2} \\

    \hyperref[par:uc4.4.4]{UC4.4.4} & \makecell{
        RFD4.4.2.3 \\ 
        RFD4.4.2.4} \\

    \hyperref[par:uc4.4.5]{UC4.4.5} & RFD4.4.5 \\




    % NICE
    \hyperref[ssub:uc4.5]{UC4.5} & RFD4.5 \\

    \hyperref[par:uc4.5.1]{UC4.5.1}  & \makecell{
        RFD4.5.1 \\
        RFD4.5.1.1 \\
        RFD4.5.1.2 \\
        RFD4.5.1.3} \\

    \hyperref[par:uc4.5.2]{UC4.5.2} & RFO4.5.2\\

    \hyperref[par:uc4.5.3]{UC4.5.3} & \makecell{
        RFO4.5.3.1 \\
        RFO4.5.3.2} \\

    \hyperref[par:uc4.5.4]{UC4.5.4} & RFD4.5.4 \\

    \hyperref[par:uc4.5.5]{UC4.5.5} & RFF4.5.5 \\

    \hyperref[par:uc4.5.6]{UC4.5.6} & RFD4.5.6 \\

    \hyperref[ssub:uc4.6]{UC4.6} & RFD4.6 \\

    \hyperref[par:uc4.6.1]{UC4.6.1} & RFD4.6.1 \\

    \hyperref[par:uc4.6.2]{UC4.6.2} & RFD4.6.2 \\

    \hyperref[par:uc4.6.3]{UC4.6.3} & RFD4.6.3 \\


    \hyperref[ssub:uc4.7]{UC4.7} & RFD4.7 \\

    \hyperref[par:uc4.7.1]{UC4.7.1} & RFD4.7.1 \\

    \hyperref[par:uc4.7.2]{UC4.7.2} & RFD4.7.2 \\

    \hyperref[par:uc4.7.3]{UC4.7.3} & RFD4.7.3 \\

    \hyperref[par:uc4.7.4]{UC4.7.4} & RFD4.4.7.4 \\

    \hyperref[par:uc4.7.5]{UC4.7.5} & RFD4.7.5 \\

    \hyperref[spar:uc4.7.5.6]{UC4.7.5.6} & RFD4.7.5.6 \\

    \hyperref[spar:uc4.7.5.7]{UC4.7.5.7} & RFD4.7.5.7 \\

    \hyperref[ssub:uc4.8]{UC4.8} & RFD4.8 \\

    \hyperref[sub:uc5]{UC5} & RFD5 \\

    \hyperref[sub:uc6]{UC6} & RFD6 \\

    \hyperref[sub:uc7]{UC7} & RFD7 \\

    \hyperref[sub:uc8]{UC8} & RFD8 \\

    \hyperref[sub:uc9]{UC9} & RFD9 \\

    \hyperref[sub:uc10]{UC10} & RFD10 \\

    \caption{Fonte - Requisiti}
    \label{tab:fonte_requisiti}
\end{longtable}


\newpage

\subsubsection{Requisiti - Fonte}
\label{sssec:requisiti_fonte}

\rowcolors{2}{white!80!lightgray!90}{white}
\renewcommand{\arraystretch}{2}
\begin{longtable}[H]{>{\centering\bfseries}m{8cm} >{\centering\arraybackslash}m{8cm}}
    
    \rowcolor{lightgray}
    {\textbf{Requisiti}} & {\textbf{Fonte}}  \\
    \endfirsthead%
    \rowcolor{lightgray}
    {\textbf{Requisiti}} & {\textbf{Fonte}}  \\
    \endhead%
    \rowcolor{white}
    \multicolumn{2}{c}{\textit{Continua alla pagina successiva}}
    \endfoot%
    \endlastfoot%
    %CORPO TABELLA

    RFO1 & \hyperref[sub:uc1]{UC1} \\
    RFO1.1 & \hyperref[ssub:uc1.1]{UC1.1} \\
    RFO1.2 & \hyperref[ssub:uc1.2]{UC1.2} \\
    RFO1.3 & \hyperref[ssub:uc1.3]{UC1.3} \\
    RFO1.3.1 & \hyperref[par:uc1.3.1]{UC1.3.1} \\
    RFO1.3.1.1 & \hyperref[spar:uc1.3.1.1]{UC1.3.1.1} \\
    RFO1.3.1.2 & \hyperref[spar:uc1.3.1.2]{UC1.3.1.2} \\
    RFO1.3.1.3 & \hyperref[spar:uc1.3.1.3]{UC1.3.1.3} \\
    RFO1.3.2 & \hyperref[par:uc1.3.2]{UC1.3.2} \\
    RFO1.4 & \hyperref[ssub:uc1.4]{UC1.4} \\
    RFO2 & \hyperref[sub:uc2]{UC2} \\
    RFO2.1 & \hyperref[ssub:uc2.1]{UC2.1} \\
    RFO2.2 & \hyperref[ssub:uc2.2]{UC2.2} \\
    RFO2.3 & \hyperref[ssub:uc2.3]{UC2.3} \\
    RFO2.4 & \hyperref[ssub:uc2.4]{UC2.4} \\
    RF02.5 & \hyperref[ssub:uc2.5]{UC2.5} \\
    RFO2.6 & \hyperref[ssub:uc2.6]{UC2.6} \\
    RFO3 & \hyperref[sub:uc3]{UC3} \\
    RFO3.1 & \hyperref[ssub:uc3.1]{UC3.1} \\
    RFO3.2 & \hyperref[ssub:uc3.2]{UC3.2} \\
    RFO3.3 & \hyperref[ssub:uc3.3]{UC3.3} \\
    RFD4 & \hyperref[sub:uc4]{UC4} \\
    RFD4.1 & \hyperref[ssub:uc4.1]{UC4.1} \\
    RFD4.2 & \hyperref[ssub:uc4.2]{UC4.2} \\
    RFD4.2.1 & \hyperref[par:uc4.2.1]{UC4.2.1} \\
    RFD4.2.2 & \hyperref[par:uc4.2.2]{UC4.2.2} \\
    RFD4.2.3 & \hyperref[par:uc4.2.3]{UC4.2.3} \\
    RFD4.2.4 & \hyperref[par:uc4.2.4]{UC4.2.4} \\
    RFD4.2.5 & \hyperref[par:uc4.2.5]{UC4.2.5} \\
    RFD4.3 & \hyperref[ssub:uc4.3]{UC4.3} \\
    RFD4.3.1 & \hyperref[par:uc4.3.1]{UC4.3.1} \\
    RFD4.3.1.1 & \hyperref[par:uc4.3.1]{UC4.3.1} \\
    RFD4.3.1.2 & \hyperref[par:uc4.3.1]{UC4.3.1} \\
    RFD4.3.1.3 & \hyperref[par:uc4.3.1]{UC4.3.1} \\
    RFD4.3.1.4 & \hyperref[par:uc4.3.1]{UC4.3.1} \\
    RFD4.3.2 & \hyperref[par:uc4.3.2]{UC4.3.2} \\
    RFD4.3.2.1 & \hyperref[par:uc4.3.2]{UC4.3.2} \\
    RFD4.3.2.2 & \hyperref[par:uc4.3.2]{UC4.3.2} \\
    RFD4.3.2.3 & \hyperref[par:uc4.3.2]{UC4.3.2} \\
    RFD4.3.3 & \hyperref[par:uc4.3.3]{UC4.3.3} \\
    RFD4.4 & \hyperref[ssub:uc4.4]{UC4.4} \\
    RFD4.4.1 & \hyperref[par:uc4.4.1]{UC4.4.1} \\
    RFD4.4.2 & \hyperref[par:uc4.4.1]{UC4.4.2} \\
    RFD4.4.2.1 & \hyperref[par:uc4.4.3]{UC4.4.3} \\
    RFD4.4.2.2 & \hyperref[par:uc4.4.3]{UC4.4.3} \\
    RFD4.4.2.3 & \hyperref[par:uc4.4.4]{UC4.4.4} \\
    RFD4.4.3.4 & \hyperref[par:uc4.4.4]{UC4.4.4} \\
    RFD4.4.5 & \hyperref[par:uc4.4.5]{UC4.4.5} \\
    RFD4.5 & \hyperref[ssub:uc4.5]{UC4.5} \\
    RFD4.5.1 & \hyperref[par:uc4.5.1]{UC4.5.1} \\
    RFD4.5.1.1 & \hyperref[par:uc4.5.1]{UC4.5.1} \\
    RFD4.5.1.2 & \hyperref[par:uc4.5.1]{UC4.5.1} \\
    RFD4.5.1.3 & \hyperref[par:uc4.5.1]{UC4.5.1} \\
    RFO4.5.2 & \hyperref[par:uc4.5.2]{UC4.5.2} \\
    RFO4.5.3.1 & \hyperref[par:uc4.5.3]{UC4.5.3} \\
    RFO4.5.3.2 & \hyperref[par:uc4.5.3]{UC4.5.3} \\
    RFD4.5.4 & \hyperref[par:uc4.5.4]{UC4.5.4} \\
    RFF4.5.5 & \hyperref[par:uc4.5.5]{UC4.5.5} \\
    RFD4.5.6 & \hyperref[par:uc4.5.6]{UC4.5.6} \\
    RFD4.6 & \hyperref[ssub:uc4.6]{UC4.6} \\
    RFD4.6.1 & \hyperref[par:uc4.6.1]{UC4.6.1} \\
    RFD4.6.2 & \hyperref[par:uc4.6.2]{UC4.6.2} \\
    RFD4.6.3 & \hyperref[par:uc4.6.3]{UC4.6.3} \\
    RFD4.7 & \hyperref[ssub:uc4.7]{UC4.7} \\
    RFD4.7.1 & \hyperref[par:uc4.7.1]{UC4.7.1} \\
    RFD4.7.1.1 & \hyperref[par:uc4.7.1]{UC4.7.1} \\
    RFD4.7.1.2 & \hyperref[par:uc4.7.1]{UC4.7.1} \\
    RFD4.7.1.3 & \hyperref[par:uc4.7.1]{UC4.7.1} \\
    RFD4.7.2.4 & \hyperref[par:uc4.7.2]{UC4.7.2} \\
    RFD4.7.3 & \hyperref[par:uc4.7.3]{UC4.7.3} \\
    RFD4.7.4 & \hyperref[par:uc4.7.4]{UC4.7.4} \\
    RFD4.7.5 & \hyperref[par:uc4.7.5]{UC4.7.5} \\
    RFD4.7.5.6 & \hyperref[spar:uc4.7.5.6]{UC4.7.5.6} \\
    RFD4.7.5.7 & \hyperref[spar:uc4.7.5.7]{UC4.7.5.7} \\
    RFD4.8 & \hyperref[ssub:uc4.8]{UC4.8} \\
    RFO5 & \hyperref[sub:uc5]{UC5} \\
    RFO6 & \hyperref[sub:uc6]{UC6} \\
    RFO7 & \hyperref[sub:uc7]{UC7} \\
    RFO8 & \hyperref[sub:uc8]{UC8} \\
    RFO9 & \hyperref[sub:uc9]{UC9} \\
    RFO10 & \hyperref[sub:uc10]{UC10} \\
    RQO1 & Capitolato \\
    RQO2 & Capitolato \\
    RQO3 & Capitolato \\
    RQO4 & Interno \\
    RQO5 & Interno \\
    RQD6 & Capitolato \\
    RQO7 & Interno \\
    RVO1 & Capitolato \\
    RVO2 & Capitolato \\
    RVo3 & Capitolato \\
    RVO4 & Capitolato \\
    RVO5 & Capitolato \\
    RVO6 & Capitolato \\
    RVO7 & Capitolato \\
    RVO8 & Capitolato \\
    RVO9 & Capitolato \\
    RVO10 & Capitolato \\
    RVD11 & Capitolato \\
    RVO12 & Interno\\
    \caption{Requisiti - Fonte}%
    \label{tab:requisiti_fonte}
\end{longtable}


\newpage

\subsection{Riepilogo}
\label{sub:riepilogo}

\rowcolors{2}{white!80!lightgray!90}{white}
\renewcommand{\arraystretch}{2}
\begin{longtable}[H]{>{\centering\bfseries}m{3cm} >{\centering}m{3cm} >{\centering}m{3cm} >{\centering}m{3cm} >{\centering\arraybackslash}m{3cm}}
  \caption{Riepilogo}%
  \label{tab:riepilogo}                                                    \\
  \rowcolor{lightgray}
  {\textbf{Tipologia}} & {\textbf{Obbligatorio}} & {\textbf{Facoltativo}} & {\textbf{Desiderabile}} & {\textbf{Totale}} \\
  \endfirsthead%
  \rowcolor{lightgray}
  {\textbf{Tipologia}} & {\textbf{Obbligatorio}} & {\textbf{Facoltativo}} & {\textbf{Desiderabile}} & {\textbf{Totale}}   \\
  \endhead%
  \rowcolor{white}
  \multicolumn{5}{c}{\textit{Continua alla pagina successiva}}
  \endfoot%
  \endlastfoot%
  \textbf{Funzionale} & 21 & 1 & 50 & 71 \\
  \textbf{Qualità} & 6 & 0 & 1 & 7 \\
  \textbf{Vincolo} & 10 & 0 & 1 & 11 \\
\end{longtable}

\end{document}

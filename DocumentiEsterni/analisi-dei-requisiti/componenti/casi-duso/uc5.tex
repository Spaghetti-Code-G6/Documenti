\newpage
\subsection{UC5 - Visualizzazione Errore}
\label{subsec:uc5}

%TODO: Cambiare numerazione? Al momento ha senso da una parte ma non è giustificata qui visto che sottocasi non sono veri e propri.

\subsubsection{UC5.1.1 - Visualizzazione errore inserimento dati da file}
\label{subsec:uc5.1.1}
\begin{itemize}
    \item \textbf{Descrizione}: All'utente viene mostrato un messaggio d'errore al reperimento
                                dei dati dal file e continua ad utilizzare 
                                il software senza aver correttamente caricato un dataset valido.

    \item \textbf{Attore primario}: Utente.
    
    \item \textbf{Precondizione}:   Il caricamento di dati dal file.

    \item \textbf{Postcondizione}:  Viene visualizzato un messaggio di errore sul reperimento dei 
                                    dati che lo avvisa della mancata formazione di un dataset per il
                                    corretto utilizzo di HDviz.

\end{itemize}


\subsubsection{UC5.1.2 - Visualizzazione errore inserimento dati da database}
\label{subsec:uc5.1.2}
\begin{itemize}
    \item \textbf{Descrizione}: All'utente viene mostrato un messaggio d'errore al reperimento
                                dei dati dal database e continua ad utilizzare 
                                il software senza aver correttamente caricato un dataset valido.

    \item \textbf{Attore primario}: Utente.
    
    \item \textbf{Precondizione}:   Il caricamento di dati dal database fallisce.

    \item \textbf{Postcondizione}:  Viene visualizzato un messaggio di errore sul reperimento dei 
                                    dati che lo avvisa della mancata formazione di un dataset per il
                                    corretto utilizzo di HDviz.

\end{itemize}



\subsubsection{UC5.2.1 - Visualizzazione errore tipo dati per un grafo Force Field}
\label{subsec:uc5.1.2}
\begin{itemize}
    \item \textbf{Descrizione}: Dopo aver caricato un dataset non contentente campi categorici, 
                                l’utente seleziona la costruzione di “Force Field” ma 
                                la sua rappresentazione risulta impossibile.

    \item \textbf{Attore primario}: Utente.
    
    \item \textbf{Precondizione}:   è stato caricato un dataset che non contentiene campi categorici e l’utente 
                                    seleziona la costruzione di una visualizzazione “Force Field”.

    \item \textbf{Postcondizione}:  Viene visualizzato un messaggio di errore.
    
    \item \textbf{Scenario Principale}: 
    \begin{enumerate}
        \item L'utente seleziona la creazione di una visualizzazione per un "Force Field".
        \item L'utente viene informato dell'impossibilità di creare il grafico mediante un messaggio pop-up.
    \end{enumerate}
\end{itemize}



\subsubsection{UC5.2.3 - Visualizzazione errore tipo dati per una Heat Map}
\label{subsec:uc5.1.2}
\begin{itemize}
    \item \textbf{Descrizione}: Dopo aver caricato un dataset non contentente campi numerici, 
                                l’utente seleziona la costruzione di una "Heat Map" ma 
                                la sua rappresentazione risulta impossibile.

    \item \textbf{Attore primario}: Utente.
    
    \item \textbf{Precondizione}:   è stato caricato un dataset che non contentiene campi numerici e l’utente 
                                    seleziona la costruzione di una visualizzazione di una "Heat Map".

    \item \textbf{Postcondizione}:  Viene visualizzato un messaggio di errore.
    
    \item \textbf{Scenario Principale}: 
    \begin{enumerate}
        \item L'utente seleziona la creazione di una visualizzazione per una "Heat Map".
        \item L'utente viene informato dell'impossibilità di creare il grafico mediante un messaggio pop-up.
    \end{enumerate}
\end{itemize}
\newpage

\subsection{UC2 - Creazione di un grafico}
\label{sub:uc2}

%TODO: Add correct image
\begin{figure}[h]
    \centering
    \includegraphics[width=0.8\textwidth]{componenti/casi-duso/diagrammi/UC2.pdf}
    \caption{Diagramma rappresentante UC2}
    \label{fig:UC2}
\end{figure}


\begin{itemize}
    \item \textbf{Descrizione}: L’utente vuole procedere con la fase di esplorazione
                                dati mediante la visualizzazione del dataset
                                attraverso uno dei diversi grafici proposti dall’applicativo
                                che ne costruisce uno e lo visualizza.
	
    \item \textbf{Attore primario}: Utente;
    
    \item \textbf{Precondizione}:   In HDviz è stato creato un dataset valido.

    \item \textbf{Postcondizione}:  Viene calcolato il grafico della tipologia scelta dall'utente dai dati 
									dal dataset corrente. 
									
									Viene inoltre mostrata la nuova visualizzazione all'utente.

	\item \textbf{Scenario principale}:
		\begin{enumerate}
			\item L'utente seleziona l'opzione che desidera tra le tipologie di grafico. (UC2.1)
			\item HDviz visualizza il grafico ottenuto dalla costruzione della scelta dell'utente.
		\end{enumerate}
\end{itemize}

\subsubsection{UC2.1 Selezione grafico}
\label{ssub:UC2.1}
\begin{itemize}

	\item \textbf{Descrizione}: L’utente seleziona la tipologia di grafico che desidera costruire.

    \item \textbf{Attore primario}: Utente.

	\item \textbf{Precondizione}:   Un dataset è stato correttamente importato. 

									L'utente ha aperto il menu di creazione di un grafico.

    \item \textbf{Postcondizione}:  Viene calcolato il grafico della tipologia selezionata dall'utente.

	\item \textbf{Generalizzazioni:}:  L'utente seleziona il grafico desiderato tra:

		\begin{enumerate}
			
			\item Selezione di Scatterplot matrix (UC2.2).
			\item Selezione di Force Field (UC2.3).
			\item Selezione di Heat Map (UC2.4).
			\item Selezione di Proiezione Lineare Multiasse (UC2.6).
			\item Selezione di Distance Map (UC2.6)
			
		\end{enumerate}

\end{itemize}


\subsubsection{UC2.1 Selezione di Scatterplot matrix}
\label{ssub:UC2.2}
\begin{itemize}

    \item \textbf{Attore primario}: Utente;

    \item \textbf{Precondizione}:   L'utente ha selezionato la voce \emph{Scatterplot Matrix} dal menu di creazione di un grafico.

    \item \textbf{Postcondizione}:  Viene calcolato il grafico di tipo \emph{Scatterplot Matrix}.

	\item \textbf{Scenario Principale}: HDviz calcola uno \emph{Scatterplot Matrix} dal dataset corrente.
\end{itemize}


\subsubsection{UC2.3 Selezione di Force Field}
\label{ssub:UC2.3}
\begin{itemize}

    \item \textbf{Attore primario}: Utente;

    \item \textbf{Precondizione}:   L'utente ha selezionato la voce \emph{Force Field} dal menu di creazione di un grafico.

    \item \textbf{Postcondizione}:  Viene calcolato il grafico di tipo \emph{Force Field}.
	
	\item \textbf{Scenario Principale}: HDviz calcola un \emph{Force Field} dal dataset corrente.

\end{itemize}


\subsubsection{UC2.4 Selezione di Heat Map}
\label{ssub:UC2.4}
\begin{itemize}

    \item \textbf{Attore primario}: Utente;

	\item \textbf{Precondizione}:   L'utente ha selezionato la voce \emph{Heat Map} dal menu di creazione di un grafico.

    \item \textbf{Postcondizione}:  Viene calcolato il grafico di tipo \emph{Heat Map}.

	\item \textbf{Scenario Principale}: HDviz calcola una \emph{Heat Map} dal dataset corrente.

\end{itemize}


\subsubsection{UC2.5 Selezione di Proiezione Lineare Multiasse}
\label{ssub:UC2.5}
\begin{itemize}

    \item \textbf{Attore primario}: Utente;

    \item \textbf{Precondizione}:   L'utente ha selezionato la voce \emph{PLMA} dal menu di creazione di un grafico.

    \item \textbf{Postcondizione}:  Viene calcolato il grafico di tipo \emph{"Proiezione lineare multiasse"}.

	\item \textbf{Scenario Principale}: HDviz calcola una \emph{Proiezione Lineare Multiasse} dal dataset corrente.
\end{itemize}

\subsubsection{UC2.6 Selezione di Distance Map}
\label{ssub:UC2.6}
\begin{itemize}
	\item \textbf{Attore primario}:		Utente;
	\item \textbf{Precondizione}:		L'utente ha selezionato la voce \emph{Distance Map} dal menu di creazione di un grafico.
	\item \textbf{Postcondizione}:		Viene calcolato il grafico di tipo\emph{"Distance Map"}.
	\item \textbf{Scenario Principale}: HDviz calcola una  \emph{Distance Map} dal dataset corrente.
\end{itemize}